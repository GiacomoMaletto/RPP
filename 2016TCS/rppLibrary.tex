\section{A library of functions in $ \RPP $}
\label{section:A library of functions in TRRF}

% We introduce functions to  further simplify the constructions inside $\RPP$.
% Quite often, the functions we are going to introduce rely on auxiliary arguments 
% we dub as \emph{ancillary arguments\LP{/variables}{ *** se fai una ricerca in questo file NON TROVI mai la parola ``variable'', al piu trovi ``position'' che considero un eufimesmo comodo quanto rozzo, secondo me parlare di variabili alimenterebbe confusione *** }} or \emph{ancillae} in analogy to 
% \cite{thomsen2015lncs}, for instance. 
% Ancillae have two main purposes. Either they temporarily record  copies of values for later use and are set back 
% to their initial value at the end of a computation 
% or
% they eventually hold the result of a computation, if properly initialized.
% \LP{}{Adapting the nomenclature introduced in \cite{toffoli80lncs}, 
% we say that an ancilla is a \temporarychannel whenever its initial and final value are the same.}%%%
% \LR{}{*** Secondo me il commento in rosso contraddice quel che abbiamo detto giusto prima a proposito
% del fatto che le ancille servono per memorizzare il risultato di una computazione. *** }\LP{ ***
% Le ancillae sono argomenti ausiliari utili alla computazione, che hanno due usi:
% ancille-risultato e ancille-temporanee! Dato che secondo, ***}

We introduce functions to  further simplify programming inside $\RPP$ by relying on \temporarychannels,
 \emph{ancillary arguments} or \emph{ancillae}.
They are additional arguments that can:
(i) hold the result of the computation;
(ii) temporarily record copies of values for later use;
(iii) temporarily store intermediate results.
Without loss of generality, we use ancillae in a very disciplined way in order to simplify their understanding.
Tipically, ancillae are initialized to zero. We cannot forget (reversibly) the content of ancillae, 
so they have to be carefully considered in compositions albeit sometimes we neglect their contents
(tipically, just forwarding them). 
Notions that recall the meaning of ancillae are common to the studies on reversible and quantum computation. They are
``temporary'' lines, storages, channels, bits and variables \cite{toffoli80lncs}.
%Moreover, following the quantum literature sometimes the ``ancillary'' replaces the adjective ``temporary''  (e.g. see \cite{thomsen2015lncs}).
% (Some references are \cite{abdessaied16book,DeVos2010book,DeVos12rc,toffoli80lncs,thomsen2015lncs}).
%%%%%%%%%%%%%%%
\paragraph{General increment and decrement} 
Let $ i, j $ and $ m\in \Nat $ be distinct.
Two functions $ \rprInc_{j;i}, \rprDec_{j;i}\in \RPP^m $ exist such that:
\begin{align*}
\arraycolsep=1.4pt
\begin{array}{rcl}
 \left. {\scriptsize \begin{array}{r} x_1\\ \cdots \\ x_j\\ \cdots \\ x_i       \\ \cdots \\x_m\end{array}} \right[
 & \rprInc_{j;i} &
 \left] {\scriptsize \begin{array}{l} x_1\\ \cdots \\ x_j\\ \cdots \\ x_i + |x_j|\\ \cdots \\x_m \end{array} } \right.
\end{array}
&\qquad\textrm{and}&
\arraycolsep=1.4pt
\begin{array}{rcl}
 \left. {\scriptsize \begin{array}{r} x_1\\ \cdots \\ x_j\\ \cdots \\ x_i      \\ \cdots \\x_m\end{array}} \right[
 & \rprDec_{j;i} &
 \left] {\scriptsize \begin{array}{l} x_1\\ \cdots \\ x_j\\ \cdots \\ x_i - |x_j|\\ \cdots \\x_m \end{array} } \right.
\end{array}
\enspace.
\end{align*}
We define them as:
\begin{align*}
\rprInc_{j;i}  & := \rprIt{m}{j,\langle i\rangle }{\rprSucc} 
&&&
\rprDec_{j;i}  & := \rprIt{m}{j,\langle i\rangle}{\rprPred}
\enspace .
\end{align*}

%%%%%%%%%%%%%%%
\paragraph{A function that compares two integers}
Let $k,j,i, p,q$ be pairwise distinct such that $k,j,i, p,q \leq n$, for a given arity $ n\in\Nat $.
A function $ \rprLess_{i,j,p,q;k}\in\RPP^n $ exists that implements the following relation:
\begin{align*}
\arraycolsep=1.4pt
\begin{array}{rcl}
  \left. {\scriptsize 
          \begin{array}{r} 
             x_1\\ \cdots 
             \\[1.10mm]
             x_k
             \\[1.25mm] 
             \cdots \\
             x_n
          \end{array}} 
  \right[
 & \rprLess_{i,j,p,q;k} &
   \left] {\scriptsize 
          \begin{array}{l}
             x_1\\ \cdots 
           \\[-1mm]
            x_k
            + \begin{cases}
                1 & \textrm{ if } x_i <    x_j\\
                0 & \textrm{ if }      x_i \geq x_j
                \enspace .
            \end{cases}
           \\[-1.10mm]
           \cdots \\ x_n
          \end{array} 
           } 
   \right.
\end{array}
\end{align*}

The function $\rprLess_{i,j,p,q;k}$ is expected to behave in accordance with the intended meaning whenever the 
arguments $ x_p, x_q $ are initially contains $0$, viz. they are ancillae.
Tipically, $x_k$ will be used as ancilla initialized to zero and it is not a temporary argument.
If the value of $x_k$ is initially $ 0 $, then $\rprLess_{i,j,p,q;k}$ returns  $ 0 $ or $ 1 $ in $ x_k $ 
depending on the result of comparing the values $ x_i $ and $ x_j $.  
Both $x_p$ and $x_q$ serve to duplicate the values of $x_i$ and $x_j$, respectively.
The copies allow to circumvent the linearity constraints in presence of nested selections.

Let  $ \Set{j_1,\ldots, j_{n-5}} = \Set{1,\ldots, n}\setminus \Set{k,j,i, p,q} $. We define
% \begin{subequations}
%   \begin{align}
%     \rprLess_{i,j,p,q;k} := &\ \AUTOrprSwap{n}{k,p,q,i,j,j_1,\ldots,j_{n-5}}
%                                             {1,2,3,4,5,6  ,\ldots,n } \rprSeq\
%     \rprInc_{5;3} \ \rprSeq\  \rprInc_{4;2}    \      \rprSeq
%     \label{less:subeq1}  \\
%     &   (\mathsf{F} \  \rVert \ \rprId^{n-5})  \ \rprSeq 
%       \label{less:subeq2}\\
%     &
%     \rprInv{\big(\rprInc_{5;3} \ \rprSeq\  \rprInc_{4;2} \big) }
%     \rprSeq\ \rprInv{\left( \AUTOrprSwap{n}{k,p,q,i,j,j_1,\ldots,j_{n-5}}
%                                             {1,2,3,4,5,6  ,\ldots,n } \right)} 
%     \label{less:subeq3}
%     \\
%    \nonumber      \mbox{where}\qquad\quad\\  
%    \nonumber
%     \mathsf{F}     := &\
%     \mathsf{If}^{5}_{5,\langle1,2,3,4\rangle}   [
%           \mathsf{If}^{4}_{4,\langle1,2,3\rangle}  [\mathsf{SameSign}, \rprSucc^3_1,  \rprSucc^3_1 ]  \\
%     \nonumber   & \phantom{\mathsf{If}^{5}_{5,\langle1,2,3,4\rangle}\ }
%     ,\mathsf{If}^{4}_{4,\langle1,2,3\rangle}   [ \rprId^3 , \rprId^3 ,    \rprSucc^3_1   ]  \\
%     \nonumber   &\phantom{\mathsf{If}^{5}_{5,\langle1,2,3,4\rangle}\ }
%     ,\mathsf{If}^{4}_{4,\langle1,2,3\rangle} [\rprId^3 ,   \rprId^3  ,\mathsf{SameSign}]] \\[5mm]
%     \nonumber%\mathsf{BothPos}
%     \mathsf{SameSign} := &\ \rprDec_{2;3}
%                            \rprSeq \mathsf{If}^{3}_{3,\langle1,2\rangle}[\rprSucc^2_1,\rprId^2,\rprId^2]
%                            \rprSeq  \rprInv{(\rprDec_{2;3})}
%     % \nonumber
%     % \mathsf{BothNeg} := &\ %\rprInc_{2;3} 
%     %                        \rprDec_{2;3}
%     %                        \rprSeq \mathsf{If}^{3}_{3,\langle1,2\rangle}[\rprSucc^2_1, \rprId^2, \rprId^2 ] 
%     %                        \rprSeq \rprInv{ (\rprDec_{2;3}) }
%     \enspace .
%   \end{align}
% \end{subequations}

  \begin{align}
    \rprLess_{i,j,p,q;k} := &\ \AUTOrprSwap{n}{k,p,q,i,j,j_1,\ldots,j_{n-5}}
                                            {1,2,3,4,5,6  ,\ldots,n } \rprSeq\
    \rprInc_{5;3} \ \rprSeq\  \rprInc_{4;2}    \      \rprSeq
    \label{less:subeq1}  \\
    &   (\mathsf{F} \  \rVert \ \rprId^{n-5})  \ \rprSeq 
      \label{less:subeq2}\\
    &
    \rprInv{\big(\rprInc_{5;3} \ \rprSeq\  \rprInc_{4;2} \big) }
    \rprSeq\ \rprInv{\left( \AUTOrprSwap{n}{k,p,q,i,j,j_1,\ldots,j_{n-5}}
                                            {1,2,3,4,5,6  ,\ldots,n } \right)} 
    \label{less:subeq3}
      \end{align}
 where $\mathsf{F} $ is
$$\mathsf{If}^{5}_{5,\langle1,2,3,4\rangle}   \big[
          \mathsf{If}^{4}_{4,\langle1,2,3\rangle}  [\mathsf{SameSign}, \rprSucc^3_1,  \rprSucc^3_1 ]
            \pmb,\mathsf{If}^{4}_{4,\langle1,2,3\rangle}   [ \rprId^3 , \rprId^3 ,    \rprSucc^3_1   ] 
            \pmb,\mathsf{If}^{4}_{4,\langle1,2,3\rangle} [\rprId^3 ,   \rprId^3  ,\mathsf{SameSign}]\big]$$
and $\quad \mathsf{SameSign} := \ \rprDec_{2;3}
                           \rprSeq \mathsf{If}^{3}_{3,\langle1,2\rangle}[\rprSucc^2_1,\rprId^2,\rprId^2]
                           \rprSeq  \rprInv{(\rprDec_{2;3})}\quad$.
% \begin{align}
%     \mathsf{F}     := &\
%     \mathsf{If}^{5}_{5,\langle1,2,3,4\rangle}   [
%           \mathsf{If}^{4}_{4,\langle1,2,3\rangle}  [\mathsf{SameSign}, \rprSucc^3_1,  \rprSucc^3_1 ]  \\
%     \nonumber   & \phantom{\mathsf{If}^{5}_{5,\langle1,2,3,4\rangle}\ }
%     ,\mathsf{If}^{4}_{4,\langle1,2,3\rangle}   [ \rprId^3 , \rprId^3 ,    \rprSucc^3_1   ]  \\
%     \nonumber   &\phantom{\mathsf{If}^{5}_{5,\langle1,2,3,4\rangle}\ }
%     ,\mathsf{If}^{4}_{4,\langle1,2,3\rangle} [\rprId^3 ,   \rprId^3  ,\mathsf{SameSign}]] \\[5mm]
%     \nonumber%\mathsf{BothPos}
%     \mathsf{SameSign} := &\ \rprDec_{2;3}
%                            \rprSeq \mathsf{If}^{3}_{3,\langle1,2\rangle}[\rprSucc^2_1,\rprId^2,\rprId^2]
%                            \rprSeq  \rprInv{(\rprDec_{2;3})}
%     % \nonumber
%     % \mathsf{BothNeg} := &\ %\rprInc_{2;3} 
%     %                        \rprDec_{2;3}
%     %                        \rprSeq \mathsf{If}^{3}_{3,\langle1,2\rangle}[\rprSucc^2_1, \rprId^2, \rprId^2 ] 
%     %                        \rprSeq \rprInv{ (\rprDec_{2;3}) }
%     \enspace .
%   \end{align}

\noindent
The \rprSComName at the lines \eqref{less:subeq1} and \eqref{less:subeq3} are one the inverse of the other.
The leftmost function at line \eqref{less:subeq1} moves the five relevant 
arguments in the first five positions\footnote{
The order of the first five elements has been carefully devised in order to avoid index changes of 
arguments due to the removal of intermediate arguments to satisfy the  linearity constraint of 
the $\mathsf{If}$.}.
The rightmost function at line \eqref{less:subeq1} duplicates $x_i$ and $x_j$ into the ancillae
$x_p$ and $x_q$, respectively. The functions at line \eqref{less:subeq3} undo what 
those ones at line \eqref{less:subeq1} have done. In between them,
at line \eqref{less:subeq2} 
the function $ \mathsf{F} \  \rVert \ \rprId^{n-5} $ receives a given 
$\langle x_k,x_p+|x_i|,x_q+|x_j|,x_i,x_j,x_{j_1},\ldots,x_{j_{n-5}}\rangle$ 
as argument and operates on its first five elements.
We explain $ \mathsf{F}\in\RPP^5$ by the following case analysis scheme:
\begin{align*}
\begin{array}{rrcccl||l}
                 &                 & x_i > 0         & x_i = 0     & x_i < 0       &       \\
                 \hline\hline  % &                 &                 &             &                &  &         \\
\mathsf{If}^{5\phantom{|}}_{5,\langle1,2,3,4\rangle} 
      [&\mathsf{If}^{4}_{4,\langle1,2,3\rangle} [&\mathsf{SameSign},&  \rprSucc^3_1,   &   \rprSucc^3_1    &] & x_j > 0 \\
      ,&\mathsf{If}^{4}_{4,\langle1,2,3\rangle} [& \rprId^3,    &  \rprId^3,    &  \rprSucc^3_1   &] & x_j = 0 \\
      ,&\mathsf{If}^{4}_{4,\langle1,2,3\rangle} [&   \rprId^3,  &  \rprId^3,     &\mathsf{SameSign}&]]& x_j < 0 
\end{array}
\enspace .
\end{align*}
For example, 
let us assume $x_j <0$ and $x_j < 0$. The nested selections of $ \mathsf{F} $ eventually apply
$ \mathsf{SameSign} $ to a tuple of three elements $ \langle x_k, x_p+|x_j|, x_q+|x_i|\rangle $.
The effect of  $\rprDec_{2;3} $ is to update the third element of the 
tuple yielding $ \langle x_k, x_p+|x_j|, x_q+|x_i|-x_p-|x_j|\rangle $. So, whenever the initial
value of the temporary arguments $ x_p $ and $ x_q $ is $0$, we can deduce which among
$ x_i $ and $ x_j $ is greater than the other. The value of $ x_k $ is incremented only when
the difference is positive. $ \mathsf{SameSign} $ concludes by unfolding its local subtraction.


%%%%%%%%%%%%%%%
\paragraph{A function that multiplies two integers}
Let $k,j,i$ be pairwise distinct such that $k,j,i\leq n$, for any $ n\in\Nat $.
A function $ \rprMult_{k,j;i}\in\RPP^n  $ exists such that:
%\begin{align*}
%\arraycolsep=1.4pt
%\begin{array}{rcl}
%  \left. {\scriptsize 
%          \begin{array}{r} 
%             x_1\\ \cdots 
%             \\[1.10mm]
%             x_i
%             \\[1.25mm] 
%             \cdots \\
%             x_n
%          \end{array}} 
%  \right[
% & \rprMult_{k,j;i} &
%   \left] {\scriptsize 
%          \begin{array}{l}
%             x_1\\ \cdots 
%           \\[-3mm]
%            x_i +
%             \begin{cases}
%               \underbrace{(\phantom{-} x_j + \ldots + x_j)}_{\mid x_k\mid }
%                      & \textrm{ if } x_k, x_j \textrm{ have same sign}\\
%               \overbrace{(-x_j - \ldots - x_j)}
%                      & \textrm{ if } x_k, x_j \textrm{ have different sign}
%             \end{cases}
%           \\[-3mm]
%           \cdots \\ x_n
%          \end{array} 
%           } 
%   \right.
%\enspace.
%\end{array}
%\end{align*}
\begin{align*}
\arraycolsep=1.4pt
\begin{array}{rcl}
\left. {\scriptsize 
	\begin{array}{r} 
	x_1\\ \cdots 
	\\[1.10mm]
	x_i
	\\[1.25mm] 
	\cdots \\
	x_n
	\end{array}} 
\right[
& \rprMult_{k,j;i} &
\left] {\scriptsize 
	\begin{array}{l}
	x_1\\ \cdots 
	\\[-1mm]
	  (x_i + |x_k| \times |x_j| ) \times
	  \begin{cases}
	  \phantom{-}1  & \textrm{ if } x_k, x_j \textrm{ have same sign}\\
	  -1  &\textrm{ if } x_k, x_j \textrm{ have different sign}
	  \end{cases}
	\\[-1mm]
	\cdots \\ x_n
	\end{array} 
} 
\right.
\enspace.
\end{array}
\end{align*}
The function $ \rprMult_{k,j;i} $ behaves in accordance with the intended meaning whenever 
the ancilla $ x_i$ is initially set to $0$. In that case
$ \rprMult_{k,j;i} $ yields $ |x_j|\times |x_k| $ in position $ i $ which is multiplied by $ -1 $
if, and only if, $ x_j $ and $ x_k $ have different sign. We define:
\begin{align*}
\rprMult_{k,j;i} 
& = 
\rprIt{n}{k,\langle i,j\rangle}{ % \rprInc_{1;2}
	                            \rprInc_{2;1}
                               }
\rprSeq
\rprIf{n}{k,\langle i,j\rangle}
      {\rprIf{2}{1,\langle 2\rangle}
             {\rprId}{\rprId}{\rprNeg}}
      {\rprIf{2}{1,\langle 2\rangle}
             {\rprId}{\rprId}{\rprId}}
      {\rprIf{2}{1,\langle 2\rangle}
             {\rprNeg}{\rprId}{\rprId}} 
\end{align*}
which gives the result by first nesting the iterations of $ \rprInc_{2;1} $ inside an explicit iteration 
(acting on 3 arguments) and then setting the correct sign.

%%%%%%%%%%%%%%%%%%%%%%%%%%%%%%%%%%%%%%%%%%%%%%%%%%%%%%%%%%%%%%%%%%%%%%%%%%%%%%%%%%%%%%%%%%%%%%%%%%%%%%%%%%%%%
\paragraph{A function that encodes a bounded minimization}
Let $\mathsf{F}_{i;j} \in \RPP^n$ with $ i \neq j $ and $ i, j\leq n $. 
For any $\langle x_1,\ldots,x_i,\ldots, x_n\rangle$ and
any $y\in\Nat$, which we call \emph{range}, we look for the minimum integer 
$v$ such that, both $ v\geq 0$  and
$\mathsf{F}_{i;j}\langle x_1,\ldots, x_{i-1}, x_i + v, x_{i+1}, \ldots, x_n\rangle$ 
returns a negative value in position $j$, when a such $ v $ exists.
If $ v $ does not exist we  simply return $ |y| $.          
Summarizing, $ \lpMU{\mathsf{F}_{i;j}}\in\RPP^{n+4} $ is:
\par
{\hspace{-.05\textwidth}
\scalebox{.95}{\parbox{\linewidth}{%
\begin{align*}
    \arraycolsep=1.4pt
    \begin{array}{rcl}
      \left. {\scriptsize 
              \begin{array}{r} 
                 x_1\\ \cdots  \\ x_{n}
                 \\
                 x_{n+1}
                 \\
                 0 \\  0 \\  x_{n+4} 
              \end{array}} 
      \right[
     & \rotatebox{90}{$\hspace{-5mm}\lpMU{\mathsf{F}_{i;j}}$} &
       \left] {\scriptsize 
              \begin{array}{l}
                 x_1\\ \cdots  \\ x_{n}
               \\[-9mm]
                x_{n+1}
                + 
         \begin{cases}
           v
           & \textrm{whenever } 0\leq v\! <\!|x_{n+4}| \textrm{ is such that }\\
           &  \mathsf{F}_{i;j}\langle \ldots,x_{i-1},x_i + v,x_{i+1},\ldots \rangle 
                              = \langle \ldots,y_{j-1},z' ,y_{j+1},\ldots \rangle  \textrm{ and } z' < 0 \\
           & \textrm{and, for all } u \textrm{ such that } 0 \!\leq \! u\! < \! v ,\\
           &  \hfill \mathsf{F}_{i;j} \langle \ldots,x_{i-1},x_i +u,x_{i+1},\ldots \rangle
                              = \langle \ldots,y_{j-1},z,y_{j+1},\ldots \rangle \textrm{ and } z \geq 0; \\[3mm]
           | x_{n+4} |
           & \textrm{otherwise.}\\
         \end{cases}
               \\[-9mm]
                  0 \\  0 \\  x_{n+4} 
              \end{array}
               } 
       \right.
    \end{array}
    \end{align*}}}}
\par\noindent
The function $ \lpMU{\mathsf{F}_{i;j}} $ behaves in accordance with the here above specification when:
(i) the arguments $x_{n+1},x_{n+2}$ and $ x_{n+3} $, that we use as temporary arguments, are set to $ 0 $, and
(ii) $x_{n+4}$ is set to contain the range $ y $.
We plan to use the ancilla $x_{n+3}$ as flag,  the ancilla $x_{n+2}$ as a counter from $0$ to $x_{n+4}$ and,
the ancilla $x_{n+1}$ as store for $v$ (or $0$).
We define: % $\lpMU{\mathsf{F}_{i;j}}$ as follows:
     $$\begin{array}{l}
    \lpMU{\mathsf{F}_{i;j}} :=\\[3mm]
    \qquad
     \rprIt{}
           {}
           {  (\rprPCom{\mathsf{F}_{i;j}}{ \rprId^{3}}) \rprSeq
             \mathsf{If}^{n+3}
                   _{j\!,\scalebox{.59}{$\langle n\!+\!1\!,\!n\!+\!2\!,n\!+\!3\rangle$}\!}
              \big[{\rprId^3\!},
                   {\rprId^3\!},
                   {(\rprIf{}
                          {}
                          {\rprId^2\!}
                          {\rprInc_{2;1}}
                          {\rprId^2\!}
                    \rprSeq
                    \rprSucc_3^{3})
                   }\big] \rprSeq
             \rprInv{(\rprPCom{\mathsf{F}_{i;j}}{ \rprId^{3}})} \rprSeq
             (\rprSucc_i^{n} \rVert % \rprSucc_{n+2}
                                   \rprSucc^{3}_{2}
             )
             }
   \rprSeq  \\[3mm]
    \qquad
%     &
     \rprIf{n+4}
           {n+3,\langle n+1,n+4\rangle}
           {\rprId}
           {\rprInc_{2;1}\,}
           {\rprId}
  \rprSeq    \\[3mm]
%     & \ 
    \qquad
\rprInv{\left(
   \rprIt{}
           {}
           {  (\rprPCom{\mathsf{F}_{i;j}}{ \rprId^{3}}) \rprSeq
             \mathsf{If}^{n+3}
                   _{j\!,\scalebox{.59}{$\langle n\!+\!1\!,\!n\!+\!2\!,n\!+\!3\rangle$}\!}
              \big[{\rprId^3\!},
                   {\rprId^3\!},
                   { \rprSucc_3^{3})
                   }\big] \rprSeq
             \rprInv{(\rprPCom{\mathsf{F}_{i;j}}{ \rprId^{3}})} \rprSeq
             (\rprSucc_i^{n} \rVert \rprSucc^{3}_{2}
             
             )
             }
  \right)
}
    \enspace .
    %\end{aligned}
    \end{array}$$
 %  \end{minipage}
 % } % scalebox
 % \end{center}


\noindent The first line of the here above definition iterates its whole argument as many times as specified by 
the value of the range in $ x_{n+4} $:
\par
\resizebox{.98\textwidth}{!}{
\hspace*{-5mm}\begin{minipage}{\textwidth}
\begin{align*}
&
(\rprPCom{\mathsf{F}_{i;j}}{ \rprId^{3}}) \rprSeq
\mathsf{If}^{n+3}
_{j\!,\scalebox{.59}{$\langle n\!+\!1\!,\!n\!+\!2\!,n\!+\!3\rangle$}\!}
\big[{\rprId^3\!},
{\rprId^3\!},
{(\rprIf{}
	{}
	{\rprId^2\!}
	{\rprInc_{2;1}}
	{\rprId^2\!}
	\rprSeq
	\rprSucc_3^{3})
}\big] \rprSeq
\rprInv{(\rprPCom{\mathsf{F}_{i;j}}{ \rprId^{3}})} \rprSeq
(\rprSucc_i^{n} \rVert % \rprSucc_{n+2}
\rprSucc^{3}_{2}
) & 1\textrm{-st step,} \\
& \vdots \\
&
\rprSeq(\rprPCom{\mathsf{F}_{i;j}}{ \rprId^{3}}) \rprSeq
\mathsf{If}^{n+3}
_{j\!,\scalebox{.59}{$\langle n\!+\!1\!,\!n\!+\!2\!,n\!+\!3\rangle$}\!}
\big[{\rprId^3\!},
{\rprId^3\!},
{(\rprIf{}
	{}
	{\rprId^2\!}
	{\rprInc_{2;1}}
	{\rprId^2\!}
	\rprSeq
	\rprSucc_3^{3})
}\big] \rprSeq
\rprInv{(\rprPCom{\mathsf{F}_{i;j}}{ \rprId^{3}})} \rprSeq
(\rprSucc_i^{n} \rVert % \rprSucc_{n+2}
\rprSucc^{3}_{2}
)  & k\textrm{-th step,} \\
& \vdots \\
&
\rprSeq(\rprPCom{\mathsf{F}_{i;j}}{ \rprId^{3}}) \rprSeq
\mathsf{If}^{n+3}
_{j\!,\scalebox{.59}{$\langle n\!+\!1\!,\!n\!+\!2\!,n\!+\!3\rangle$}\!}
\big[{\rprId^3\!},
{\rprId^3\!},
{(\rprIf{}
	{}
	{\rprId^2\!}
	{\rprInc_{2;1}}
	{\rprId^2\!}
	\rprSeq
	\rprSucc_3^{3})
}\big] \rprSeq
\rprInv{(\rprPCom{\mathsf{F}_{i;j}}{ \rprId^{3}})} \rprSeq
(\rprSucc_i^{n} \rVert % \rprSucc_{n+2}
\rprSucc^{3}_{2}
)  & x_{n+4}\textrm{-th step;} \\
\end{align*}
\end{minipage}
} %resizebox
\par\noindent
let us assume that the $ x_j $ is negative at step $ k$.
We have two cases. 
\begin{itemize}
	\item 
The first case is with $ x_{n+3} $ equal to 0.
By design, this means that $ x_j $ has never become negative before. 
So, $ x_i $ contains the value $ v $ we are looking for. The firs step of:
\begin{align}
\label{align:min inner if}
\rprIf{}{}{\rprId^2\!}{\rprInc_{2;1}}{\rprId^2\!}\rprSeq\rprSucc_3^{3}
\end{align}
is to store $ x_{n+2} $ into $ x_{n+1} $. 
The reason is that, by design, the increments applied to  $ x_{n+2} $  and  $ x_i $ are   always stepwise aligned.
The second step of~\eqref{align:min inner if} is to increment $ x_{n+3} $, preventing
any further change of the value of $ x_{n+1} $ by the subsequent steps of the iteration.
	\item 
The second case is with $ x_{n+3} $ different from 0.
By design, this means that $ x_j $ has become negative in some step before the current $ k $-th one.
So, \eqref{align:min inner if} skips $ 	\rprInc_{2;1} $. The increment $ \rprSucc^{3}_{2} $
that operates on $ x_{n+3} $ is harmless because it just reinforces the idea that the value
$ v $ we were looking for is ``frozen'' in $ x_{n+1} $.
\end{itemize}
After any occurrence of~\eqref{align:min inner if} it is necessary to unfold the last application of 
$\mathsf{F}_{i;j}$ by means of $ \rprInv{\mathsf{F}_{i;j}} $.
Then, increasing $ x_i $ and $ x_{n+2} $ supplies a new value to $\mathsf{F}_{i;j}$ 
and keeps $ x_{n+2} $ aligned with $ x_i $.

\bigskip

After the first iteration completes, the evaluation follows with the second line
$\quad %\begin{align*}
\rprIf{n+4}
{n+3,\langle n+1,n+4\rangle}
{\rprId}
{\rprInc_{2;1}\,}
{\rprId}
\rprSeq
\quad$ %\end{align*}
in the definition of $ \lpMU{\mathsf{F}_{i;j}} $. It takes care of two cases.
\begin{itemize}
	\item 
	The value of $ x_{n+3} $ is not zero: somewhere in the course of the iteration
    $\mathsf{F}_{i;j}$ became negative. We have set $ x_{n+1} $ in accordance with that.	
	\item 
	The value of $ x_{n+3} $ is zero: $\mathsf{F}_{i;j}$  became negative in the course of the iteration.
	By definition, $ \rprInc_{2;1} $ sets $ x_{n+1} $ to the value of the range in $ x_{n+4} $.
\end{itemize}
The last line of $ \lpMU{\mathsf{F}_{i;j}} $ undoes what the first line did,
without altering the values of $ x_{n+1} $. This is why 
$ \rprIf{}{}{\rprId^2\!}{\rprInc_{2;1}}{\rprId^2\!} $ is missing.
In fact, the last line of $ \lpMU{\mathsf{F}_{i;j}} $
is an application of Bennett's trick \cite{bennett73ibm} in our functional programming setting.
 % \LR

%%%\par
%%%We conclude by formalizing a notation for a simple generalization of $ \lpMU{\mathsf{F}_{i;j}} $. 
%%%Let $ k,r \leq m \in\Nat $, with $ m \geq n+4 $ and $ r\neq k $.
%%%By definition $ \rprBMu{r}{k}{\mathsf{F}}{i}{j}\in\RPP^{m}$ is:
%%%    \begin{align*}
%%%    \arraycolsep=1.4pt
%%%    \begin{array}{rcl}
%%%      \left. {\scriptsize 
%%%              \begin{array}{r} 
%%%                 x_1\\ \cdots \\[6.5mm] x_{k} \\[7.5mm] \cdots  \\ x_{m}
%%%              \end{array}} 
%%%      \right[
%%%     & \rotatebox{90}{$\hspace{-5mm}\rprBMu{r}{k}{\mathsf{F}}{i}{j}$} &
%%%       \left] {\scriptsize 
%%%              \begin{array}{l}
%%%                 x_1\\ \cdots
%%%                    \\[-5mm] x_{k}
%%%                + 
%%%                \begin{cases}
%%%                  v
%%%                  & \textrm{whenever } 0\leq v\! <\!|x_{r}| \textrm{ is such that}\\
%%%                  &  \mathsf{F}_{i;j}\langle \ldots,x_{i-1},x_i + v,x_{i+1},\ldots \rangle 
%%%                                     = \langle \ldots,y_{j-1},z' ,y_{j+1},\ldots \rangle 
%%%                  \textrm{ and } z' < 0  \\
%%%                  & \textrm{and, for all } u \textrm{ such that } 0 \!\leq \! u\! < \! v ,\\
%%%                  &   \mathsf{F}_{i;j} \langle \ldots,x_{i-1},x_i + u,x_{i+1},\ldots \rangle
%%%                                     = \langle \ldots,y_{j-1},z,y_{j+1},\ldots \rangle
%%%                                     \textrm{ and } z \geq 0; \\\\\\
%%%                  | x_{r} |
%%%                  & \textrm{otherwise \enspace .}
%%%                \end{cases}
%%%               \\[-2mm] \cdots \\
%%%               x_{m}
%%%              \end{array}
%%%               } 
%%%       \right.
%%%    \end{array}
%%%    \end{align*}
%%%In it, the value $ v $ needs not to be stored in the argument just after the $ n $ arguments of $ \mathsf{F}_{i;j} $.
%%%Any $ x_k $ can contain it. Moreover, the argument that contains the range is not necessary the last one, as in the definition
%%%of $ \mathsf{F}_{i;j} $.

%%%%%%%%\paragraph{Replication trees in $ \RPP $}
%%%%%%%%For every $ n\geq 1 $ and $ x $, a function $ \rprNabla^n\in\RPP^{n+1}$ exists such that:
%%%%%%%%\begin{align*}
%%%%%%%%\arraycolsep=1.4pt
%%%%%%%%\begin{array}{rcl}
%%%%%%%% \left. {\scriptsize \begin{array}{r} 
%%%%%%%%                       x_1 \\ \ldots \\ x_n  \\ x
%%%%%%%%                     \end{array}} \right[
%%%%%%%% & \rprNabla^n &
%%%%%%%% \left] {\scriptsize \begin{array}{l}
%%%%%%%%                       x_1 + x\\ \ldots \\ x_n + x \\ x
%%%%%%%%                     \end{array} } \right.
%%%%%%%% \enspace .
%%%%%%%%\end{array}
%%%%%%%%\end{align*}
%%%%%%%%With $ x_1,\ldots,x_n $ set to $ 0 $, then $ \rprNabla^n \la x_1,\ldots,x_n\ra $ yields
%%%%%%%%$ n $ replicas of the value $ x $. By definition:
%%%%%%%%\begin{align*}
%%%%%%%%\rprNabla^n & =
%%%%%%%%\begin{cases}
%%%%%%%%\rprInc^{2}_{2;1}
%%%%%%%%& \text{if } n = 1
%%%%%%%%\\\\
%%%%%%%%(\rprPCom{\rprInc^{2}_{2;1}}
%%%%%%%%         {\rprId^{n-1}}
%%%%%%%%) \rprSeq \cdots
%%%%%%%%\\
%%%%%%%%\cdots \rprSeq
%%%%%%%%(\rprPCom{\rprId^{i}}
%%%%%%%%         {\rprPCom{\rprInc^{2}_{2;1}}
%%%%%%%%                  {\rprId^{n-i}}}
%%%%%%%%) \rprSeq \cdots 
%%%%%%%%\\
%%%%%%%%\qquad \qquad 
%%%%%%%%\cdots \rprSeq
%%%%%%%%(\rprPCom{\rprId^{n-1}}
%%%%%%%%         {\rprInc^{2}_{2;1}}
%%%%%%%%) 
%%%%%%%%& \text{if } n \geq 2, \text{for every } 1 \leq i < n
%%%%%%%%\enspace .
%%%%%%%%\end{cases}
%%%%%%%%\end{align*}




%%%%%%%%%%%%%%%%%%%%%%%%% LucaP:emacs-configuration
%%% Local Variables:
%%% mode: latex
%%% TeX-master: "main.tex"
%%% ispell-local-dictionary: "american"
%%% End:
%%%%%%%%%%%%%%%%%%%%%%%%% LucaP:emacs-configuration