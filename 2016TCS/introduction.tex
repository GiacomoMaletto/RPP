\section{Introduction}\label{section:Introduction}

Mainstream models of computations focus on one of the two possible directions of computation. 
We typically think how to model 
the way the computation evolves from inputs to outputs while 
we (reasonably) overlook the behavior in the opposite direction, from outputs to inputs. 
Generally speaking, models of computations are neither backward deterministic nor reversible.

For a very rough intuition about what reversible computation deals with, we start by an example.
Let us think about our favorite programming language and think of implementing the
addition between two natural numbers
$ m $ and $ n $. Very likely --- of course without absolute certainty --- we 
shall end up with some procedure \texttt{sum} which takes two arguments and which yields their sum. 
For example, \texttt{sum}($ 3 $,$ 2 $) would yield $ 5 $.
What if we think of asking for the values $ m $ and $ n $ such that \texttt{sum}($ m $,$ n $) = $ 5 $?
If we had implemented \texttt{sum} in a prolog-like language, then we could exploit its 
non deterministic evaluation mechanism to list all the pairs $ (0,5), (1,4), (2,3), (3,2), (4,1) $ and
$ (5,0) $ every of which would well be a correct instance of ($ m, n $). 
In a reversible setting we would obtain
exactly the pair we started from. \Ie, the computation would be backward deterministic.
The interest for the backward determinism arose in the sixties of the last century, 
studying the thermodynamics of computation and the connections between information theory, computability and entropy. 
Since then, the interest for the reversible computing has slowly but incessantly grown.

A forcefully non-exhaustive list of references follows.
Reversible Turing machines are defined in \cite{axelsen11lncs,bennett73ibm,jacopini90siam,lecerf63} while
\cite{axelsen11lncs,axelsen16acta,jacopini89tcs,li1996royal} 
study some of their computational theoretic properties.
Many research efforts have been focused on  boolean functions and cellular 
automata as models of the reversible
computation \cite{Morita2008101,toffoli80lncs}.
Moreover, some research focused on reversible programming languages \cite{DBLP:conf/popl/JamesS12,yokoyama08acm}.
Of course, the interest to build a comprehensive knowledge about reversible computation is wider than 
the mere interest for its computational theoretic aspects. 
The book \cite{perumalla2013chc} is a comprehensive 
introduction to the subject. It spans from low-power-consumption reversible hardware to emerging alternative computational
models like quantum  \cite{guerriniMM15,zorzi14mscs} or bio-inspired \cite{giannini2015tcs} of which reversibility is one the unavoidable building blocks.

\paragraph{Goal} 
The focus of this work is on a \emph{functional model} of reversible computation. 
Specifically, we look for a  sufficiently expressive  class of recursive permutations able to represent all
 Primitive Recursive Functions ($ \PRF $) \cite{rogers1967theory,soare1987book}
whose relevance is sometimes traced to the slogan
``programs which terminate but do not belong to $ \PRF $ are rarely of practical interest.''


We aim at formalizing a language that we identify as Reversible Primitive Permutations ($ \RPP $) and which 
retains the more interesting aspects of $\PRF$.
In analogy to $ \PRF $, our goal is to get a functional characterization of computability
-- but in a reversible setting, of course --- because functions are handier in order to compose 
algorithms. Other models, like, for example, Turing machines-oriented ones are
more convincing from an implementation view-point.
The ability to represent Cantor pairing \cite{rosenberg2009book} is one of the main properties that 
the functional characterization we are looking for must satisfy.
With Cantor pairing available it is possible to express all interesting total properties about the 
traces\footnote{Kleene's $T_n$ predicate, Kleene's normal form theorem and technical tools related to them \cite{cutland1980book,odifreddi1989book,soare1987book}.}
of Turing machines, reversible or not.
The other must-have property of our functional characterization is closure under inversion, which is something
very natural to ask for in a class of permutations and of reversible computing models.

Our quest is challenging because various negative results could have played  against it.
First of all, we remark that the class of all (total) recursive permutations 
cannot be effectively enumerated (see \cite[Exercise 4-6, p.55]{rogers1967theory}). 
In \cite{koz1972ail} a constructive generation of all the recursive permutations is given starting
from primitive recursive permutations. Since no enumeration exists for the latter, none can exist for the former.
Worst, the class of primitive recursive permutations is not closed under inversion \cite{kuznecov50sssr,PaoliniPiccoloRoversiICTCS2015,soare1987book}.
Since the above negative results hold also for the class of elementary permutations 
\cite{cannonito1969jsl,kalimullin03permutations,koz1974ail}, there is no
hope to find any effective description neither of the class of recursive 
permutations nor of the classes of primitive recursive permutations and of elementary permutations.


\paragraph{Comparison with the known literature}
Our quest to identify the reversible  analogous of $ \PRF $  must be throughly related to the following works.
\begin{itemize}
\item 
The programming language $ \mathsf{SRL} $  restricts $ \mathsf{LOOP} $, a language 
for programming $\PRF$ functions \cite{meyer1967acm}. 
Matos introduces $ \mathsf{SRL} $ and some of its variants in \cite{matos03tcs}.
He is the first using $ \Int $ --- the natural numbers with sign --- as the ground type for 
classes of reversible functions.
We share the choice with him.
The work \cite{matos03tcs} focuses on the algebraic aspects of his languages and its relations with 
matrix groups. 

%\todo{Questions 1 and 2, Reviewer 1}
The study of the classes of functions that $ \mathsf{SRL} $ variants can identify
  is part of Matos' work.
Variants of $\mathsf{SRL}$ are complete with respect to reversible boolean circuits \cite{matos2016notes}.
Proving that some given variant of $\mathsf{SRL}$ is $ \mathsf{PRF} $-complete,
i.e. that it represents all the functions in $ \mathsf{PRF} $, naturally ends up with the quest 
to encode the ``test for 0'' like in the proof that $ \mathsf{LOOP} $ and $ \mathsf{PRF} $ are equivalent \cite{meyer1967acm}. 
We \emph{conjecture} that no variant of Matos' languages exists able to simulate a conditional 
control on the flow of execution that, instead, $ \RPP $ contains by definition.
Of course, proving the conjecture false, would promote $\mathsf{SRL}$ to be (in a reasonable sense) the minimal reversible and 
$ \mathsf{PRF} $-complete language.

\item The precursor of this work is \cite{PaoliniPiccoloRoversiICTCS2015}. It introduces the class of functions $ \RPRF $ which is closed by inversion and is $\PRF$-complete. 
The completeness of $ \RPRF $ relies on \emph{built-in} Cantor pairings.
In this paper we show that we can get rid of \emph{built-in} Cantor pairings inside $ \RPP $.
With no \emph{built-in} pairing operators at hand the design of $ \RPP $ relies on operators which are more fine-grained as compared
to the ones used for the definition of $ \RPRF $.
This orients the programming style to enjoy a couple of interesting features.
Inside $ \RPP $ it is natural avoiding to save the entire history of a calculation within a single argument, i.e. into a single ancilla.
This allows to clean the garbage at the end of the simulation of any $ f\in\PRF $, something which is reminiscent of the simulation 
of Turing machines devised in \cite{bennett1989siamjc}.

\item 
Finally we discuss \cite{jacopini89tcs}. 
It introduces the class of reversible functions $ \JMF $ which is as expressive as Kleene's
partial recursive functions \cite{cutland1980book,odifreddi1989book}. 
Therefore, the focus of \cite{jacopini89tcs} is on partial reversible functions while ours  is on total ones. 
The expressiveness of $ \JMF $ is clearly stronger than the one of  $ \RPP $. 
However, we see $ \JMF $ as less abstract than $ \RPP $ for two reasons.
On one side, the primitive functions of $ \JMF $ relies on a given specific binary representation of natural numbers.
On the other, it is not evident that $ \JMF $ is the extension of a total sub-class
analogous to $ \RPP $ which should ideally correspond to $ \PRF $, but in a reversible setting.
\end{itemize}

%%%%%%%%%%%%%%%%%%%%%%%%%%%%
\paragraph{Contents}
We propose a formalism that identifies a class of functions which we call Reversible Primitive Permutations
($ \RPP $) and which is strictly included in the set of permutations.
Section~\ref{section:The class RPP of Reversible Primitive Permutations} defines $ \RPP $ in analogy
with the definition of $ \PRF $, \ie $ \RPP $ is the closure of composition schemes on basic functions.

The functions of $ \RPP $ have identical arity and co-arity and take 
$ \Int $ --- and not only $ \Nat $ --- as their 
domain because $ \Nat $ is not a group. So, $ \RPP $ is sufficiently abstract 
to avoid any reference to any specific 
encoding of numbers and strongly connects our work to Matos' one \cite{matos03tcs}.

For example, in $ \RPP $ we can define a \texttt{sum} that given the two integers $ m $ and $ n $
yields $ m+n $ and $ n $. Let us represent \texttt{sum} as:
\begin{align}
\label{align:sum example}
\arraycolsep=1.4pt
\begin{array}{rcl}
 \left. {\scriptsize \begin{array}{r} m    \\ n \end{array}} \right[
 & \texttt{sum} &
 \left] {\scriptsize \begin{array}{l} m + n\\ n \end{array} } \right.
\end{array}
\enspace .
\end{align}
The implementation of \texttt{sum} inside $ \RPP $ exploits an iteration scheme that iterates $ n $
times the successor, starting on the initial value $ m $ of the first input.
$ \RPP $ implies the existence of a (meta and) \emph{effective} procedure which produces 
the inverse $ \rprInv{f}\in\RPP $ of every given $ f\in\RPP $. 
\Ie, :
\begin{align}
\label{align:diff example}
\arraycolsep=1.4pt
\begin{array}{rcl}
 \left. {\scriptsize \begin{array}{r} p    \\ n \end{array}} \right[
 & \rprInv{\texttt{sum}} &
 \left] {\scriptsize \begin{array}{l} p - n\\ n \end{array} } \right.
\end{array}
\end{align}
belongs to $ \RPP $ and it ``undoes'' \texttt{sum} by iterating $ n $ times
the predecessor on $ p $. So if $ p = m + n $, then $ p - n = m + n - n = m$.
We remark we could have internalized the operation that yields the inverse of a 
function inside $\RPP$ like in \cite{matos03tcs}. We do not internalize it
to avoid mutually recursive definitions in $ \RPP $.

Concerning the \emph{expressiveness}, $ \RPP $ is closed under inversion and it is both $ \PRF $-complete and $ \PRF $-sound.
Completeness is the really relevant property between the two because this means that 
$ \RPP $ subsumes the class $ \PRF $.
This result is in  Section~\ref{section:Completeness of RPP}. It requires quite a lot 
of preliminary work that one can find in Sections~\ref{section:Generalizations inside RPP}, 
\ref{section:A library of functions in TRRF} and~\ref{section:Cantor pairing} 
and whose goal is to encode a bounded minimalization and Cantor pairing. 
The embedding relies on various key aspects of reversible computing. The principal
ones are:  (i) the use of ancillary variables to clone information
and (ii) the compositional programming under the pattern that Bennett's trick dictates
(cf. Section~\ref{section:Some recursion theoretic side effects of RPP}.)

%%%%%%%%%%%%%%%%%%%%%%%%%%%%
\paragraph{Contributions}
 $\RPP$ is a total class of reversible functions closed under inversion, which is $ \PRF $-complete 
and sound. We think that it can play a key role in the setting of reversible computations analogous to that played by $\PRF$ in classical recursion theory.
In particular, it can be used to devise the analogous of Kleene's normal form theorem and Kleene's predicates in the reversible setting;
it can be also used to formalize a reversible arithmetic as the primitive recursive one (a.k.a. Skolem arithmetic).

%As a last remark, we underline that the speculations leading us to the identification
%of $ \RPP $ could slightly widen our foundational perspective on the ``realm'' 
%of (classic and reversible) recursive computations and on their mathematical formalizations. 
%We present our reflections in Section~\ref{section:Some recursion theoretic side effects of RPP} 
%together with (some) possible future work.




%%%%%%%%%%%%%%%%%%%%%%%%% servono ad emacs
%%% Local Variables:
%%% mode: latex
%%% TeX-master: "main.tex"
%%% ispell-local-dictionary: "american"
%%% End: