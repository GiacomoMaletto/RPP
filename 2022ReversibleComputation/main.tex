\documentclass[runningheads]{llncs}

\usepackage[utf8x]{inputenc}
\usepackage[T1]{fontenc}


% Text organization
\usepackage{comment}
\usepackage{csquotes}
\usepackage{caption} % subfigures
\usepackage{subcaption}


\usepackage{amsmath}
\usepackage{amsfonts}
\usepackage{amssymb}
\usepackage{stmaryrd}
\SetSymbolFont{stmry}{bold}{U}{stmry}{m}{n}
\newcommand*{\qeda}{\hfill\ensuremath{\blacksquare}}%
\newcommand*{\qedb}{\hfill\ensuremath{\square}}%
\newtheorem{fact}{Fact}[section]
\usepackage{latexsym}
\usepackage{cmll}
\usepackage{xspace}

% graphics
\usepackage{mathdots}
\usepackage{nicematrix}
\setcounter{MaxMatrixCols}{20}
\usepackage{rotating}

\usepackage{xcolor}
\definecolor{keywordcolor}{rgb}{0.7, 0.1, 0.1} % red
\definecolor{commentcolor}{rgb}{0.4, 0.4, 0.4} % grey
\definecolor{symbolcolor}{rgb}{0.0, 0.1, 0.6}  % blue
\definecolor{sortcolor}{rgb}{0.1, 0.5, 0.1}    % green
\definecolor{aawhite}{rgb}{0.97,0.97,0.97}
\definecolor{awhite}{rgb}{0.90,0.90,0.90}
\definecolor{lgreen}{rgb}{0.94,1.0,0.98}
\definecolor{dgreen}{rgb}{0.0,0.3,0.1}
\definecolor{sgreen}{rgb}{0.0,0.7,0.3}
\definecolor{lgreen}{rgb}{0.94,1.0,0.98}
\definecolor{bgreen}{rgb}{0.00,0.50,0.25}
\definecolor{dblue}{rgb}{0.0,0.1,0.6}
\definecolor{lorange}{rgb}{1, .85, .60}
\definecolor{lblue}{rgb}{.80, .90, .95}
\definecolor{mixed}{rgb}{0.0,0.3,0.3}
\definecolor{dred}{rgb}{0.6,0.2,0.0}
\definecolor{sred}{rgb}{0.7,0.2,0.0}
\definecolor{ddred}{rgb}{0.3,0.1,0.0}
\definecolor{turq}{rgb}{0.28,0.82,0.80}
\definecolor{lyellow}{rgb}{1.00,0.97,0.94}
\definecolor{mygreen}{rgb}{0,0.6,0}
\definecolor{mygray}{rgb}{0.5,0.5,0.5}
\definecolor{mymauve}{rgb}{0.58,0,0.82}
\definecolor{codegreen}{rgb}{0,0.6,0}
\definecolor{codegray}{rgb}{0.5,0.5,0.5}
\definecolor{codepurple}{rgb}{0.58,0,0.82}
\definecolor{backcolour}{rgb}{0.96,0.96,0.96}

\usepackage{graphicx}

\usepackage{hyperref}
\hypersetup{
    colorlinks=true,
    linkcolor=sred,
    filecolor=magenta,
    urlcolor=cyan,
    citecolor=sgreen,
    bookmarks=true,
}


% code
\usepackage{listings}
\def\lstlanguagefiles{lstlean.tex}
\lstset{language=lean}
\usepackage{float}
\floatstyle{boxed}
\restylefloat{table}
\restylefloat{figure}

%\usepackage{listings}
%\renewcommand{\verb}{\lstinline}
%% Yarel related definitions
%\input{yarel-lst.tex}
%\lstdefinestyle{yarel-style}{
%	backgroundcolor=\color{backcolour},
%	commentstyle=\color{codegreen},
%	keywordstyle=\color{magenta},
%	numberstyle=\tiny\color{codegray},
%	stringstyle=\color{codepurple},
%	%	basicstyle=\fontsize{9}{13}\selectfont\ttfamily,
%	basicstyle=\ttfamily\footnotesize,
%	breakatwhitespace=false,
%	breaklines=true,
%	captionpos=b,
%	keepspaces=true,
%	numbers=left,
%	numbersep=5pt,
%	showspaces=false,
%	showstringspaces=false,
%	showtabs=false,
%	tabsize=2
%}
%\lstset{
%	language={yarel},
%%	basicstyle=\fontsize{9}{13}\selectfont\ttfamily,
%	basicstyle=\small\ttfamily, % Global Code Style
%	captionpos=b, % Position of the Caption (t for top, b for bottom)
%	extendedchars=true, % Allows 256 instead of 128 ASCII characters
%	tabsize=2, % number of spaces indented when discovering a tab
%	columns=fixed, % make all characters equal width
%	keepspaces=true, % does not ignore spaces to fit width, convert tabs to spaces
%	showstringspaces=false, % lets spaces in strings appear as real spaces
%	breaklines=true, % wrap lines if they don't fit
%	frame=trbl, % draw a frame at the top, right, left and bottom of the listing
%	frameround=tttt, % make the frame round at all four corners
%	framesep=4pt, % quarter circle size of the round corners
%	numbers=left, % show line numbers at the left
%	numberstyle=\tiny\ttfamily, % style of the line numbers
%	commentstyle=\color{yarelGreen}, % style of comments
%	keywordstyle=\color{mymauve}, % style of keywords
%	%	stringstyle=\color{yarelBlue}, % style of strings
%	identifierstyle=\color{dblue},
%	stringstyle=\color{orange},
%	backgroundcolor=\color{gray!5},
%	mathescape=true,
%	sensitive=false, % keywords are not case-sensitive
%}
%\newcommand{\verby}[1]{\lstinline[language=yarel,style=yarel-style]+#1+}
%% JAVA related definitions
%\input{java-lst.tex}
%\lstdefinestyle{java-style}{
%	backgroundcolor=\color{backcolour},
%	commentstyle=\color{codegreen},
%	keywordstyle=\color{magenta},
%	numberstyle=\tiny\color{codegray},
%	stringstyle=\color{codepurple},
%	basicstyle=\fontsize{9}{13}\selectfont\ttfamily,
%%	basicstyle=\ttfamily\footnotesize,
%	breakatwhitespace=false,
%	breaklines=true,
%	captionpos=b,
%	keepspaces=true,
%	numbers=left,
%	numbersep=5pt,
%	showspaces=false,
%	showstringspaces=false,
%	showtabs=false,
%	tabsize=2
%}
%\lstset{
%	language={java},
%	morekeywords={cached,case,default,extension,false,import,JAVA,WORKFLOWSLOT,let,new,null,private,create,switch,this,true,reexport,around,if,then,else, def, val, var, private, class, static, return, as, instanceof, for, override, boolean},
%	morekeywords=[2]{filter, toList, last, head},
%	keywordstyle=[2]\color{eclipseOrange}\textit,
%	morekeywords=[3]{FAST, NORMAL, IN, PROVIDED},
%	keywordstyle=[3]\color{eclipseBlue}\textsl,
%	morekeywords=[4]{},          % <-- Your keywords here (w/ orange)
%	keywordstyle=[4]\color{eclipseOrange},
%	morecomment=[l]{//},
%	morecomment=[s]{/*}{*/},
%	morestring=[b]",
%}
%\newcommand{\verbj}[1]{\lstinline[language=java,style=java-style]+#1+}
%\newcommand{\vj}[1]{\lstinline[language=java,style=java-style]+#1+}
%%% lstlisting stop definitions


% macro
% \newcommand{\svdots}{\scriptscriptstyle \boldsymbol \vdots}
\newcommand{\perm}[1]{\scriptstyle \left\rmoustache #1 \right\rmoustache}
\newcommand{\bloch}[2]{\Block[draw=white,fill=lblue,line-width=.5mm,rounded-corners]{#1}{#2}} % https://en.wikipedia.org/wiki/Ernest_Bloch
\newcommand{\oloch}[2]{\Block[draw=white,fill=lorange,line-width=.5mm,rounded-corners]{#1}{#2}}
\newcommand{\conv}[1]{\overbracket[.5pt][1pt]{\underbracket[.5pt][1pt]{#1}}}

\newcommand{\NN}{\mathbb{N}}
\newcommand{\ZZ}{\mathbb{Z}}
\newcommand{\RR}{\mathbb{R}}

\newcommand{\floor}[1]{\left\lfloor #1 \right\rfloor}

\newcommand{\gl}{\text{\guillemotleft}}
\newcommand{\gr}{\text{\guillemotright}}


\newcommand{\ORPP}{\mathsf{ORPP}}
\newcommand{\rppf}{\mathsf{f}}
\newcommand{\rppg}{\mathsf{g}}
\newcommand{\rpph}{\mathsf{h}}
\newcommand{\rppId}{\mathsf{Id}}
\newcommand{\rppNe}{\mathsf{Ne}}
\newcommand{\rppSu}{\mathsf{Su}}
\newcommand{\rppPr}{\mathsf{Pr}}
\newcommand{\rppSw}{\mathsf{Sw}}
\newcommand{\rppCo}{\fatsemi}
\newcommand{\rppPa}{\Vert}
\newcommand{\rppIt}{\mathsf{It}}
\newcommand{\rppIta}{\mathsf{Ita}}
\newcommand{\rppItr}{\mathsf{Itr}}
\newcommand{\rppIf}{\mathsf{If}}
\newcommand{\rppinc}{\mathsf{inc}}
\newcommand{\rppdec}{\mathsf{dec}}
\newcommand{\rppmul}{\mathsf{mul}}
\newcommand{\rppsquare}{\mathsf{square}}
\newcommand{\rppless}{\mathsf{less}}
\newcommand{\rppcp}{\mathsf{cp}}
\newcommand{\rppcpi}{\mathsf{cp_{input}}}
\newcommand{\rpptr}{\mathsf{tr}}
\newcommand{\rppcu}{\mathsf{cu}}
\newcommand{\rppcui}{\mathsf{cu_{input}}}
\newcommand{\rppcustep}{\mathsf{cu_{step}}}
\newcommand{\rppdiv}{\mathsf{div}}
\newcommand{\rppdivstep}{\mathsf{div_{step}}}
\newcommand{\rppsqrt}{\mathsf{sqrt}}
\newcommand{\rppsqrtstep}{\mathsf{sqrt_{step}}}
\newcommand{\rpprewire}[1]{\lfloor #1 \rceil}
\newcommand{\rppmkpair}{\mathsf{mkpair}}
\newcommand{\rppmkpairi}{\mathsf{mkpair_{input}}}
\newcommand{\rppunpair}{\mathsf{unpair}}
\newcommand{\rppunpairi}{\mathsf{unpair_{input}}}
\newcommand{\rppunpairifwd}{\mathsf{unpair_{input, \rightarrow}}}
\newcommand{\rppprecstep}{\mathsf{prec_{step}}}
\newcommand{\rppprecfwd}{\mathsf{prec_{\rightarrow}}}
\newcommand{\rppprec}{\mathsf{prec}}
\newcommand{\rppconvert}{\mathsf{convert}}
\newcommand{\rppNu}{\mathsf{Nu}}
\newcommand{\rppZe}{\mathsf{Ze}}
\newcommand{\rppEz}{\mathsf{Ez}}
\newcommand{\rppbin}{\mathsf{bin}}


\newcommand{\prZero}{\mathtt{0}}
\newcommand{\prSucc}{\mathtt{S}}
\newcommand{\prPred}{\mathtt{P}}
\newcommand{\prProj}[2]{\mathtt{P}^{#1}_{#2}} % P^n_i
\newcommand{\prCom}[1]{\circ{[#1]}} % composition
\newcommand{\prPrec}{\mathtt{R}}
\newcommand{\pradd}{\mathtt{ADD}}
\newcommand{\prack}{\mathtt{ACK}}

% Macros
\newcommand{\RPP}{\textsf{RPP}\xspace}
\newcommand{\UPRF}{\textsf{UPRF}\xspace}
\newcommand{\PRF}{\textsf{PRF}\xspace}
\newcommand{\PR}{\textsf{PR}\xspace}
\newcommand{\CPP}{\textsf{C}\xspace}
\newcommand{\LEAN}{\textsf{LEAN}\xspace}
\newcommand{\LEANFour}{\textsf{LEAN4}\xspace}
\newcommand{\RPRF}{\textsf{R}\PRF\xspace} % reversible primitive
\newcommand{\JMF}{\textsf{RI}\xspace} % Jacopini Mentrasti
\newcommand{\Janus}{\textsf{Janus}\xspace}
\newcommand{\Matita}{\textsf{Matita}\xspace}
\newcommand{\SRL}{\textsf{SRL}\xspace}
\newcommand{\Nat}{\mathbb{N}}
\newcommand{\Int}{\mathbb{Z}}


\begin{document}
%\title{A formal certification that Reversible Primitive Permutations are Primitive-recursive complete}
\title{Certifying algorithms and relevant properties of  Reversible Primitive Permutations with \LEAN}

\author{Giacomo Maletto\inst{1} \and
	    Luca Roversi\inst{2}\orcidID{0000-0002-1871-6109}}

\authorrunning{G. Maletto, L. Roversi}

\institute{
    Università degli Studi di Torino, Dipartimento di Matematica  -- Italy\\
	\email{giacomo.maletto@edu.unito.it}\\
	\and
	Università degli Studi di Torino, Dipartimento di Informatica -- Italy\\
	\email{luca.roversi@unito.it}}

\maketitle
\begin{abstract}
%TODO
\end{abstract}

%=====================
\section{Introduction}
\label{section:Introduction}
Studies focused on questions posed by Maxwell, regarding the sensibility of the principles which Thermodynamics is based on, recognized the fundamental role that Reversible Computation can play to that purpose.

Once identified, it has been apparent that Reversible Computation constitutes the context in which to frame relevant aspects in areas of the computer science; they can span from reversible hardware design which can offer a greener foot-print, as compared to classical hardware, to unconventional computational models --- we think to quantum or bio-inspired ones, for example ---, passing through parallel computation and the synchronization issues that it rises, or debuggers that help tracing back to the origin of a bug, or the consistent transactions roll-back in data-base management systems, just to name some. The book \cite{perumalla2013chc} is a comprehensive introduction to the subject; the book \cite{DBLP:books/daglib/0025734}, focused on the low-level aspects of Reversible Computation, concerning the realization of reversible hardware, and
\cite{DBLP:series/eatcs/Morita17}, focused on how models of Reversible Computation like Reversible Turing Machines (RTM), and Reversible Cellular Automata (RCA) can be considered universal and how to prove that they enjoy such a property, can be complement and integrate \cite{perumalla2013chc}.

This work focuses on the \emph{functional model} \RPP \cite{DBLP:journals/tcs/PaoliniPR20} of Reversible Computation.
\RPP stands for (the class of) Reversible Primitive Permutations which can be seen as a possible reversible counterpart of \PRF, the class of Primitive Recursive functions \cite{rogers1967theory}.
We recall that \RPP, in analogy with \PRF, is defined as the small class built on some given basic reversible function,
and which is closed under suitable composition schemes.

The very functional nature of the elements in \RPP is at the base of reasonably accessible proofs of the following properties:
\begin{itemize}
\item \RPP is \PRF-complete, i.e. every function $ f $ in \PRF has a counterpart $ \hat{f} $ in \RPP that simulates $ f $ \cite{DBLP:journals/tcs/PaoliniPR20};

\item \RPP can be naturally extended to become Turing-complete \cite{Paolini2018NGC} by means of a composition scheme analogous to the one that extends \PRF to the Turing-complete class \PR of Partial Recursive functions;

\item According to \cite{MatosRC2020}, \RPP and the reversible programming language \SRL \cite{matos03tcs} are equivalent, so the fix-point problem is undecidable for \RPP as well \cite{2318_1734164MatosPaoliniRoversiTCSICTCS18}.
\end{itemize}

The motivation to write this work is to bring evidence that expressing Reversible Computation by means of recursively defined computational models, like the class \RPP, \emph{naturally} offers the possibility to certify the correctness, or other interesting properties we might be interested to, of reversible algorithms, by means of some proof-assistant.
We recall that proof-assistants supply an integrated environment to formalize data-types, to implement algorithms that work on those data-types, to formalize specifications of the implementations, and to prove in full detail that those implementation match their specifications, increasing their dependability.

%%%%%%%%%%%%%%%%%%%%%%%%%%%%
\paragraph{Contributions.}
Roughly, we show how to express \RPP and its evaluation mechanism inside the proof-assistant \LEAN \cite{Lean3}. On one side we are able to certify the correctness of every reversible function that we write in \RPP with respect to a given specification. On the other, we certify the correctness of the main result in \cite{DBLP:journals/tcs/PaoliniPR20} which states that \RPP is \PRF-complete. In more detail:
\begin{itemize}
    \item we give a strong guarantee that \RPP is \PRF-complete.
    Relying on existing proofs that the classes \PRF and \UPRF --- the class of \emph{Unary} Primitive recursive Functions ---, are equivalent, inside the proof-assistant \LEAN, we prove that, for every function $ f $ of \UPRF, a function in \RPP exists that represents and simulates $ f $.
    Despite the proof is fully detailed, besides fixing some small bugs, we also conceptually and technically simplify the original \PRF-completeness proof of \RPP in \cite{DBLP:journals/tcs/PaoliniPR20};

    \item on one side the simplification is a consequence of the representation of \UPRF available as a \LEAN library. On the other, and more remarkably,  the two following aspects characterize the simplifications:
    \begin{itemize}
        \item we introduce a finite iteration scheme which is more primitive, but expressive enough to encode the finite iteration schemes originally in both \RPP and \SRL;
        \item we completely rework and simplify the algorithm that implements Cantor Pairing \cite{Cantor1878,DBLP:journals/corr/Szudzik17} in \RPP \cite{DBLP:journals/tcs/PaoliniPR20}.
        We recall that Cantor Pairing is an isomorphism $ \mathbb{N}\times\mathbb{N} \simeq \mathbb{N} $. Inside the proof of \PRF-completeness, Cantor Pairing works as a stack; it allows us to correctly manage the computation of a recursive scheme of \UPRF by means of an element in \RPP.
    \end{itemize}

    \item we show that the \RPP function that we define to implement Cantor Pairing is one possible instance of a more general pattern that allows us to uniquely associate pairs of $ \mathbb{N}\times\mathbb{N}$ to a single element of $ \mathbb{N}$. This means that we are able to show further isomorphims  $ \mathbb{N}\times\mathbb{N}\simeq \mathbb{N}$ in \RPP by just modifying a core of Cantor Pairing.
\end{itemize}

%---------------------
\paragraph{Related work.}
Works that relate algorithms implemented by means of a formalism that can express Reversible Computation seems not to abound. We are aware of \cite{paoliniTYPES2015}. By means of the proof-assistant \Matita \cite{Asperti2007}, it certifies that a denotational semantics for the imperative reversible programming language \Janus \cite[Section 8.3.3]{perumalla2013chc} is fully abstract with respect to the operational semantics.

An early proposal to express reversible computations in a functional formalism is \cite{jacopini89tcs}. It introduces the class \JMF of reversible functions which is as expressive as \PR. So, \JMF is stronger than \RPP, but we see \JMF as less abstract than \RPP for two reasons: (i) the primitive functions of \JMF depend on a given specific binary representation of natural numbers; (ii) unlike \RPP, which we can see as \PRF in a reversible setting, it is not evident to us that \JMF can be considered the natural extension of a total class analogous to \RPP.

%---------------------
\paragraph{Contents.}
This work illustrates the relevant parts of the BSc Thesis \cite{MalettoBSc2021} which comes with \cite{MalettoRPPLEAN2021}, a \LEAN project that certifies algorithms and properties of \RPP.
Section~\ref{section:Reversible Primitive Permutations} recalls the class \RPP by commenting on the main design aspects that characterize its definition inside \LEAN.
Section~\ref{section:Some certified algorithms inside LEAN} defines and proves correct new reversible algorithms central to prove that \RPP is \PRF-complete.
Section~\ref{section:Unary Primitive Recursive Functions} recalls the main aspects of \UPRF, with particular attention to the computational behavior of its recursion scheme.
Section~\ref{section:The UPRF-completeness of RPP} illustrates the points that characterize the porting of the original \RPP \PRF-complete inside \LEAN.
Section~\ref{section:Conclusion and developments} underlies further more general contribution of the work, a reason why we choose \LEAN,  and natural developments.

%=====================
\section{Reversible Primitive Permutations (\RPP) }
\label{section:Reversible Primitive Permutations}

\begin{figure}
    \centering
        \begin{lstlisting}
        inductive RPP : Type
        -- Base functions
        | Id (n : ℕ) : RPP -- Identity
        | Ne : RPP         -- Sign-change
        | Su : RPP         -- Successor
        | Pr : RPP         -- Predecessor
        | Sw : RPP         -- Transposition
        -- Inductively defined functions
        | Co (f g : RPP) : RPP   -- Parallel composition
        | Pa (f g : RPP) : RPP   -- Series composition
        | It (f : RPP) : RPP     -- Finite iteration
        | If (f g h : RPP) : RPP -- Selection
        infix `‖`  : 55 := Pa -- Notation for the Parallel composition
        infix `;;` : 50 := Co -- Notation for the Series composition
        \end{lstlisting}
    \caption{The class \RPP as a data-type \lstinline|RPP| in \LEAN.}
    \label{fig:RPP-LEAN}
\end{figure}

We use the data-type \lstinline|RPP| in Figure~\ref{fig:RPP-LEAN}, as defined in \LEAN, to recall from \cite{DBLP:journals/tcs/PaoliniPR20} that the class \RPP is the smallest class of functions
that contains five base functions, named as in the definition, and all the functions that we can generate by the composition schemes whose name is next to the corresponding clause in Figure~\ref{fig:RPP-LEAN}. For ease of use and readability the two last lines in Figure~\ref{fig:RPP-LEAN} introduce infix notations for series and parallel compositions.

\begin{example}[A  term of type {\normalfont \lstinline|RPP|}]
\label{example:A first legal term of type RPP}
In \lstinline|RPP| we can write
\lstinline|(Id 1‖Sw);;(It Su)‖(Id 1);;(Id 1‖If Su (Id 1) Pr)| which we can also represent as a diagram:
\[
\begin{NiceMatrix}
x_0&\bloch{1-1}{\mbox{\lstinline|Id 1|}}&y_0&\bloch{2-1}{\mbox{\lstinline|It  Su|}}&w_0&\bloch{1-1}{\mbox{\lstinline|Id 1|}}&z_0
\\
x_1&\bloch{2-1}{\mbox{\lstinline|Sw|}}&y_1& &w_1&\bloch{2-1}{\mbox{\lstinline|If Su (Id 1) Pr|}}&z_1
\\
x_2&&y_2&\bloch{1-1}{\mbox{\lstinline|Id 1|}}&w_2&
&z_2
\end{NiceMatrix}
\enspace .
\]
The term we have just defined is a series composition of three parallel compositions. The first parallel composition is between a ``single wire'' \lstinline|Id 1| and the transposition \lstinline|Sw|.
The second parallel composition is between the finite iteration of
the successor \lstinline|Su| and a single wire.
The third parallel composition is between a single wire and a selection among \lstinline|Su|, \lstinline|Id 1| and the predecessor \lstinline|Pr|. Since \LEAN forces to use functional notation, \lstinline|It Su|, for example, is the \emph{application} of the unary constructor \lstinline|It| to the nullary constructor \lstinline|Su| that takes no arguments.
\qeda
\end{example}

\begin{figure}
\centering
\begin{lstlisting}
def arity : RPP → ℕ
  | (Id n)     := n
  | Ne         := 1
  | Su         := 1
  | Pr         := 1
  | Sw         := 2
  | (f ‖ g)    := f.arity + g.arity
  | (f ;; g)   := max f.arity g.arity
  | (It f)     := 1 + f.arity
  | (If f g h) := 1 + max (max f.arity g.arity) h.arity
\end{lstlisting}
\caption{Arity of every \lstinline|f: RPP|.}
\label{fig:RPP-arity}
\end{figure}

The function in Figure~\ref{fig:RPP-arity} computes the arity of any \lstinline|f:RPP| from the structure of \lstinline|f|, once fixed the arities of the base functions.

\begin{remark}[Multiple wires are base functions]
\label{remark:Multiple wires are base functions}
Every finite set of \lstinline|n| parallel wires
$
\begin{NiceMatrix}
    x_0&\bloch{1-1}{\mbox{\lstinline|Id 1|}}&y_0
    \\
    &\vdots&
    \\
    x_{n-1}&\bloch{1-1}{\mbox{\lstinline|Id 1|}}&y_{n-1}
\end{NiceMatrix}
$, written \lstinline|Id n|, and represented as
$
\begin{NiceMatrix}
    x_0&\bloch{3-1}{\mbox{\lstinline|Id n|}}&y_0
    \\
    \vdots&&\vdots
    \\
    x_{n-1}&&y_{n-1}
\end{NiceMatrix}
$, \emph{is a base} function with arity \lstinline|n|.
\qeda
\end{remark}

\begin{figure}
\begin{lstlisting}
def inv : RPP → RPP
  | (Id n)     := Id n -- self-dual
  | Ne         := Ne   -- self-dual
  | Su         := Pr
  | Pr         := Su
  | Sw         := Sw   -- self-dual
  | (f ‖ g)    := inv f ‖ inv g
  | (f ;; g)   := inv g ;; inv f
  | (It f)     := It (inv f)
  | (If f g h) := If (inv f) (inv g) (inv h)
notation f `⁻¹` := inv f
\end{lstlisting}
\caption{Inverse \lstinline|inv f| of every \lstinline|f:RPP|.}
\label{fig:RPP-inv}
\end{figure}

For any given \lstinline|f:RPP|, the function \lstinline|inv| in Figure~\ref{fig:RPP-inv} builds an element with type \lstinline|RPP|. The definition of \lstinline|inv| let the successor \lstinline|Su| be inverse of the predecessor \lstinline|Pr| and let every other base function be self-dual.
Moreover, the function \lstinline|inv| distributes over finite iteration \lstinline|It|, selection \lstinline|If|, and parallel composition \lstinline|‖|, while it requires to exchange the order of the arguments before distributing over the series composition \lstinline|;;|. The last line with \lstinline|notation| suggests that \lstinline|f⁻¹| is the inverse of \lstinline|f|. We shall prove it once given the operational semantics of \lstinline|RPP|.

\begin{figure}
\begin{lstlisting}
def ev : RPP → list ℤ → list ℤ
| (Id n)     X                    := X
| Ne         (x :: X)             := -x :: X
| Su         (x :: X)             := (x + 1) :: X
| Pr         (x :: X)             := (x - 1) :: X
| Sw         (x :: y :: X)        := y :: x :: X
| (f ;; g)   X                    := ev g (ev f X)
| (f ‖ g)    X         := ev f (take f.arity X) ++ ev g (drop f.arity X)
| (It f)     (x :: X)             := x :: ((ev f)^[↓x] X)
| (If f g h) (0 :: X)             := 0 :: ev g X
| (If f g h) (((n : ℕ) + 1) :: X) := (n + 1) :: ev f X
| (If f g h) (-[1 + n] :: X)      := -[1 + n] :: ev h X
| _          X                    := X
notation `‹` f `›` := ev f -- ‹ \f ›
\end{lstlisting}
\caption{Operational semantics of elements in \lstinline|RPP|.}
\label{fig:RPP-ev}
\end{figure}

\paragraph{Operational semantics of {\normalfont \lstinline|RPP|}.}
The function \lstinline|ev| in Figure~\ref{fig:RPP-ev} interprets an element of \lstinline|RPP| as a function from a list of integers to a list of integers. Originally, in \cite{DBLP:journals/tcs/PaoliniPR20}, \RPP is a class of functions with type $ \ZZ^n \longrightarrow \ZZ^n $. We use \lstinline|list ℤ| in place of tuples of \lstinline|ℤ| to exploit the rich \LEAN library of properties on lists to save quite a large amount of formalization work.

Let us give a look at the clauses in Figure~\ref{fig:RPP-ev}.
\lstinline|Id n| leaves the input list \lstinline|X| untouched.
\lstinline|Ne| negates the head of the list, \lstinline|Su| increments, and \lstinline|Su| decrements it.
\lstinline|Sw| is the transposition that switches the first two elements of its argument.
The series composition \lstinline|f;;g| first applies \lstinline|f| and then
\lstinline|g|.
The parallel composition \lstinline|f‖g| splits the arguments into two parts, the ``topmost'' one \lstinline|take f.arity X| with as many elements as the arity of \lstinline|f|, the ``lowermost'' one \lstinline|drop g.arity X| with as many elements as the arity of \lstinline|g|, and concatenates by the append \lstinline|++| the two resulting lists. The finite iteration
\lstinline|It f| iterates \lstinline|f| as many times as the value of the head \lstinline|x| of the argument, if \lstinline|x| contains a non negative value; otherwise it is the identity on the whole \lstinline|x::X|. This behavior is the meaning of \lstinline|(ev f)^[↓x]|.
The selection \lstinline|If f g h| chooses one among \lstinline|f|, \lstinline|g|, and \lstinline|h|, depending on the argument head \lstinline|x|: it is \lstinline|g| with \lstinline|x = 0|, it is \lstinline|f| with \lstinline|x > 0|, and \lstinline|h| with \lstinline|x < 0|.
The last line of Figure~\ref{fig:RPP-ev} sets a handy notation for \lstinline|ev|.

\begin{remark}[We keep the definition of {\normalfont \lstinline|ev|}  simple]
\label{remark:We keep the definition of ev simple}
Our definition of \lstinline|ev| implies that, for any \lstinline|f:RPP|, we can freely apply \lstinline|f| to any \lstinline|X:list ℤ| even when \lstinline|X.length < f.arity|; in that case \lstinline|f X| is \lstinline|X|.
This might sounds odd, but, as we shall see, it simplifies the proofs of must-have properties of \lstinline|RPP|.
\qeda
\end{remark}

%======================
\subsection{Both {\normalfont \lstinline|inv f|} and {\normalfont \lstinline|f|} are each other inverse}
Once defined \lstinline|inv| in Figure~\ref{fig:RPP-inv} and \lstinline|ev| in Figure~\ref{fig:RPP-ev} we can prove both:
\begin{lstlisting}
    theorem inv_co_l (f : RPP) (X : list ℤ) : ‹f ;; f⁻¹› X = X
    theorem inv_co_r (f : RPP) (X : list ℤ) : ‹f⁻¹ ;; f› X = X .
\end{lstlisting}
We start by focusing on the main details to prove \lstinline|theorem inv_co_l| in \LEAN. The proof proceeds by (structural) induction on \lstinline|f|, which generates 9 cases, one for each clause that defines \lstinline|RPP|. One can go through the majority of them smoothly.
Some comments about the more challenging cases follow.

\paragraph{Series composition.} Let \lstinline|f| be \lstinline|Co|. The step-wise proof of \lstinline|inv_co_l| is:
\begin{lstlisting}
‹(f;;g);;(f;;g)⁻¹› X
     = ‹f⁻¹›(‹g⁻¹›(‹g›(‹f› X))) -- by definition
     = ‹f⁻¹›(‹g;;g⁻¹ (‹f› X))   -- by a general ind. hyp.
 (!) = ‹f⁻¹›(‹f› X)             -- by definition
     = ‹f;;f⁻¹› X               -- by a general ind. hyp.
     = X ,
\end{lstlisting}
where the equivalence \lstinline|(!)| holds thanks to the right inductive hypothesis on the behavior of \lstinline|g;;g⁻¹| that \emph{must} be
\lstinline|∀ (X : list ℤ), ‹g;;g⁻¹› X = X|.
\qedb

\paragraph{Parallel composition.} Let \lstinline|f| be \lstinline|Pa|. The step-wise proof of \lstinline|inv_co_l| is:
\begin{lstlisting}
‹f‖g;;(f‖g)⁻¹› X              -- by definition
     = ‹f‖g;;f⁻¹‖g⁻¹› X       -- not obvious
 (!) = ‹(f;;f⁻¹)‖(g;;g⁻¹)› X  -- by definition
     = ‹f;;f⁻¹›(take f.arity X)++‹g;;g⁻¹›(drop f.arity X) -- by ind. hyp.
     = take f.arity X ++ drop f.arity X -- property of ++ (append)
     = X ,
\end{lstlisting}
where the equivalence \lstinline|(!)| holds because we can prove both:
\begin{lstlisting}
lemma pa_co_pa (f f' g g' : RPP) (X : list ℤ) :
  f.arity = f'.arity → ‹f‖g ;; f'‖g'› X = ‹(f;;f') ‖ (g;;g')› X ,
lemma arity_inv (f : RPP) : f⁻¹.arity = f.arity .
\end{lstlisting}
Proving \lstinline|lemma arity_inv|, i.e. that the arity of a function does not change if we invert it, assures that we can prove \lstinline|lemma pa_co_pa|, i.e. that series and parallel compositions smoothly distribute reciprocally.
\qedb

\paragraph{Iteration.} Let \lstinline|f| be \lstinline|It|.
In this case, the most complex goal to prove is \lstinline|‹It f;;It f⁻¹› x::X = x::X| which reduces to \lstinline|‹f⁻¹›^[↓x] (‹f›^[↓x] X') = X'|, where, we recall, the notation \lstinline{‹f›^[↓x]} means ``\lstinline{‹f›} applied \lstinline{x} times, if \lstinline|x| is positive''. Luckily this last statement is both formalized as \lstinline|function.left_inverse (g^[n]) (f^[n])|, and proven as part of \LEAN library, where \lstinline{function.left_inverse} is the proposition \lstinline|∀ (g: β → α) (f: α → β) (x: α), g(f x) = x : Prop|, with \lstinline|α|, and \lstinline|β| generic types.
\qedb

\vspace{\baselineskip}
To conclude, let us see how the proof of \lstinline|inv_co_r| works. It does not copy-cat the one of \lstinline|inv_co_l|. It relies on proving:
\begin{lstlisting}
   lemma inv_involute (f : RPP) : (f⁻¹)⁻¹ = f ,
\end{lstlisting}
\noindent
which says that applying \lstinline|inv| twice is the identity, and on using \lstinline|inv_co_l|:
\begin{lstlisting}
    ‹f⁻¹ ;; f› X = X -- which, by inv_involute, is equivalent to
    ‹f⁻¹ ;; (f⁻¹)⁻¹› X = X -- which holds because it is an instance of (inv_co_l f⁻¹) .
\end{lstlisting}

\begin{remark}[On our simplifying choices on {\normalfont \lstinline|ev|}]
\label{remark:On our simplifying choices on ev}
A less general, but semantically more appropriate versions of  \lstinline|inv_co_l| and \lstinline|inv_co_r| could be:
\begin{lstlisting}
    theorem inv_co_l (f : RPP) (X : list ℤ) :
                f.arity ≤ X.length → ‹f ;; f⁻¹› X = X
    theorem inv_co_r (f : RPP) (X : list ℤ) :
                f.arity ≤ X.length → ‹f⁻¹ ;; f› X = X
\end{lstlisting}
because, recalling Remark~\ref{remark:We keep the definition of ev simple}, \lstinline|f X| makes sense when \lstinline{f.arity ≤ X.length}.
Fortunately, the way we defined \lstinline{RPP} allows us to state \lstinline|inv_co_l| or \lstinline|inv_co_r| in full generality
with no reference to \lstinline|f.arity ≤ X.length|.
\qeda
\end{remark}

%======================
\subsection{How {\normalfont \lstinline|RPP|} differs from original {\normalfont \RPP}}
The definition of \lstinline|RPP| of \LEAN is really very close to the original \lstinline{RPP}, but not identical. The goal is to simplify the overall task of formalization and certification. The brief list of changes follows.
\begin{itemize}
    \item As already outlined, \lstinline|It| and \lstinline|If| use the head of the input list to iterate or choose, because \LEAN pattern matching makes it obvious to take the head of a list. On the contrary, both iteration and selection of \RPP were driven by the value of the last element in the input tuple.

    \item \lstinline|Id n|, for any \lstinline|n:ℕ|, is primitive in \lstinline|RPP| and derived in \RPP. In some cases it is useful to have \lstinline|Id 0| available.

    \item Using \lstinline|list ℤ → list ℤ| as the domain of the function that interpets any given element \lstinline|f:RPP| avoids to let the type of \lstinline|f:RPP| depend on the arity of \lstinline|f|. To know the arity of \lstinline|f| it is enough to invoke \lstinline|arity f|. Finally, we observe that getting rid of a dependent type like, say, \lstinline|RPP n|, allows us to escape situations in which we would need to prove that \lstinline{RPP (n+1)} and \lstinline{RPP (1+n)} are the same type.

    \item The finite iterator \lstinline|It f (x::t): list ℤ| subsumes the finite iterators \lstinline|ItR| in \RPP, and \lstinline|for| in \SRL.

    We recall that \lstinline|ItR f (x₀,x₁,...,xₙ₋₂,x)| evalutes to \lstinline|f(f(...f(x₀,x₁,...,xₙ₋₂)...))| with $ \mid\!\!\mbox{\lstinline|x|}\!\!\mid $ occurrences of \lstinline|f|.
    Instead, \lstinline|for(f) x|\footnote{In fact, the syntax would be ``\lstinline|for x (f)|''. Our choice simplifies the translation \eqref{align:It for}.} evaluates to \lstinline|f(f(...f(x₀,x₁,...,xₙ₋₂)...))| with \lstinline|x| occurrences of \lstinline|f|, if \lstinline|x > 0|; it evaluates to \lstinline|f⁻¹(f⁻¹(...f⁻¹(x₀,x₁,...,xₙ₋₂)...))| with \lstinline|-x| occurrences of \lstinline|f⁻¹|, if \lstinline|x < 0|; it behaves like the identity if \lstinline|x = 0|.

    We can define both \lstinline|ItR| and \lstinline|for| in terms of \lstinline|It|:
    \begin{align}
    \label{align:It ItR}
        \mbox{\lstinline|ItR f|}
        & =
        \mbox{\lstinline|(It f);;Ne;;(It f);;Ne|} \\
    \label{align:It for}
        \mbox{\lstinline|for(f)|}
        & =
        \mbox{\lstinline|(It f);;Ne;;(It f⁻¹);;Ne|}
    \end{align}
    \begin{example}[How does \eqref{align:It ItR} work?]
    \label{example:How align:It ItR works}
    Whenever \lstinline|x > 0|, the leftmost \lstinline|It f| in \eqref{align:It ItR} iterates \lstinline|f|, while the rightmost one does nothing because \lstinline|Ne| in the middle negates \lstinline|x|.
    On the contrary, if \lstinline|x < 0|, the leftmost \lstinline|It f| does nothing and the iteration is performed by the righmost iteration, because \lstinline|Ne| in the middle negates \lstinline|x|. In both cases, the last \lstinline|Ne| restores \lstinline|x| to its intial sign. But this is the behaviour of \lstinline|ItR| as we wanted. \qeda
    \end{example}

    Last, but not least, to convince \LEAN that \lstinline|It| terminates, it is much simpler than to convince it about the termiantion of \lstinline|ItR| or \lstinline|for|.
\end{itemize}

%=====================
\section{Some certified algorithms inside \LEAN}
\label{section:Some certified algorithms inside LEAN}

%=====================
\section{Unary Primitive Recursive Functions (\UPRF) }
\label{section:Unary Primitive Recursive Functions}

%=====================
\section{The \UPRF-completeness of \RPP}
\label{section:The UPRF-completeness of RPP}

%=====================
\section{Conclusion and developments}
\label{section:Conclusion and developments}
On one side this work gives concrete, and not always immediate examples of reversible programming in a proof-assistant. This has a value per se, because, firstly, programming reversible algorithms is not as much wide-spread as iterative, or recursive, programming in classical programming formalisms, and, secondly, we supply an environment in which we can certify them.
On the other, we shall be able to migrate our work to \LEANFour which is going to be delivered in the near future. \LEANFour exports its source code as efficient \CPP code \cite{2021-LEAN4-MouraUllrich}. So, we shall be able to export our reversible algorithms as efficient extensions of \LEANFour, or as standalone, and possibly reversible, \CPP applications.

Concerning possible developments, the most ambitious one is to keep developing a full fledged environment in which it is possible to develop certified reversible software directly or indirectly.

We mean \ldots

%------------------
\bibliographystyle{abbrv}
\bibliography{bib-minimal}

%\appendix

\end{document}