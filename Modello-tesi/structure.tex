%----------------------------------------------------------------------------------------
%	VARIOUS REQUIRED PACKAGES AND CONFIGURATIONS
%----------------------------------------------------------------------------------------

\usepackage[utf8]{inputenc}
\usepackage[italian]{babel}
\usepackage[T1]{fontenc}
\usepackage[round]{natbib}\citeindextrue % Round brackets around citations, change to square for square brackets
\usepackage{graphicx} % Required to include images
%\usepackage{color} % Required for custom colors
\usepackage[dvipsnames]{xcolor}
\definecolor{aawhite}{rgb}{0.97,0.97,0.97}
\definecolor{awhite}{rgb}{0.90,0.90,0.90}
\definecolor{lgreen}{rgb}{0.94,1.0,0.98}
\definecolor{dgreen}{rgb}{0.0,0.3,0.1}
\definecolor{sgreen}{rgb}{0.0,0.7,0.3}
\definecolor{lgreen}{rgb}{0.94,1.0,0.98}
\definecolor{bgreen}{rgb}{0.00,0.50,0.25}
\definecolor{dblue}{rgb}{0.0,0.1,0.6}
\definecolor{lblue}{rgb}{0.8,0.8,1.0}
\definecolor{mixed}{rgb}{0.0,0.3,0.3}
\definecolor{dred}{rgb}{0.6,0.2,0.0}
\definecolor{sred}{rgb}{0.7,0.2,0.0}
\definecolor{ddred}{rgb}{0.3,0.1,0.0}
\definecolor{turq}{rgb}{0.28,0.82,0.80}
\definecolor{lyellow}{rgb}{1.00,0.97,0.94}
\definecolor{mygreen}{rgb}{0,0.6,0}
\definecolor{mygray}{rgb}{0.5,0.5,0.5}
\definecolor{mymauve}{rgb}{0.58,0,0.82}
\definecolor{codegreen}{rgb}{0,0.6,0}
\definecolor{codegray}{rgb}{0.5,0.5,0.5}
\definecolor{codepurple}{rgb}{0.58,0,0.82}
\definecolor{backcolour}{rgb}{0.96,0.96,0.96}

\usepackage{amsmath,amssymb,theorem} % Math packages
\usepackage{listings} % Required for including snippets of code
\usepackage{booktabs} % Required for better horizontal rules in tables
\usepackage{xspace} % Provides the ability to use an intelligent space which is used in \institution and \department
\usepackage[printonlyused,withpage]{acronym} % Include a list of acronyms
\usepackage{rotating} % Allows tables and figures to be rotated
\usepackage{hyperref} % Required for links and changing link options
\hypersetup{
    colorlinks=true,
    linkcolor=sred,
    filecolor=magenta,
    urlcolor=cyan,
    citecolor=sgreen,
    bookmarks=true,
}
%\hypersetup{colorlinks, breaklinks, linkcolor=black,citecolor=black,filecolor=black,urlcolor=black} % Set up hyperlinks including colors for references, urls and citations

\usepackage{microtype} % Slightly tweak font spacing for aesthetics

\usepackage{listings}
\lstdefinelanguage{Coq}{
    morekeywords = [1]{Inductive, Fixpoint, Theorem, Lemma, Example},
    morekeywords = [2]{match, with, end},
    morekeywords = [3]{Require, Import, Set, Implicit, Arguments, Qed, Proof},
    morekeywords = [4]{},
    morekeywords = [5]{ },
    keywordstyle = [1]\color{dred},
    keywordstyle = [2]\color{dgreen},
    keywordstyle = [3]\color{codepurple},
    keywordstyle = [4]\color{orange},
    keywordstyle = [5]\color{lblue},
    sensitive = true,
    morecomment = [l]{//},
    morecomment = [s]{(*}{*)},
    morecomment = [s]{(**}{*)},
    commentstyle={\color{dgreen}},
    morestring = [b]",
    morestring = [b]',
    basicstyle={\small\ttfamily\bfseries},
    stringstyle={\ttfamily\small\color{orange}},
    numbers=left,
    numberstyle=\tiny\color{mygray},
    xrightmargin=0em,
    xleftmargin=3em,
    stepnumber=1,
    numbersep=1em,
    lineskip=-0.5ex,
    mathescape=true,
    showstringspaces=false,
    %    frame={tb},
    frame=none,
    breaklines=true,
    columns=[l]{fullflexible},
    keepspaces=true,
    literate=
        {á}{{\'a}}1
        {à}{{\`a}}1
        {é}{{\'e}}1
        {è}{{\`e}}1
        {ì}{{\`i}}1
        {ò}{{\`o}}1
        {ù}{{\`u}}1
        {∀}{{forall}}1
}

\lstnewenvironment{Coq}{\lstset{language={Coq},}}{}

\lstdefinestyle{Coq-style}{
	backgroundcolor=\color{backcolour},
	commentstyle=\color{codegreen},
	keywordstyle=\color{magenta},
	numberstyle=\tiny\color{codegray},
	stringstyle=\color{codepurple},
	basicstyle=\fontsize{9}{13}\selectfont\ttfamily,
%	basicstyle=\ttfamily\footnotesize,
	breakatwhitespace=false,
	breaklines=true,
	captionpos=b,
	keepspaces=true,
	numbers=left,
	numbersep=5pt,
	showspaces=false,
	showstringspaces=false,
	showtabs=false,
	tabsize=2
}
\renewcommand{\verb}{\lstinline}
\newcommand{\verbCoq}[1]{\lstinline[language=Coq,style=Coq-style]+#1+}
\newcommand{\vCoq}[1]{\lstinline[language=Coq,style=Coq-style]+#1+}
%%% lstlisting stop definitions


\makeatletter
\renewcommand{\fnum@figure}{\textsc{\figurename~\thefigure}} % Make the "Figure 1.1" text in small caps
\makeatother

%----------------------------------------------------------------------------------------
%	PAGE LAYOUT
%----------------------------------------------------------------------------------------

% The memoir class used in this template contains the ability to set the stock paper size and the trimmed size independently. It also has the ability to show trim lines showing where stock paper should be trimmed to get the final book size. This can all be a bit confusing so please see the memoir class documentation for more information.

% By default, the paper size is a4paper which is 29.7cm × 21cm. To change this, simply change "a4paper" in the \documentclass[a4paper,...]{memoir} command in thesis.tex to another size such as "letterpaper".
% By default, the trimmed size is 24cm x 17cm and trim lines are shown. To remove trim lines, simply remove "showtrims" from the \documentclass[showtrims,...]{memoir} command in thesis.tex. The size of the trimmed content is set with the \settrimmedsize{}{} command below.
% If you wish to remove trims and set the content to fit the paper size (i.e. no trimming at all), all you have to do is remove "showtrims" as above and comment out the \settrimmedsize{}{} command below.

%\setstocksize{24cm}{17cm} % Uncomment to manually set the stock size and override the setting in \documentclass
\settrimmedsize{24cm}{17cm}{*} % Change the trimmed area size or comment out this line entirely to fit the content to the paper size without trimming
\setlrmarginsandblock{37.125mm}{*}{0.9} % The first bracket specifies the spine margin, the second the edge margin and the third the ratio of the spine to the edge. Only one or two values are required and the remaining one(s) can be a star (*) to specify it is not needed. By default the edge margin is 10% smaller and
\setulmarginsandblock{37.125mm}{*}{*} % The first bracket specifies the upper margin, the second the lower margin and the third the ratio of the upper to the lower. Only one or two values are required and the remaining one(s) can be a star (*) to specify it is not needed.
\setmarginnotes{17pt}{51pt}{\onelineskip} % The size of marginal notes, the three values in curly brackets are \marginparsep, \marginparwidth and \marginparpush
\setheadfoot{\onelineskip}{2\onelineskip} % Sets the space available for the header and footer
\setheaderspaces{*}{2\onelineskip}{*} % Sets the spacing above and below the header
\setlength{\trimtop}{0pt} % Sets the spacing above the trimmed area, i.e. moved the trimmed area down the page if positive

% Comment the two lines below to reverse the position of the trimmed content on the stock paper, i.e. odd pages will have content on the right side instead of the left and even pages will have content on the left side instead of the right
\setlength{\trimedge}{\stockwidth}
\addtolength{\trimedge}{-\paperwidth}

\checkandfixthelayout % Makes sure your specifications are correct and implements them in the document

%----------------------------------------------------------------------------------------
%	CHAPTER HEADING STYLE
%----------------------------------------------------------------------------------------

\makeatletter
\makechapterstyle{thesis}{
\renewcommand{\chapternamenum}{}
\setlength{\beforechapskip}{0pt}
\setlength{\midchapskip}{0pt}
\setlength{\afterchapskip}{0pt}
\renewcommand{\chapnamefont}{\LARGE}
\renewcommand{\chapnumfont}{\chapnamefont}
\renewcommand{\chaptitlefont}{\chapnamefont}
\renewcommand{\printchapternum}{}
\renewcommand{\afterchapternum}{}
\renewcommand{\printchaptername}{}
\renewcommand{\afterchaptertitle}{\chapnumfont\hfill\thechapter\\\vspace*{-.3cm}\hrulefill\vspace*{6cm}\\}
}
\makeatother

%----------------------------------------------------------------------------------------
%	TABLE OF CONTENTS DEPTH
%----------------------------------------------------------------------------------------

\maxsecnumdepth{subsubsection}
\maxtocdepth{subsection}

%----------------------------------------------------------------------------------------
%	MATH THEOREM DEFINITIONS
%----------------------------------------------------------------------------------------

\theoremstyle{plain}
\newtheorem{thm}{Theorem}[section] % Defines the theorem environment
\newtheorem{prop}[thm]{Proposition} % Defines the proposition environment
\newtheorem{proof}{Proof}[section] % Defines the proof environment
\newtheorem{definition}{Definition}[section] % Defines the definition environment
\newtheorem{example}{Example}[section] % Defines the example environment
\newtheorem{rem}{Remark} % Defines the remark environment
\newtheorem{note}{Note}[section] % Defines the note environment

%----------------------------------------------------------------------------------------
%	CODE SNIPPET CONFIGURATION
%----------------------------------------------------------------------------------------

\lstset{
  basicstyle=\ttfamily\small,
  basewidth=0.55em,
  showstringspaces=false,
  numbers=left,
  numberstyle=\tiny,
  numbersep=2.5pt,
  keywordstyle=\bfseries\ttfamily,
  breaklines=true
}
% Examples of list environments for different programming languages, you will likely need to specify your own
\lstnewenvironment{pseudoc}{\lstset{frame=lines,language=C,mathescape=true}}{}
\lstnewenvironment{logs}{\lstset{frame=lines,basicstyle=\footnotesize\ttfamily,numbers=none}}{}
\lstnewenvironment{cc}{\lstset{frame=lines,language=C}}{}
\lstnewenvironment{c64}{\lstset{backgroundcolor=\color{c64},basewidth=0.65em,basicstyle=\commodoreface\color{c64light},numbers=none,framerule=10pt,rulecolor=\color{c64light},frame=tb,framexbottommargin=30pt}}{}
\lstnewenvironment{html}{\lstset{frame=lines,language=html,numbers=none}}{}
\lstnewenvironment{pseudo}{\lstset{frame=lines,mathescape=true,morekeywords={learn_string_domain, save_model}}}{}
\lstnewenvironment{pseudoctiny}{\lstset{language=C,mathescape=true,basicstyle=\tiny\sffamily}}{}
\lstnewenvironment{cctiny}{\lstset{language=C,basicstyle=\tiny\sffamily}}{}
\lstnewenvironment{pseudotiny}{\lstset{mathescape=true,basicstyle=\tiny\sffamily}}{}