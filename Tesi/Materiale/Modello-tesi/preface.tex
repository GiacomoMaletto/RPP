
\section*{Preface}
This thesis embraces all the efforts that I put during the last three years as a PhD student at Politecnico di Milano. I have been working under the supervision of Prof. S. Zanero and Prof. G. Serazzi, who is also the leader of the research group I am part of. In this time frame I had the wonderful opportunity of being ``initiated'' to research, which radically changed the way I look at things: I found my natural \emph{``thinking outside the box''} attitude --- that was probably well-hidden under a thick layer of lack-of-opportunities, I took part of very interesting joint works --- among which the year I spent at the Computer Security Laboratory at UC Santa Barbara is at the first place, and I discovered the Zen of my life.

My research is all about \emph{computers} and every other technology possibly related to them. Clearly, the way I look at computers has changed a bit since when I was seven. Still, I can remember me, typing on that \textsf{Commodore} 64 in front of a tube TV screen, trying to get that d---n routine written in \textsf{Basic} to work. I was just playing, obviously, but when I recently found a picture of me in front of that screen...it all became clear.

So, although my attempt of writing a program to authenticate myself was a little bit naive --- being limited to a print instruction up to that point apart, of course --- I thought \emph{``maybe I am not in the wrong place, and the fact that my research is still about security is a good sign''}!

Many years later, this work comes to life. There is a humongous amount of people that, directly or indirectly, have contributed to my research and, in particular, to this work. Since my first step into the lab, I will not, ever, be thankful enough to Stefano, who, despite my skepticism, convinced me to submit that application for the PhD program. For trusting me since the very first moment I am thankful to Prof. G. Serazzi as well, who has been always supportive. For hosting and supporting my research abroad I thank Prof. G. Vigna, Prof. C. Kruegel, and Prof. R. Kemmerer. Also, I wish to thank Prof. M. Matteucci for the great collaboration, Prof. I. Epifani for her insightful suggestions and Prof. H. Bos for the detailed review and the constructive comments.

On the colleagues-side of this acknowledgments I put all the fellows of Room 157, Guido, the crew of the seclab and, in particular, Wil with whom I shared all the pain of paper writing between Sept '08 and Jun '09.

On the friends-side of this list Lorenzo and Simona go first, for being our family.

I have tried to translate in simple words the infinite gratitude I have and will always have to Valentina and my parents for being my fixed point in my life. Obviously, I failed.

\begin{flushright}
\textsc{\theauthor}\\
Milano\\
September 2009
\end{flushright}

\cleartoverso % Force a break to an even page
