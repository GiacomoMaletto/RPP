\section{Expressiveness of $ \RPP $}
\label{section:Completeness of RPP}

We start to show how to represent stacks as natural numbers in $ \RPP $.
\Ie, given $ x_1, x_2, \ldots, x_n  \in \Nat$
we can encode them into $ \la   x_n, \ldots, \la x_2, x_1 \ra \ldots \ra \in \Nat$
by means of a sequence of $ \rprPush$ on a suitable ancillae, while a corresponding
sequence of $ \rprPop $ can decompose it as expected.

%%%%%%%%%%%%%%%
\begin{proposition}[Representing stacks in $ \RPP $]
\label{proposition:Representing stacks in RPP}
Functions $ \rprPush, \rprPop\in\RPP^{10}$ exist such that:
\begin{align*}
\arraycolsep=1.4pt
\begin{array}{rcl}
 \left. {\scriptsize \begin{array}{r} 
                       s\\ x\\ 0^{8}
                     \end{array}} \right[
 & \rprPush &
 \left] {\scriptsize \begin{array}{l}
                       \CP(s,x)\\ 0\\ 0^{8}
                     \end{array} } \right.
\end{array}
&&
\arraycolsep=1.4pt
\begin{array}{rcl}
 \left. {\scriptsize \begin{array}{r} 
                       \CP(s,x)\\ 0\\ 0^{8}
                     \end{array}} \right[
 & \rprPop &
 \left] {\scriptsize \begin{array}{l}
                       s\\ x\\ 0^{8}
                     \end{array} } \right.
\end{array}
\enspace ,
\end{align*}
for every value $ s \in\mathbb{N}$ (the ``stack'') and $ x \in\mathbb{N}$ (the element 
one has to push on or pop out the stack.)
\end{proposition}
\begin{prf}
The function 
$\operatorname{zClean}:= 
(\rprPCom{\rprId^2}{\CU})\, 
 \rprSeq \rprDec^{10}_{4;1}\, 
 \rprSeq \rprDec^{10}_{5;2}\, 
 \rprSeq \rprInv{(\rprPCom{\rprId^2}{\CU})} 
$ 
is such that:
$$\begin{array}{rcl}
 \left. {\scriptsize \begin{array}{r} 
                       s\\ x\\ \CP(s,x)\\ \\[-1mm] 0^{7}
                     \end{array}} \right[
 & \operatorname{zClean} &
 \left] {\scriptsize \begin{array}{l}
                       0\\ 0\\ \CP(s,x)\\  \\[-1mm] 0^{7}
                     \end{array} } \right.
\end{array}$$
so,
$\rprPush : =
(\rprPCom{\CP} {\rprId^{6}})\, 
          \rprSeq \operatorname{zClean}
       \rprSeq (\rprPCom{\rprSwap{3}{1,2,3}{3,2,1}}{\rprId^{6}})$ 
and 
$\rprPop  := \rprInv{\rprPush}$.
\qed
\end{prf}



\subsection{$ \RPP $ is $ \PRF $-complete} $ \RPP $ is expressive enough to represent 
the class $ \PRF $ of Primitive Recursive Functions \cite{cutland1980book,odifreddi1989book}, which we recall for easy 
of reference. $ \PRF $ is the smallest class of functions on natural numbers that:
\begin{itemize}
\item 
 contains the functions $\prZero^n(x_1,\ldots,x_n):=0$, the successors 
$\prSucc^n_i(x_1,\ldots,x_n) := x_i + 1$  and  the projections 
$ \prProj{n}{i}(x_1,\ldots,x_n) := x_i $ for all $1\leq i\leq n$;

\item 
 is closed under composition, viz. $ \PRF $  includes the function $f(\myVec{x}) := h(g_1(\myVec{x}), \ldots , g_m(\myVec{x}))$)
whenever there are  $g_1, \ldots, g_m, h\in\PRF$ of suitable arity; and,

\item 
 is closed under primitive recursion, viz. $ \PRF $ includes the function $f$ defined   by means of the schema $f(\myVec{x},0):= g(\myVec{x})$ and 
$f(\myVec{x},y+1):= h(f(\myVec{x},y),\myVec{x},y)$ whenever there are  $g,h\in\PRF$  of suitable arity.
\end{itemize}


In the following, $\PRF^n$ denotes the class of functions $f\in \PRF$ with arity $n$.

%%%%%%%%%%%%%
\begin{definition}[$ \RPP $-definability of any $ f\in\PRF $]
\label{definition:RPP-definability of any f in PRF}
Let $n, a\in\mathbb N$ and $ f\in\PRF^n $. We say that $ f $ is $ \RPP^{n+a+1} $-definable  whenever
there is a function  $ d_f\in\RPP^{n+a+1} $  such that, for every $x, x_1, \ldots, x_n\in\Nat$:
\begin{align*}
\arraycolsep=1.4pt
\begin{array}{rcl}
 \left. {\scriptsize 
            \begin{array}{r}
             x \\ x_1 \\ \cdots \\ x_n
             \\[1mm]
              0^a
            \end{array} 
         } \right[
 & d_f &
 \left] {\scriptsize 
            \begin{array}{l}
             x + f (x_1, \ldots, x_n) \\ x_1 \\ \cdots \\ x_n
             \\[1mm]
              0^a
            \enspace .
            \end{array} 
        } \right.
\end{array}
\end{align*}
\end{definition}

Three observations are worth doing.
Definition~\ref{definition:RPP-definability of any f in PRF} relies on $a+1$ arguments used as ancillae.
If $ f\in\PRF^n $ is $ \RPP^{n+a+1} $-definable, then $f$ is also
$ \RPP^{n+k} $-definable for any $k\geq a+1$ by using \rprWeaName (cf. Section \ref{section:Generalizations inside RPP}).
Definition \ref{definition:RPP-definability of any f in PRF} improves 
the namesake one in \cite[Def.3.1, p.236]{PaoliniPiccoloRoversiICTCS2015} because it gets rid of an output
line:  this line was a ``waste bin'' containing part of the computation trace that, eventually, was useless. 
The encoding proposed in this paper does not need the ``waste bin'' anymore.
%%%%%%%%%%%%%%%%%%%%%%%%%%%%%%%%%%%%%%%%%%%%%%%%%%%%%%%%%%%%%%%%%%%
%%%%%%%%%%%%%%%%%%%%%%%%%%%%%%%%%%%%%%%%%%%%%%%%%%%%%%%%%%%%%%%%%%%
%%%%%%%%%%%%%%%%%%%%%%%%%%%%%%%%%%%%%%%%%%%%%%%%%%%%%%%%%%%%%%%%%%%
%%%%%%%%%%%%%   VECCHIA COMPLETEZZA INIZIO
%%%%%%%%%%%%%\begin{definition}[Embedding $ \prrpr{(\cdot)} $ of $ \PR $ into $ \RPP $]
%%%%%%%%%%%%%The function $\prrpr{(\cdot)}: \PR \longrightarrow \RPP $ is inductively defined
%%%%%%%%%%%%%on the structure of its argument. 
%%%%%%%%%%%%%\begin{itemize}
%%%%%%%%%%%%%\item 
%%%%%%%%%%%%%The basic cases are as follows:  
%%%%%%%%%%%%%\begin{align*}
%%%%%%%%%%%%%%%%%%%% 
%%%%%%%%%%%%%\prrpr{\prZero^n}
%%%%%%%%%%%%%& = \rprId^2
%%%%%%%%%%%%%\\
%%%%%%%%%%%%%%%%%%%% 
%%%%%%%%%%%%%\prrpr{\prSucc}
%%%%%%%%%%%%%& = (\rprPCom{\rprId^1}
%%%%%%%%%%%%%            {\rprSucc}) \rprSeq \rprInc^2_{2;1} \rprSeq \rprInv{(\rprPCom{\rprId^1}
%%%%%%%%%%%%%                                                                     {\rprSucc})}
%%%%%%%%%%%%%\\
%%%%%%%%%%%%%\prrpr{(\prProj{k}{i})}
%%%%%%%%%%%%%& = \rprInc^{k+1}_{i;1} 
%%%%%%%%%%%%%\enspace
%%%%%%%%%%%%%\end{align*}
%%%%%%%%%%%%%where $ \prrpr{\prZero}, \prrpr{\prSucc} \in\RPP^{2} $
%%%%%%%%%%%%%and $ \prrpr{(\prProj{k}{i})} \in\RPP^{k+1} $.
%%%%%%%%%%%%%
%%%%%%%%%%%%%\item 
%%%%%%%%%%%%%%%%%%%%%%%%%%%%%% composition
%%%%%%%%%%%%%Let $ g_1,\ldots,g_k\in\PR^{n} $ and $ h\in\PR^{k} $ in $ \prCom{h,g_1,\ldots,g_k}\in\PR^{n}$.
%%%%%%%%%%%%%By inductive hypothesis and weakening
%%%%%%%%%%%%%we can assume that $ \prrpr{g_1},\ldots,\prrpr{g_k}\in\PR^{n+p} $ and 
%%%%%%%%%%%%%$ \prrpr{h}\in\PR^{k+q} $ exist, for some $ p $ and $ q $ big enough.
%%%%%%%%%%%%%Then $ \prrpr{\prCom{h,g_1,\ldots,g_k}}\in\RPP^{1+n+m} $ is:
%%%%%%%%%%%%%\begin{align*}
%%%%%%%%%%%%%\prrpr{\prCom{h,g_1,\ldots,g_k}}
%%%%%%%%%%%%%& =
%%%%%%%%%%%%%{H}
%%%%%%%%%%%%%\rprSeq
%%%%%%%%%%%%%(\rprPCom{\rprId}{\rprPCom{\prrpr{h}}{\rprId^{(n+p)k}}}) \rprSeq
%%%%%%%%%%%%%(\rprPCom{\rprInc^2_{2;1}}{\rprId^{k+q+(n+p)k}}) \rprSeq
%%%%%%%%%%%%%\rprInv{H}
%%%%%%%%%%%%%\enspace ,        
%%%%%%%%%%%%%\end{align*}
%%%%%%%%%%%%%where:
%%%%%%%%%%%%%\begin{align*}
%%%%%%%%%%%%%H
%%%%%%%%%%%%%& =
%%%%%%%%%%%%%(\rprPCom{\rprId}{\rprPCom{\rprGSwap{nk}_1}{\rprPCom{\rprId^{(p+1)k}}{\rprId^{1+q}}}})\rprSeq
%%%%%%%%%%%%%(\rprPCom{\rprId}
%%%%%%%%%%%%%         {\rprPCom{\overbrace{\rprPCom{\rprNabla^{k-1}}{\rprPCom{\cdots}{\rprNabla^{k-1}}}}^{n}}
%%%%%%%%%%%%%                  {\rprPCom{\rprId^{(p+1)k}}
%%%%%%%%%%%%%                           {\rprId^{1+q}}}})\rprSeq
%%%%%%%%%%%%%\\ & \phantom{= }\ \
%%%%%%%%%%%%%(\rprPCom{\rprId}{\rprPCom{\rprGSwap{(n+p+1)k}_2}{\rprId^{1+q}}})\rprSeq
%%%%%%%%%%%%%(\rprPCom{\rprId}
%%%%%%%%%%%%%         {\rprPCom{\rprPCom{\prrpr{g_1}}{\rprPCom{\cdots}{\prrpr{g_k}}}}
%%%%%%%%%%%%%                  {\rprId^{1+q}}})\rprSeq
%%%%%%%%%%%%%\\ & \phantom{= }\ \
%%%%%%%%%%%%%(\rprPCom{\rprId}{\rprGSwap{(n+p+1)k+1+q}_3})\rprSeq
%%%%%%%%%%%%%(\rprPCom{\rprId}{\rprPCom{\prrpr{h}}{\rprId^{(n+p)k}}})
%%%%%%%%%%%%%\\
%%%%%%%%%%%%%\rprGSwap{}_1
%%%%%%%%%%%%%& =
%%%%%%%%%%%%%\rprSwap
%%%%%%%%%%%%%{nk}
%%%%%%%%%%%%%{1,2 ,\ldots,n ,n+1,\ldots,n+k-1,\ldots,n+(n-1)(k-1)+1,\ldots,nk}
%%%%%%%%%%%%%{k,2k,\ldots,nk,1  ,\ldots,k-1  ,\ldots,(n-1)k+1      ,\ldots,nk-1}
%%%%%%%%%%%%%%\\
%%%%%%%%%%%%%%\rprGSwap{}_2
%%%%%%%%%%%%%%& =
%%%%%%%%%%%%%%\rprSwap
%%%%%%%%%%%%%%{(n+1+p)k}
%%%%%%%%%%%%%%{1,\ldots,k,\ldots,(n-1)k+1,\ldots,nk,
%%%%%%%%%%%%%%  nk+1,\ldots,(n+1)k,
%%%%%%%%%%%%%%   (n+1)k+1,\ldots,(n+1)k+p,\ldots,(n+1)k+(k-1)p,\ldots,(n+1)k+kp
%%%%%%%%%%%%%%}
%%%%%%%%%%%%%%{2,\ldots,(k-1)(n+1+p)+2,\ldots,n+1,\ldots,(k-1)(n+1+p)+(n+1),
%%%%%%%%%%%%%%  1,\ldots,(k-1)(n+1+p)+1,
%%%%%%%%%%%%%%   n+1+1,\ldots,n+1+p,\ldots,
%%%%%%%%%%%%%%    (k-1)(n+1+p)+(n+1)+1,\ldots,(k-1)(n+1+p)+(n+1)+p
%%%%%%%%%%%%%%}
%%%%%%%%%%%%%%\\
%%%%%%%%%%%%%%\rprGSwap{}_3
%%%%%%%%%%%%%%& =
%%%%%%%%%%%%%%\rprSwap
%%%%%%%%%%%%%%{(n+p+1)k+1+q}
%%%%%%%%%%%%%%{1,2      ,\ldots,1+n    ,2+n    ,\ldots,1+n+p,
%%%%%%%%%%%%%%  \ldots,
%%%%%%%%%%%%%%   (k-1)(1+n+p)+1,\ldots,k(1+n)+(k-1)p,
%%%%%%%%%%%%%%    k(1+n)+(k-1)p+1      ,\ldots,k(1+n+p),
%%%%%%%%%%%%%%     k(1+n+p)+1,k(1+n+p)+2,\ldots,k(1+n+p)+1+q
%%%%%%%%%%%%%%}
%%%%%%%%%%%%%%{2,1+k+q+1,\ldots,1+k+q+n,2+k+q+n,\ldots,1+k+q+n+p,
%%%%%%%%%%%%%%  \ldots,
%%%%%%%%%%%%%%   1+k           ,\ldots,1+k+q+(k-1)(n+p)+n),
%%%%%%%%%%%%%%    1+k+q+(k-1)(n+p)+n+1),\ldots,1+k+q+(k-1)(n+p)+n+p),
%%%%%%%%%%%%%%     1         ,k+2       ,\ldots,1+k+q
%%%%%%%%%%%%%%}
%%%%%%%%%%%%%\end{align*}
%%%%%%%%%%%%%
%%%%%%%%%%%%%%\begin{align*}
%%%%%%%%%%%%%%\arraycolsep=1.4pt
%%%%%%%%%%%%%%\begin{array}{rcl}
%%%%%%%%%%%%%% \left. {\scriptsize 
%%%%%%%%%%%%%%         \begin{array}{r} 
%%%%%%%%%%%%%%           x_{1}\\ x_{2} \\ \ldots\\ x_{n}\\x_{n+1}\\\ldots\\x_{n+k-1}\\
%%%%%%%%%%%%%%             \ldots\\ \ldots\\ x_{n+(n-1)(k-1)+1} \\ \ldots\\ x_{n+(n-1)(k-1)+(k-1)}
%%%%%%%%%%%%%%         \end{array}} 
%%%%%%%%%%%%%%\right[
%%%%%%%%%%%%%% & \rprGSwap{nk}_1 &
%%%%%%%%%%%%%% \left] {\scriptsize 
%%%%%%%%%%%%%%         \begin{array}{l}
%%%%%%%%%%%%%%           x_{k}\\ x_{2k} \\ \ldots\\ x_{nk}\\x_{1}\\\ldots\\x_{k-1}\\
%%%%%%%%%%%%%%             \ldots\\ \ldots\\ x_{(n-1)k+1} \\ \ldots\\ x_{(n-1)k+(k-1)}
%%%%%%%%%%%%%%         \end{array} } \right.
%%%%%%%%%%%%%%\end{array}
%%%%%%%%%%%%%%\end{align*}
%%%%%%%%%%%%%\begin{align*}
%%%%%%%%%%%%%\arraycolsep=1.4pt
%%%%%%%%%%%%%\begin{array}{rcl}
%%%%%%%%%%%%% \left. {\scriptsize 
%%%%%%%%%%%%%         \begin{array}{r} 
%%%%%%%%%%%%%           x_{1}\\ \ldots\\ x_{k} \\\ldots\ldots\\x_{(n-1)k+1}\\\ldots\\x_{nk}\\
%%%%%%%%%%%%%             x_{nk+1} \\ \ldots\\ x_{(n+1)k}\\ 
%%%%%%%%%%%%%              x_{(n+1)k+1}\\ \ldots\\ x_{(n+1)k+p}\\\ldots\ldots\\
%%%%%%%%%%%%%               x_{(n+1)k+(k-1)p}\\ \ldots\\ x_{(n+1)k+kp}
%%%%%%%%%%%%%         \end{array}} 
%%%%%%%%%%%%%\right[
%%%%%%%%%%%%% & \rprGSwap{(n+1+p)k}_2 &
%%%%%%%%%%%%% \left] {\scriptsize 
%%%%%%%%%%%%%         \begin{array}{l}
%%%%%%%%%%%%%           x_{2} \\ \ldots\\ x_{(k-1)(n+1+p)+2}\\\ldots\ldots\\ x_{n+1}\\\ldots\\ x_{(k-1)(n+1+p)+(n+1)}\\
%%%%%%%%%%%%%             x_{1}\\ \ldots\\ x_{(k-1)(n+1+p)+1}\\ 
%%%%%%%%%%%%%              x_{n+1+1}\\ \ldots\\ x_{n+1+p}\\\ldots\ldots\\
%%%%%%%%%%%%%                x_{(k-1)(n+1+p)+(n+1)+1}\\ \ldots\\ x_{(k-1)(n+1+p)+(n+1)+p}
%%%%%%%%%%%%%         \end{array} } \right.
%%%%%%%%%%%%%\end{array}
%%%%%%%%%%%%%\end{align*}
%%%%%%%%%%%%%\begin{align*}
%%%%%%%%%%%%%\arraycolsep=1.4pt
%%%%%%%%%%%%%\begin{array}{rcl}
%%%%%%%%%%%%% \left. {\scriptsize 
%%%%%%%%%%%%%         \begin{array}{r} 
%%%%%%%%%%%%%           x_{1}\\ x_{2}       \\ \ldots \\x_{1+n}  \\x_{1+n+1}    \\\ldots\\x_{1+n+p}
%%%%%%%%%%%%%             \\ \ldots\\ \ldots\\ 
%%%%%%%%%%%%%           x_{(k-1)(1+n+p)+1}\\ x_{(k-1)(1+n+p)+1+1} \\ \ldots \\ x_{k(1+n)+(k-1)p}\\
%%%%%%%%%%%%%           x_{k(1+n)+(k-1)p+1} \\\ldots\\x_{k(1+n+p)}\\
%%%%%%%%%%%%%           x_{k(1+n+p)+1}\\x_{k(1+n+p)+1+1}\\\ldots\\x_{k(1+n+p)+1+q}
%%%%%%%%%%%%%         \end{array}} 
%%%%%%%%%%%%%\right[
%%%%%%%%%%%%% & \rprGSwap{(n+p+1)k+1+q}_3 &
%%%%%%%%%%%%% \left] {\scriptsize 
%%%%%%%%%%%%%         \begin{array}{l}
%%%%%%%%%%%%%           x_{2}\\ x_{1+k+q+1} \\ \ldots \\x_{1+k+q+n}\\x_{1+k+q+n+1}\\\ldots\\x_{1+k+q+n+p}
%%%%%%%%%%%%%             \\ \ldots\\ \ldots\\ 
%%%%%%%%%%%%%           x_{1+k}          \\ x_{1+k+q+(k-1)(n+p)+1)} \\ \ldots \\ x_{1+k+q+(k-1)(n+p)+n)}\\
%%%%%%%%%%%%%                          x_{1+k+q+(k-1)(n+p)+n+1)}\\\ldots\\x_{1+k+q+(k-1)(n+p)+n+p)}\\
%%%%%%%%%%%%%           x_{1} \\ x_{1+k+1}\\\ldots\\x_{1+k+q}
%%%%%%%%%%%%%         \end{array} } \right.
%%%%%%%%%%%%%\end{array}
%%%%%%%%%%%%%\end{align*}
%%%%%%%%%%%%%\begin{remark}
%%%%%%%%%%%%%Some of the expressions that identify the index of an input or of an output in $ \rprGSwap{(n+1+p)k}_2$
%%%%%%%%%%%%%or $ \rprGSwap{(n+p+1)k+1+q}_3 $ can be simplified. We leave them
%%%%%%%%%%%%%in an explicit form to easy the reconstruction of the pattern for their generation.
%%%%%%%%%%%%%
%%%%%%%%%%%%%The intuition behind the structure of $ \prrpr{\prCom{h,g_1,\ldots,g_k}} $ is the one
%%%%%%%%%%%%%we may expect, up to the required number of \temporarychannels. 
%%%%%%%%%%%%%The point is to supply enough arguments to $ \prrpr{h} $ which are the
%%%%%%%%%%%%%results we get from $ \prrpr{g_1}, \ldots, \prrpr{g_k} $. Every $ \prrpr{g_i} $ needs its 
%%%%%%%%%%%%%own copy of the initial arguments $ x_1,\ldots,x_n $ which we produce by means of
%%%%%%%%%%%%%$ n $ instances of $ \rprNabla^{k-1} $. Every among 
%%%%%%%%%%%%%$ \rprGSwap{nk}_1, \rprGSwap{(n+1+p)k}_2$ and
%%%%%%%%%%%%%$ \rprGSwap{(n+p+1)k+1+q}_3 $ rearranges the inputs as required by the inductive
%%%%%%%%%%%%%hypothesis on the structure of $ \prrpr{g_1}, \ldots, \prrpr{g_k} $ and $ \prrpr{h} $.
%%%%%%%%%%%%%\end{remark}
%%%%%%%%%%%%%%%%%%%%%%%%%%%%%%%
%%%%%%%%%%%%%
%%%%%%%%%%%%%\item 
%%%%%%%%%%%%%%%%%%%%%%%%%%%%%% iteration
%%%%%%%%%%%%%Let $ g\in\PR^{k} $ and $ h\in\PR^{k+2} $ in $ \prRec{g}{h}\in\PR^{k+1}$.
%%%%%%%%%%%%%By inductive hypothesis and weakening
%%%%%%%%%%%%%we can assume that $ \prrpr{g}\in\PR^{1+k+m} $ and $ \prrpr{h}\in\PR^{1+(k+2)+m} $,
%%%%%%%%%%%%%for some $ m \geq 14 $, exist. 
%%%%%%%%%%%%%Then $ \prrpr{\prRec{h}{g}} \in \RPP^{k+m+5} $ is:
%%%%%%%%%%%%%\begin{align*}
%%%%%%%%%%%%%\prrpr{\prRec{h}{g}}
%%%%%%%%%%%%%& =
%%%%%%%%%%%%% \rprGSwap{} \rprSeq
%%%%%%%%%%%%% (\rprPCom{\rprId^4}{G}) \rprSeq
%%%%%%%%%%%%% (\rprPCom{\rprId^2}{\rprIt{4+k+m}{1}{H,Z,\rprId}})\, \rprSeq
%%%%%%%%%%%%%\\ & \phantom{= }\
%%%%%%%%%%%%% \rprInc^{5+k+m}_{5+k;1}\,\rprSeq
%%%%%%%%%%%%%\\ & \phantom{= }\
%%%%%%%%%%%%%  \rprInv{\left(
%%%%%%%%%%%%%           \rprGSwap{} \rprSeq
%%%%%%%%%%%%%           (\rprPCom{\rprId^4}{G}) \rprSeq
%%%%%%%%%%%%%           (\rprPCom{\rprId^2}{\rprIt{4+k+m}{1}{H,Z,\rprId}})
%%%%%%%%%%%%%          \right)}
%%%%%%%%%%%%%\end{align*}
%%%%%%%%%%%%%where:
%%%%%%%%%%%%%\begin{align*}
%%%%%%%%%%%%%\rprGSwap{}
%%%%%%%%%%%%%& =
%%%%%%%%%%%%%\rprSwap{5+k+m}
%%%%%%%%%%%%%        {1,2,3,\ldots,3+k-1,3+k,4+k,5+k,6+k,\ldots,5+k+m}
%%%%%%%%%%%%%        {1,2,6,\ldots,6+k-1,4  ,5  ,3  ,6+k,\ldots,5+k+m}
%%%%%%%%%%%%%\\
%%%%%%%%%%%%%G
%%%%%%%%%%%%%& =
%%%%%%%%%%%%%\rprPCom{\rprId^4}
%%%%%%%%%%%%%        {\prrpr{g}}
%%%%%%%%%%%%%\\
%%%%%%%%%%%%%Z
%%%%%%%%%%%%%& = 
%%%%%%%%%%%%%\rprPCom{\rprId^2}
%%%%%%%%%%%%%        {\rprSwap{3+k+m}
%%%%%%%%%%%%%                 {1,2,3  ,4,\ldots,4+k,5+k,\ldots,3+k+m}
%%%%%%%%%%%%%                 {1,2,3+k,2,\ldots,2+k,5+k,\ldots,3+k+m}}
%%%%%%%%%%%%%\\
%%%%%%%%%%%%%H
%%%%%%%%%%%%%& =
%%%%%%%%%%%%%\left(
%%%%%%%%%%%%%\rprPCom{\rprId^1}
%%%%%%%%%%%%%        {\rprPCom{\rprSucc}
%%%%%%%%%%%%%                 {\rprPCom{\rprSwap{k+1}
%%%%%%%%%%%%%                                   {1  ,2,\ldots,k+1}
%%%%%%%%%%%%%                                   {k+1,1,\ldots,k  }}
%%%%%%%%%%%%%                          {\rprId^{m}}}}\right) 
%%%%%%%%%%%%%\rprSeq 
%%%%%%%%%%%%%\\&\phantom{= } \ \
%%%%%%%%%%%%%\prrpr{h} \rprSeq (\rprPCom{\rprId^{k+2}}{\rprPush})\, \rprSeq 
%%%%%%%%%%%%%\\&\phantom{= } \ \
%%%%%%%%%%%%%\rprSwap{3+k+m}
%%%%%%%%%%%%%        {1  ,2,\ldots,2+k,2+k+1,2+k+2,5+k,\ldots,3+k+m}
%%%%%%%%%%%%%        {3+k,2,\ldots,2+k,2+k+2,1    ,5+k,\ldots,3+k+m}
%%%%%%%%%%%%%\enspace .
%%%%%%%%%%%%%\end{align*}
%%%%%%%%%%%%%%%%%%%%%%%%%%%%%%%
%%%%%%%%%%%%%
%%%%%%%%%%%%%\begin{remark}
%%%%%%%%%%%%%Using $ \rprPush $ and $ \rprPop $ inside $ \prrpr{\prRec{h}{g}} $ requires that we can supply
%%%%%%%%%%%%%at least 14 inputs set to the value $ 0 $ in accordance with 
%%%%%%%%%%%%%Proposition~\ref{proposition:Representing stacks in RPP}. This is why we require $ m\geq 14 $
%%%%%%%%%%%%%which is a condition we can satisfy by using the possibility of \rprWeaName functions inside 
%%%%%%%%%%%%%$ \RPP $.
%%%%%%%%%%%%%
%%%%%%%%%%%%%The intuition behind the structure of $ \prrpr{\prRec{h}{g}} $ is that we use 
%%%%%%%%%%%%%the \emph{second} argument to drive a \rprSComName of the translation of $ h\in\PRF $ which must be preceded 
%%%%%%%%%%%%%by a call to the translation of $ g\in\PRF $. The function $ Z $ serves to correctly re-organize the
%%%%%%%%%%%%%outputs of $ \prrpr{g} $ whenever the \rprSComName is $ 0 $-steps long.
%%%%%%%%%%%%%
%%%%%%%%%%%%%The use of $ \rprPush $ is crucial to pile up in the right order the intermediate values that
%%%%%%%%%%%%%$ \prrpr{h} $ produces and which depend on the length of the \rprSComName. Once got and stored 
%%%%%%%%%%%%%the result, $ \rprPop $ extracts every intermediate value to revert every step of the \rprSComName.
%%%%%%%%%%%%%\qed
%%%%%%%%%%%%%\end{remark}
%%%%%%%%%%%%%\end{itemize}
%%%%%%%%%%%%%\end{definition}
%%%%%%%%%%%%%
%%%%%%%%%%%%%
%%%%%%%%%%%%%%%%%%%%
%%%%%%%%%%%%%\begin{theorem}[$\RPP $ is $ \PRF $-complete]
%%%%%%%%%%%%%\label{theorem:RPP is PRF-complete}
%%%%%%%%%%%%%If $ f\in\PR $, then $ \prrpr{f}$ is $ \RPP $-definable.
%%%%%%%%%%%%%\end{theorem}
%%%%%%%%%%%%%\begin{prf}
%%%%%%%%%%%%%It is enough to check that $ \prrpr{f} $ satisfies the constraints
%%%%%%%%%%%%%of Definition~\ref{definition:RPP-definability of any f in PRF}, proceeding
%%%%%%%%%%%%%by structural induction on $ f $. In particular,
%%%%%%%%%%%%%if $ f $ is $ \prCom{h,g_1,\ldots,g_k} $:
%%%%%%%%%%%%%\begin{align*}
%%%%%%%%%%%%%\arraycolsep=1.4pt
%%%%%%%%%%%%%\begin{array}{rcl}
%%%%%%%%%%%%% \left. {\scriptsize 
%%%%%%%%%%%%%            \begin{array}{r}
%%%%%%%%%%%%%             x \\[0.5mm] \myVec{x}
%%%%%%%%%%%%%             \\[0.5mm] 0^{n} 
%%%%%%%%%%%%%             \\ n
%%%%%%%%%%%%%           \left \{
%%%%%%%%%%%%%           \begin{array}{l}
%%%%%%%%%%%%%             0^{k-1}\\ \cdots \\[0.5mm] 0^{k-1} 
%%%%%%%%%%%%%           \end{array} 
%%%%%%%%%%%%%           \right .
%%%%%%%%%%%%%           \\ 
%%%%%%%%%%%%%           n \left \{
%%%%%%%%%%%%%           \begin{array}{l}
%%%%%%%%%%%%%             0^{j}\\ \cdots \\[0.5mm] 0^{j} 
%%%%%%%%%%%%%           \end{array} 
%%%%%%%%%%%%%           \right .
%%%%%%%%%%%%%            \end{array} 
%%%%%%%%%%%%%         } \right[
%%%%%%%%%%%%% & \prrpr{\prCom{h,g_1,\ldots,g_k}} &
%%%%%%%%%%%%% \left] {\scriptsize 
%%%%%%%%%%%%%            \begin{array}{l}
%%%%%%%%%%%%%             x+f(g_1(\myVec{x}),\ldots,g_n(\myVec{x})) \\[0.5mm] \myVec{x}
%%%%%%%%%%%%%             \\[0.5mm] 0^{n} 
%%%%%%%%%%%%%             \\ \!\!
%%%%%%%%%%%%%           \left .
%%%%%%%%%%%%%           \begin{array}{l}
%%%%%%%%%%%%%             0^{k-1}\\ \cdots \\[0.5mm] 0^{k-1} 
%%%%%%%%%%%%%           \end{array} 
%%%%%%%%%%%%%           \right \} n
%%%%%%%%%%%%%             \\ \!\!
%%%%%%%%%%%%%           \left .
%%%%%%%%%%%%%           \begin{array}{l}
%%%%%%%%%%%%%             0^{j}\\ \cdots \\[0.5mm] 0^{j} 
%%%%%%%%%%%%%           \end{array} 
%%%%%%%%%%%%%           \right \} n
%%%%%%%%%%%%%            \end{array} 
%%%%%%%%%%%%%        } \right.
%%%%%%%%%%%%%\end{array}
%%%%%%%%%%%%%\end{align*}
%%%%%%%%%%%%%and if $ f $ is $ \prRec{g}{h} $, then:
%%%%%%%%%%%%%\begin{align*}
%%%%%%%%%%%%%\arraycolsep=1.4pt
%%%%%%%%%%%%%\begin{array}{rcl}
%%%%%%%%%%%%% \left. {\scriptsize 
%%%%%%%%%%%%%            \begin{array}{r}
%%%%%%%%%%%%%             x \\ y \\ \myVec{x}
%%%%%%%%%%%%%             \\[0.5mm] 0^{3+m} 
%%%%%%%%%%%%%            \end{array} 
%%%%%%%%%%%%%         } \right[
%%%%%%%%%%%%% & \prrpr{\prRec{g}{h}} &
%%%%%%%%%%%%% \left] {\scriptsize 
%%%%%%%%%%%%%            \begin{array}{l}
%%%%%%%%%%%%%             x+
%%%%%%%%%%%%%             h(\myVec{x},y-1,
%%%%%%%%%%%%%                h(\myVec{x},y-2,\ldots
%%%%%%%%%%%%%                  h(\myVec{x},1,g(\myVec{x}))\ldots)) 
%%%%%%%%%%%%%             \\ y \\ \myVec{x}
%%%%%%%%%%%%%             \\[0.5mm] 0^{3+m} 
%%%%%%%%%%%%%            \end{array} 
%%%%%%%%%%%%%        } \right.
%%%%%%%%%%%%%\end{array}
%%%%%%%%%%%%%\enspace ,
%%%%%%%%%%%%%\end{align*}
%%%%%%%%%%%%%where $ \myVec{x} $ stands for $x_1, \ldots, x_k$. 
%%%%%%%%%%%%%\qed
%%%%%%%%%%%%%\end{prf}
%%%%%%%%%%%%%}{
%%%%%%%%%%%%%   VECCHIA COMPLETEZZA FINE
%%%%%%%%%%%%%%%%%%%%%%%%%%%%%%%%%%%%%%%%%%%%%%%%%%%%%%%%%%%%%%%%%%%
\begin{theorem}[$\RPP $ is $ \PRF $-complete]
\label{theorem:RPP is PRF-complete}
If $ f\in\PRF^n $ then $ \prrpr{f}$ is $ \RPP^{n+a+1} $-definable, for some $a\in\mathbb{N}$.
\end{theorem}
\begin{prf}
\begin{table}%%%%%%%%%%%%%%%%%%%%%%%%%%%%%%%%%%%%%%%%%%%%%%%%%%%%%%%%%%%%%%%%%%%%%%%%%%%%%%%%%%%%%%
\hrule   
$$\begin{array}{c}
\arraycolsep=1.4pt
\begin{array}{rcl}
 \left. {\scriptsize 
            \begin{array}{r}
             x \\ x_1 \\ \cdots \\ x_n
             \\[1mm]
              0^m\\
           k\left\{
              \begin{array}{r}
             r_1 \\ \cdots\\ r_{i-1}\\[1.5mm] 0\\[.5mm] r_{i+1}\\\cdots \\ r_k
              \end{array}
              \right.
            \end{array} 
         } \right[
 & g_i^{*} &
 \left] {\scriptsize 
            \begin{array}{l}
             x  \\ x_1 \\ \cdots \\ x_n
             \\[1mm]
             0^m\\
            r_1\\ \cdots \\ r_{i-1} \\[1mm]
          g_i(x_1,\ldots,x_n) \\ 
          r_{i+1}  \\\cdots\\ r_k 
            \end{array} 
        } \right.
\end{array}
\hspace{1cm} %%%%%%%%%%%%%%%%%%%%%%%%%%%%%%%
\begin{array}{rcl}
 \left. {\scriptsize 
            \begin{array}{r}
             x \\ x_1 \\ \cdots \\ x_n
             \\[1mm]
              0^m\\[1mm]
           k\left\{
              \begin{array}{c}
             0 \\[1mm] \cdots \\[1mm] 0
              \end{array}
              \right.\!\!\!\!\!
            \end{array} 
         } \right[
 & h^* &
 \left] {\scriptsize 
            \begin{array}{l}
             x  +\prCom{h,g_1,\ldots,g_k}(x_1,\ldots,x_n)\\ x_1 \\ \cdots \\ x_n
             \\[1mm]
          0^m\\
          g_1(x_1,\ldots,x_n) \\[1mm] \cdots\\[1mm] g_k(x_1,\ldots,x_n)
            \end{array} 
        } \right.
\end{array}
\\\\
%%%%%%%%%%%%%%%%%%%%%%%%%%***************************************
\begin{array}{rcl}
 \left. {\scriptsize 
            \begin{array}{r}
             x \\ x_1 \\ \cdots \\ x_n
             \\[1mm]
              0^m\\[1mm]
           k\left\{
              \begin{array}{r}
             0 \\[1mm] \cdots \\[1mm] 0
              \end{array}
              \right.
            \end{array} 
         } \right[
 & \!\!\!G\!\!\! &
 \left] {\scriptsize 
            \begin{array}{l}
             x  \\ x_1 \\ \cdots \\ x_n
             \\[1mm]
             0^m\\
          g_1(x_1,\ldots,x_n) \\[1mm] \cdots\\[1mm] g_k(x_1,\ldots,x_n)
            \end{array} 
        } \right.
\end{array}
\hspace{-3mm}
\begin{array}{rcl}
 \left. {\scriptsize 
            \begin{array}{r}
              x \\ x_1 \\ \cdots \\ x_n
             \\[1mm]
             0^m\\
             0^k
            \end{array} 
         } \right[
 & \!\!\!H\!\!\! &
 \left] {\scriptsize 
            \begin{array}{l}
             x  +\prCom{h,g_1,\ldots,g_k}(x_1,\ldots,x_n)\\ x_1 \\ \cdots \\ x_n
             \\[1mm]
          0^m\\
          0^k
            \end{array} 
        } \right.
\end{array}
\end{array}$$
\hrule
\caption{Composition component relations}\label{compositionComponentRelations}
\end{table}

The proof is given by induction on the definition of the primitive recursive function $f$. 
%For sake of simplicity, we present some of its cases in an exemplified form.
\begin{itemize}[leftmargin=5mm]
\item  
If $f$ is $\prZero^n$ then let $\prrpr{\prZero^n}:= \rprId^{n+1}$, where $a=0$.
\item 
If $f$ is $\prSucc^n_i$ then let $\prrpr{\prSucc^n_i}:= \rprInc^{n+1}_{i+1;1} \rprSeq (\rprPCom{\rprSucc}{\rprId^n}) $, where $a=0$.
\item 
If $f$ is $ \prProj{n}{i} $ ($1\leq i\leq n$) then, let  $\prrpr{ \prProj{n}{i} }:= \rprInc^{n+1}_{i+1;1}$, where $a=0$.
\item 

Let $f=\prCom{h,g_1,\ldots,g_k}$ where $ g_1,\ldots,g_k\in\PRF^{n} $ and $ h\in\PRF^{k} $, for some $k\geq 1$.
 %Let $ g_1,\ldots,g_k\in\PRF^{n} $ and $ h\in\PRF^{k} $ in $ \prCom{h,g_1,\ldots,g_k}\in\PRF^{n}$.
Therefore, by inductive hypothesis,  there are $ \prrpr{g_1} \in \RPP^{n+a_1+1}$, $\ldots$, $\prrpr{g_k}\in\RPP^{n+a_k+1} $ and 
$ \prrpr{h}\in\RPP^{k+a_0+1} $ for $ a_0,\ldots,a_k\in\mathbb{N}$.

 Let $m=\max\{ a_0,\ldots,a_k\}$.
By using $\prrpr{g_i}$, it is easy to build $g_i^{*}\in \PRF^{n+m+k+1}$ computing 
the reversible permutation described in the top-left of Table \ref{compositionComponentRelations}.
The sequential composition can be used to compute  the permutation $G$ described in the bottom-left of Table \ref{compositionComponentRelations}.
By using $G$ and $h$, it is easy to compute $h^*$ in the top-right of Table \ref{compositionComponentRelations}.
Last, $H\in\RPP^{n+(m+k)+1}$ in the bottom-right of Table \ref{compositionComponentRelations} is 
 $\prrpr{\prCom{h,g_1,\ldots,g_k}}$ and we define it as $h^* \rprSeq \rprInv{G}$.

$H$ improves the representation of composition sketched  in \cite{PaoliniPiccoloRoversiICTCS2015} because
it reduces the number of ancillae by re-using them to compute one after the other $ g_1,\ldots,g_k\in\PRF^{n} $ and $ h\in\PRF^{k} $.

\item 
Let $f\in\PRF^{n}$ where $n\geq 1$ be defined by means of the primitive recursion on $ g\in\PRF^{n-1} $ and $ h\in\PRF^{n+1} $.
By inductive hypothesis there are $ \prrpr{g} \in \RPP^{(n-1)+a_g+1}$ and $ \prrpr{h}\in\RPP^{(n+1)+a_h+1} $ for 
$ a_g,a_h\in\mathbb{N}$.

The definition of $\prrpr{f}$ requires:
(i) a temporary argument to store the result of the previous recursive call;
(ii) a temporary argument  to stack  intermediate results;
(iii) a temporary argument  to index the current iteration step;
(iv) a temporary argument to contain the final result which is not modified when the computation is undone for cleaning temporary values; and
(v) in different times, as many temporary arguments as $a_g$ to compute $\prrpr{g}$, 
                        as many temporary arguments as $a_h$ to compute $\prrpr{h}$ and
                        $ 10 $ 
                        ancillae to $\rprPush$ and $\rprPop$ elements in the course of the computation
                        (see Proposition \ref{proposition:Representing stacks in RPP}.)


\begin{table}\hrule\centering
 $$\hspace{-3mm}
\begin{array}{c}
\arraycolsep=1.4pt
\begin{array}{rcl}
      \left. {\scriptsize 
          \begin{array}{r}
            x \\ x_1 \\ \cdots \\ x_n
            \\[1.2mm]
              0 \\ \cdots \\ 0\\[1mm]
              0^4
          \end{array}
        } \right[
      & \prrpr{f} &
      \left] {\scriptsize 
          \begin{array}{l}
            f(x_1,\ldots,x_n)\\[1pt] x_1 \\ \cdots \\ x_n
            \\[2pt]
             \!\!\left. 
              \begin{array}{l}
                0\\ \cdots\\ 0
              \end{array}\right\}\max\{a_g,a_h,10\}\\
             0^4
          \end{array} 
        } \right.
    \end{array}\hspace{8mm}
\begin{array}{rcl}
      \left. {\scriptsize 
          \begin{array}{r}
           0\\ r \\ x_1 \\ \cdots \\ x_{n-1}\\ i
            \\[1.2mm]
            0^{a-4}
            \\[1mm]
            \mathcal{S}
          \end{array}
        } \right[
      &  h_\text{step} &
      \left] {\scriptsize 
          \begin{array}{l}
           0\\ \prrpr{h}(r,x_1,\ldots,x_{n-1},i)
            \\ x_1 \\ \cdots \\ x_{n-1}\\ i+1
            \\[1mm]
            0^{a-4}
            \\[1mm]
             \langle r, \mathcal{S} \rangle
          \end{array} 
        } \right.
    \end{array}\\
%$\;$\\   
  \end{array}$$
\hrule
  \caption{Primitive Recursion Components}\label{fig:pimRecComponents}
\end{table}

Let $a:=4+max\{a_g,a_h, 10\}$. Our goal is to define $\prrpr{f}\in\RPP^{n+a+1}$ which 
behaves as described in the left of Table~\ref{fig:pimRecComponents}:
\begin{enumerate}[leftmargin=5mm]
%\setlength{\itemindent}{-.5in}
\item 
A first block of operations which we call $F_1$ reorganizes the inputs of $\prrpr{f}$ in Table~\ref{fig:pimRecComponents}
to obtain $0, 0, x_1,\ldots, x_{n-1}, \overbrace{0,\ldots,  0}^{a-2}, x_n,x$.

\item
The result of the previous step becomes the input of $\rprPCom{\rprId^1}{\rprPCom{\prrpr{g}}{\rprId^{a-a_g}}}$ 
which yields $0,g(x_1,\ldots, x_{n-1}), x_1,\ldots, x_{n-1}, \overbrace{0,\ldots, 0}^{a-2}, x_n,x$.  
We notice that $x_n$ is the argument that drives the iteration.

\item 
The previous point supplies the arguments to the $x_n$-times applications of the recursive step:
$$h_\text{step} := (\rprPCom{\prrpr{h}}{\rprId^{a-(a_h+2)}})\rprSeq \rprSucc^{n+a-1}_{n+2}
   \rprSeq \rprPush^{n+a-1}_{2;n+a-1} \rprSeq (\rprPCom{\rprBSwap}{\rprId^{n+a-3}})$$
by means of $\rprPCom{\rprIt{}{}{h_{\text{step}}}}{\rprId^1}$.
The relation that $h_\text{step}$ implements is depicted in Table~\ref{fig:pimRecComponents}.
Thus, $h_\text{step}$ takes $n+a-1$ input arguments only among the $ n+a+1 $ available
because the first one serves to the identity and the other one drives the iteration.
The function of $ h_\text{step} $ first applies $\prrpr{h}$ and then re-organizes arguments for the next iteration. Specifically, 
(i) it increments the step-index in the argument of position $(n+2)$,
(ii) it pushes the result of the previous step, which is in position $2$, 
on top of the stack , which is in its last ancilla,
(iii) it exchanges the first two arguments, the first one containing the result of the last 
iterative step and the second one containing a fresh zero produced by $\rprPush$.
\item 
We get $0, f(x_1,\ldots, x_{n}), x_1,\ldots, x_{n-1}, x_n, \overbrace{0,\ldots,  0}^{a-2}, x_n,x$
from the previous point. We add the result to the last line by means of 
$\rprInc_{2;n+a+1}$ by yielding 
$0, f(x_1,\ldots, x_{n}), x_1,\ldots, x_{n-1}, x_n, \overbrace{0,\ldots,  0}^{a-2}, x_n,x+f(x_1,\ldots, x_{n})$.

\item 
We then conclude by unwinding the first three steps.
\end{enumerate}
\end{itemize}
\par\noindent
Summing up, $\prrpr{h}$ is:
$$ F_1 \rprSeq \left(\rprPCom{\rprId^1}{\rprPCom{\prrpr{g}}{\rprId^{a-a_g}}}\right) \rprSeq \left(\rprPCom{\rprIt{}{}{h_{\text{step}}}}{\rprId^1}\right)
\rprSeq \rprInc_{2;n+a+1} \rprSeq \rprInv{\left(\rprPCom{\rprIt{}{}{h_{\text{step}}}}{\rprId^1}\right)} \rprSeq 
   \rprInv{\left(\rprPCom{\rprId^1}{\rprPCom{\prrpr{g}}{\rprId^{a-a_g}}}\right)}  \rprSeq \rprInv{F_1} 
\enspace . $$
\qed
\end{prf}

\subsection{$ \RPP $ is $ \PRF $-sound}
The mere intuition should support the evidence that every $ f\in\RPP $ has a representative inside $ \PRF $ we
can obtain via the bijection which exists between $ \Int $ and $ \Nat $. 
More precisely, every $ \RPP $-permutation can be represented  as a $ \PRF $-endofunction on tuples of natural 
numbers which encode integers. Details on how formalizing the embedding of $ \RPP $ into $ \PRF $ are in \cite{PaoliniPiccoloRoversiICTCS2015}.

%%%%%%%%%%%%%%%%%%%%%%%%% servono ad emacs
%%% Local Variables:
%%% mode: latex
%%% TeX-master: "main.tex"
%%% ispell-local-dictionary: "american"
%%% End: