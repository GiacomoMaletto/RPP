%\usepackage{marginnote}
%\usepackage[normalem]{ulem}
 %\drawLineToRightMargin
 %\marginpar{right}
\setlength{\marginparwidth}{3.8cm}
\usepackage{todonotes}
\presetkeys{todonotes}{size=\tiny,linecolor=blue,bordercolor=red,backgroundcolor=yellow!10}{}
\newcommand{\LR}[2]{\color{gray}\sout{#1}\color{blue}\  #2 \color{black}\marginpar{\qquad$\star$}}
\newcommand{\LRx}[2]{\color{gray}\ #1\ \color{blue}\  #2 \color{black}\marginpar{\qquad$\star$}}
\newcommand{\LRnm}[2]{\color{gray}\sout{#1}\color{blue}\  #2 \color{black}}
\newcommand{\LP}[2]{\color{gray}\sout{#1}\color{red}\  #2 \color{black}\marginpar{\qquad$\star$}}
\newcommand{\LPx}[2]{\color{gray}\ #1\ \color{red}\  #2 \color{black}\marginpar{\qquad$\star$}}
\newcommand{\LPnm}[2]{\color{gray}\sout{#1}\color{red}\  #2 \color{black}}

\newcommand{\myVec}[1]{\overrightarrow{{\!\!#1}}}
\newcommand{\evidenzia}[1]{\colorbox{cyan!10}{#1}}
%%%%%%%%%%%%%%%%%%%%%%%%%%%%%%%%%%%%%%%%%%%%%
%\usepackage{array}%\def\myColumn#1{\newcolumntype{L}{#1}}
\makeatletter
\renewenvironment{smallmatrix}[1][0em]{\null\,\vcenter\bgroup
  \Let@\restore@math@cr\default@tag
  \baselineskip6\ex@ \lineskip1.5\ex@ \lineskiplimit\lineskip
  \ialign\bgroup\hfil$\m@th\scriptstyle##$\hfil&&\kern#1\hfil
  $\m@th\scriptstyle##$\hfil\crcr
}{%
  \crcr\egroup\egroup\,%
}
\makeatother
\newcommand{\production}[1]{#1 &, &}
\newcommand{\commut}[2]{\production{#1}{#2}}
\makeatletter
\def\qcommut#1{\xcommut#1,\relax,}
\def\xcommut#1,{\xxcommut{#1}}
\def\xxcommut#1#2,{%
\ifx\relax#2%
#1%
\expandafter\@gobbletwo
\fi
\xxcommut{\commut{#1}{#2}}}
\makeatother
% \def\myColumn{clclclclcl}
% \newcommand{\myControl}[1]%
% {   \foreach \x in {1,2,...,#1}
%     {   \renewcommand{\myColumn}{cl\myColumn}
%     }
% }
\newcommand{\AUTOrprSwap}[3]{
%\myControl{#1}
\!\left\rmoustache\!\!%\arraycolsep=0pt
\scriptsize
\begin{smallmatrix}[0pt]%\expandafter\array\expandafter{\myColumn}%\begin{array}{#4}
\qcommut{#2}\\
\qcommut{#3}
\end{smallmatrix}%\endarray%\end{array}
\!\!\right\rmoustache^{\!\!#1}}
%%%%%%%%%%%%%%%%%%%%%%%%%%%%%%%%%%%%%%%%%%%%%
%\usepackage{etoolbox}
%\patchcmd{\subequations}{\def\theequation{\theparentequation\alph{equation}}}%
%{\def\theequation{\theparentequation.\arabic{equation}}}{}{}
\def\fcmp{\mathbin{\raise 0.6ex\hbox{\oalign{\hfil$\scriptscriptstyle \mathrm{o}$\hfil\cr\hfil$\scriptscriptstyle\mathrm{9}$\hfil}}}}



\newcommand{\loopComment}[1]{\parbox{13.8cm}{\scriptsize \slash*#1*\slash}}

\newcommand{\yBox}[1]{
\medskip\framebox{\colorbox{yellow!19}{\begin{minipage}{.89\linewidth}
#1 \end{minipage}}}\medskip
}

\newcommand{\la}{\langle}
\newcommand{\ra}{\rangle}
\renewcommand{\vec}[1]{\ensuremath{\underline{#1}}}
%\newcommand*\rectangled[1]{\tikz[baseline=(char.base)]{ \node[shape=rectangle,draw,inner sep=2pt, rounded corners=4pt, thick] (char) {\sffamily#1};}}
%\newcommand{\prrpr}[1]{\overline{#1}}
%\newcommand{\prrpr}[1]{\overbracket[.1pt][1pt]{#1}^{\scalebox{.21}{PRF}}}
%\newcommand{\prrpr}[1]{\overbracket[.1pt][1pt]{#1}^{\tikz[baseline=0em]{ \node[scale=.21]{PRF}; }  }}
%\newcommand*{\prrpr}[1]{\mathop{\!\tikz[baseline=(charluca.base)]{ \node[label={[scale=.3,label distance=-3.9mm]north:PRF}] (charluca) {$\overbracket[.1pt][1pt]{#1}$};}\!} }

\newcommand{\prrpr}[1]{\overbracket[.1pt][1pt]{\underbracket[.1pt][1pt]{#1}}}

\newcommand{\rprpr}[1]{\underline{#1}}
\newcommand{\rprm}[1]{\llparenthesis #1\rrparenthesis}
\newcommand{\eqdef}{\overset{{\tiny def}}{=}}

\newcommand{\prZero}{\mathtt{0}}
\newcommand{\prSucc}{\mathtt{S}}
\newcommand{\prPred}{\mathtt{P}}
\newcommand{\prProj}[2]{\mathtt{P}^{#1}_{#2}} % P^n_i
\newcommand{\prCom}[1]{\circ{[#1]}} % composition
\newcommand{\prRec}[2]{\mathtt{R}_{#1,#2}} % recursion

\newcommand{\rprPerm}[2]{
\left(
\!\!\!\begin{array}{c}
#1\\
#2
\end{array}\!\!\!\right)}
\newcommand{\arityI}{arity\xspace}
\newcommand{\arityO}{co-arity\xspace}

%%%%%%%%%%%%%% Terms
\newcommand{\temporarychannel} {\textit{temporary argument}\xspace}
\newcommand{\temporarychannels}{\textit{temporary arguments}\xspace}
\newcommand{\rprWea}{\mathsf{w}}
\newcommand{\rprWeaName}{\textit{weakening}\xspace}
\newcommand{\rprNeg}{\mathsf{N}}
\newcommand{\rprNegName}{\textit{sign-change}\xspace}
\newcommand{\rprId}{\mathsf{I}}
\newcommand{\rprIdz}{\rprId}
\newcommand{\rprIdName}{\textit{identity}\xspace}
\newcommand{\rprZero}{\mathsf{0}}
\newcommand{\rprZeroName}{\textit{rZero}\xspace}
\newcommand{\rprSucc}{\mathsf{S}}
\newcommand{\rprSuccName}{\textit{successor}\xspace}
\newcommand{\rprPred}{\mathsf{P}}
\newcommand{\rprPredName}{\textit{predecessor}\xspace}
\newcommand{\rprSCom}[2]{#1\fatsemi #2} % serial composition
\newcommand{\rprSComName}{\textit{series composition}\xspace}
\newcommand{\rprSeq}{\fatsemi}
\newcommand{\rprPCom}[2]{ #1 \rVert #2} % parallel composition
\newcommand{\rprPComSX}[2]{\left. #1 \right\rVert #2}
\newcommand{\rprPComName}{\textit{parallel composition}\xspace}
%%\newcommand{\rprIf}[5]{\mathsf{If}^{#1}_{#2}\mathsf{[}#3,#4,#5\mathsf{]}} % original
\newcommand{\rprIf}[5]{\mathsf{If}^{#1}_{#2}\left[#3,#4,#5\right]} 
\newcommand{\rprIfName}{\textit{selection}\xspace}
%\newcommand{\rprIt}[3]{\mathsf{It}^{#1}_{#2}\mathsf{[}#3\mathsf{]}} %% original
\newcommand{\rprIt}[3]{{\mathsf{It}^{#1}_{#2}\left[ #3 \right]}}
\newcommand{\rprItName}{\textit{finite iteration}\xspace}

\newcommand{\rprSrc}[1]{\mathsf{Src[#1]}} % source constant
\newcommand{\rprSrcName}{\textit{source constant}\xspace}
\newcommand{\rprSrcNamePlur}{\textit{source constants}\xspace}
\newcommand{\rprSnk}[1]{\mathsf{Snk[#1]}} % sink constant
\newcommand{\rprSnkName}{\textit{sink constant}\xspace}
\newcommand{\rprSnkNamePlur}{\textit{sink constants}\xspace}
\newcommand{\rprGSwap}[1]{\rprBSwap^{#1}} % swap^k generic
%\newcommand{\rprSwap}[3]{\rprBSwap^{#1}_{#2\leftrightarrow #3}} % swap^k_i,j
\newcommand{\rprSwap}[3]{\left\rmoustache\!\!
\arraycolsep=1.4pt
\scriptsize
\begin{array}{l}
#2 \\
#3
\end{array}\!\!\right\rmoustache^{\!\!#1} } % swap^k_i,j
\newcommand{\rprSwapName}{\textit{binary rewiring permutation}\xspace}
\newcommand{\rprBSwap}{\chi} % swap^2 1 and 2
\newcommand{\rprBSwapName}{\textit{binary rewiring permutation}\xspace}

\newcommand{\rprMin}[1]{\mu #1} % minimalization
\newcommand{\rprMinName}{\textit{minimalization scheme}\xspace}
\newcommand{\rprSumSet}{\mathcal{T}_{\rprSum}}
\newcommand{\rprAbs}{\mathsf{A}}
\newcommand{\rprInc}{\mathsf{inc}}

\newcommand{\rprPush}{\mathsf{push}}
\newcommand{\rprPop}{\mathsf{pop}}
\newcommand{\rprSum}{\mathsf{sum}}
\newcommand{\rprSub}{\mathsf{sub}}
\newcommand{\rprDup}{\nabla}
\newcommand{\rprCopy}{\mathsf{copy}}
\newcommand{\rprDec}{\mathsf{dec}}
\newcommand{\rprMult}{\mathsf{mult}}
\newcommand{\rprDivTwo}{\mathsf{div2}}
\newcommand{\rprBitPN}{\mathsf{bitpn}} % bit positive/negative
\newcommand{\rprLess}{\mathsf{less}} % 
\newcommand{\rprSq}{\mathsf{sq}}
\newcommand{\rprSqRoot}{\mathsf{sqroot}}
\newcommand{\rprBMu}[5]{ %
% #1 indice della variabile che contiene il minimo cercato
% #2 indice della variabile che contiene il bound
% #3 funzione in RPRF che dipende (almeno) da #1 e scrive il risultato in #4
% #4 indice della variabile col risultato di #4
\mathsf{min}_{#1;#2} ({#3}_{#4;#5})}
\newcommand{\lpMU}[1]{ %
% #1 indice della variabile che contiene il minimo cercato
% #2 indice della variabile che contiene il bound
\lpMUNoArgs ({#1})}
\newcommand{\lpMUNoArgs}{\mathsf{min}}



\newcommand{\rprrs}[1]{\mathsf{rs}_{#1}} % recursive step
\newcommand{\rprrb}[1]{\mathsf{rb}_{#1}} % recursive base
\newcommand{\rprStep}[2]{\mathsf{S}_{#1,#2}} % recursive step of translation
\newcommand{\rprBase}[1]{\mathsf{B}_{#1}} % base step of translation
\newcommand{\rprRec}[3]{\mathsf{Rec[}#1,#2,#3\mathsf{]}} % recursive scheme
\newcommand{\rprREC}[4]{{\mathsf{Rec}^{#1}[}#2,#3,#4]} % recursive scheme

\newcommand{\rprInv}[1]{{#1}^{-1}} % inverse
\newcommand{\rprInvName}{\textit{inverse}\xspace}
\newcommand{\rprrvr}{\Downarrow} % rewriting relation
\newcommand{\rprEq}{\simeq} % equivalence relation on RPRT
\newcommand{\rprQuasiEq}{\equiv} % quasi equivalence relation on RPRT
\newcommand{\rprMea}{\mathfrak{m}} % measure function of a vector
\newcommand{\rprMeaName}{\textit{measure function}\xspace} % name of the measure funciton of a vector
\newcommand{\rprSquare}[1]{\mathsf{sq}^{#1}}
\newcommand{\revCompile}{\circledR}%{\mathsf{RV}}
\newcommand{\rprNabla}{\nabla} 
\newcommand{\rprDelta}{\Delta}

%%%%%%%%%%%%%%%%%%%%%%
\makeatletter
\newcommand{\dotminus}{\mathbin{\text{\@dotminus}}}
\newcommand{\@dotminus}{%
  \ooalign{\hidewidth\raise1ex\hbox{.}\hidewidth\cr$\m@th-$\cr}%
}
\makeatother
%%%%%%%%%%%%%%%%%%%%%%
\newcommand{\etc}{etc.\xspace}
\newcommand{\ie}{i.e.\xspace}
\newcommand{\Ie}{I.e.\xspace}
\newcommand{\KR}{\mathsf{KR}}
\newcommand{\GR}{\mathsf{GR}}
%\newcommand{\PR}{\mathsf{PR}}
\newcommand{\Acker}{\operatorname{\mathsf{ Ack}}}
\newcommand{\AckerInv}{\mathsf{Ack}^{-1}}
\newcommand{\RF}{\mathsf{RF}} % recursive functions
\newcommand{\PRF}{\mathsf{P}\RF} % primitive recursive functions
\newcommand{\RPRF}{\mathsf{R}\PRF} % reversible primitive recursive functions
\newcommand{\RPP}{\mathsf{RPP}} % reversible primitive permutations
\newcommand{\ESRL}{\mathsf{ESRL}} % extended simple reversible language
\newcommand{\JMF}{\mathcal{RI}} % Jacopini Mentrasti
\newcommand{\JPRF}{\mathsf{J}\PRF}
\newcommand{\BJPRF}{\mathsf{B}\PRF}
\newcommand{\rprF}{\mathsf{F}}

\newcommand{\ZtoN}{\Psi}% morphism
\newcommand{\NtoZ}{\Phi}% morphism
\newcommand{\CP}{\operatorname{Cp}}% morphism
\newcommand{\CU}{\operatorname{Cu}}% morphism
\newcommand{\TN}{\operatorname{Tn}}% 

\newcommand{\Set}[1]{\{#1\}}
\newcommand{\Nat}{\mathbb{N}}
\newcommand{\Int}{\mathbb{Z}}
\newcommand{\cat}{\cdot}

\newcommand\tikzmark[1]{\tikz[overlay,remember picture,baseline] \coordinate (#1);}

\def\vspnodeEmpty(#1)#2#3{% vertex series parallel graph node one input
  \begin{scope}[shift={(#1)}]
    \draw (0,0) ;
    \node[circle, fill=none, minimum size=20pt,inner sep=0pt] at (.5,.7) {#3};
  \end{scope}
}

\tikzstyle{vertex}=[circle,fill=black!25,minimum size=20pt,inner sep=0pt]
%%%%%%%%%%%%%%%%%%%%%
\def\vspnodeOneInOneOut(#1)#2#3{% vertex series parallel graph node one input
  \begin{scope}[shift={(#1)}]
    \draw (0,0) ;
    \node[vertex] at (.5,.5) {#3};
%%%IN
    \draw (0,0.5) coordinate (#2 In1);
%%%OUT
    \draw (1,0.5) coordinate (#2 Out1);
  \end{scope}
}

%%%%%%%%%%%%%%%%%%%%%
\def\fPR(#1)#2#3{%
  \begin{scope}[shift={(#1)}]
    \draw (0,0) rectangle (1,1);
    \node at (.5,.5) {#3};
%%%IN
    \draw ( 0,0.8) node[right] {} -- +(-.5,0) coordinate (#2 X1);
    \draw (-.5,0.6) node[right] {$ \vdots $};
    \draw ( 0,0.2) node[right] {} -- +(-.5,0) coordinate (#2 Xk);
%%%OUT
    \draw (1,0.5) -- +(.5,0) coordinate (#2 Out);
  \end{scope}
}

%%%%%%%%%%%%%%%%%%%%%
\def\fRPR(#1)#2#3{%
  \begin{scope}[shift={(#1)}]
    \draw (0,0) rectangle (2,2.2);
%    \draw (0.5,1) -- (0.5,0);
%    \draw (0.5,0.5) -- (1,0.5);
    \node at (1,1) {#3};
%    \node at (0.75,0.25) {$\bar{Q}$};
%%%IN
    \draw ( 0,2.0) node[right] {} -- +(-.5,0) coordinate (#2 X1);
    \draw (-.5,1.8) node[right] {$ \vdots $};
    \draw ( 0,1.4) node[right] {} -- +(-.5,0) coordinate (#2 Xk);
    \draw ( 0,1.1) node[right] {} -- +(-.5,0) coordinate (#2 Zero);
    \draw ( 0,0.8) node[right] {} -- +(-.5,0) coordinate (#2 Zero1);
    \draw (-.5,0.6) node[right] {$ \vdots $};
    \draw ( 0,0.2) node[right] {} -- +(-.5,0) coordinate (#2 Zerom);
%%%OUT
    \draw (2,2.0) -- +(.5,0) coordinate (#2 Out);
    \draw (2,1.6) node[right] {} -- +(.5,0) coordinate (#2 OutX1);
    \draw (2,1.45) node[right] {$ \vdots $};
    \draw (2,1.1) node[right] {} -- +(.5,0) coordinate (#2 OutXk);
    \draw (2,0.8) node[right] {} -- +(.5,0) coordinate (#2 OutZero1);
    \draw (2,0.6) node[right] {$ \vdots $};
    \draw (2,0.2) node[right] {} -- +(.5,0) coordinate (#2 OutZerom);
  \end{scope}
}


\newtheorem{proposition}{Proposition}
\newtheorem{theorem}{Theorem}
\newtheorem{lemma}[theorem]{Lemma}
\newtheorem{corollary}[theorem]{Corollary}

\newproof{prf}{Proof} % <---- !!

\newdefinition{fact}{Fact}
\newdefinition{definition}{Definition}
\newdefinition{remark}{Remark}
\newdefinition{example}{Example}
\newdefinition{notation}{Notation}


%%%%%%%%%%%%%%%%%%%%%%%%%%%%%%
%%%%%%%%%%%%%%%%%%%%%%%%% servono ad emacs NON CANCELLARE x piacere
%%% Local Variables:
%%% mode: latex
%%% TeX-master: "main.tex"
%%% ispell-local-dictionary: "american"
%%% End: