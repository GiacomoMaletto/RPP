\section{Cantor pairing functions}
\label{section:Cantor pairing}
Pairing functions provide a mechanism to uniquely encode two natural numbers into a 
single natural number \cite{rosenberg2009book}. 
We show how to represent Cantor pairing functions as functions of $\RPP$ 
restricted on natural numbers,
albeit it is possible to re-formulate the pairing function on integers (see \cite{PaoliniPiccoloRoversiICTCS2015}.)

\begin{definition}\label{grishpan}
Cantor pairing is a pair of isomorphisms $ \CP: \Nat^2 \longrightarrow \Nat $ and 
$ \CU: \Nat \longrightarrow \Nat^2 $ which embed $ \Nat^2 $ into $ \Nat$:
\begin{align}
\label{align: CP definition}
\CP(x,y) = \, & x + \sum_{i=0}^{x+y} i\\
\nonumber
\CU(z) =\, &  \left\langle z - \sum_{i=0}^{k - 1} i\;\pmb,\;(k - 1) 
                          - (z - \sum_{i=0}^{k - 1} i)
                          \right\rangle 
         % \text{ such that }
         %  \left\{ \begin{array}{l}
         %    x = z - (\sum_{i=0}^{k - 1} i),\\
         % y = (k - 1) - x
         %  \end{array} \right.
\ \  \begin{minipage}{.4\linewidth}
 where $k$  is the least value s.t. 
$k \leq z$ and  $z < \sum_{i=0}^{k} i$.
  \end{minipage}
\end{align}
%\todo{\texttt{http://www.math.drexel.edu/~tolya/cantorpairing.pdf}}
\end{definition}

The pairing functions in the above Equation~\eqref{align: CP definition} rely on the notion of triangular number $T_n= \sum_{i=0}^{n} i$, for any $ n\in\Nat $.

%%%%%%%%%%%%%%%%%%%%%%%%%%%%%%%%%%%%%%
\begin{lemma}
\label{lemma:triangular number}
For every $ x_1, x_2 $ and $ x_3 \in\mathbb{N}$, the two following functions exist:
$$\arraycolsep=1.4pt
\begin{array}{rcl}
\left. {\scriptsize \begin{array}{r} x_1\\ x_2 \end{array}} \right[
& \operatorname{T2} &
\left] {\scriptsize \begin{array}{l} x_1 + 1\\ x_2+x_1+1 \end{array} } \right.
\end{array}\qquad
\arraycolsep=1.4pt
\begin{array}{rcl}
\left. {\scriptsize \begin{array}{r} x_1\\ x_2\\ x_3 \end{array}} \right[
& \operatorname{T3} &
\left] {\scriptsize \begin{array}{l} x_1\\ x_2 + x_1\\ x_3 + x_2 x_1 + \sum^{x_1}_{i=0} i \end{array} } \right. 
\enspace .
\end{array}$$
and belong to $ \RPP^2 $ and $ \RPP^3 $, respectively.
\end{lemma}
\begin{proof}
Let $\operatorname{T2} $ be $\rprSCom{\rprSucc^2_1} {\rprInc^2_{1;2}}$
and let $\operatorname{T3}\in\RPP^3 $ be $\rprIt{3}{1,\langle 2,3\rangle}{\operatorname{T2}}$.
The result is immediate because we are interested in the behaviour of the operator only on positive numbers,
viz.  $ x_1, x_2 , x_3 \in\mathbb{N}$.
\end{proof}

We shall systematically neglect to specify the relational behavior of
the coming definitions of functions on negative input values.
 
%%%%%%%%%%%%%%%%%%%%%%%%%%%%%%%%%%%%%%%++++++++++++++++++++++++
\begin{theorem}%[$ \CP: \Nat^2 \longrightarrow \Nat $ and Triangular numbers in $ \RPP $]
\label{theorem:CP in RPP}
For every $x\in\mathbb{N}$, the following function is in $\RPP^4$:
\begin{align*}
%\arraycolsep=1.4pt
%\begin{array}{rcl}
% \left. {\scriptsize \begin{array}{r} x\\ 0\\[0.75mm] 0 \end{array}} \right[
% & \operatorname{Tn} &
% \left] {\scriptsize \begin{array}{l}
%                       x\\ 0\\[0.75mm] (\sum^{x}_{i=0} i) \\
%                     \end{array} } \right.
%\end{array}
%&\textrm{ and }&
\arraycolsep=1.4pt
\begin{array}{rcl}
 \left. {\scriptsize \begin{array}{r} x\\ y\\[1mm] 0\\[1mm] 0  \end{array}} \right[
 & \CP &
 \left] {\scriptsize \begin{array}{l}
                       x\\ y\\[1mm]  (\sum^{x+y}_{i=0} i ) + x\\[0.75mm] 0
                     \end{array} } \right.
\end{array}
\enspace .
\end{align*}
\end{theorem}
\begin{prf}
Let $\CP$ be 
$\rprInc^{4}_{1;2} \rprSeq\rprInc^{4}_{1;4} \rprSeq \rprSwap{4}{1,2,3,4}{2,3,1,4} 
  \rprSeq (\rprPCom{\operatorname{T3}}{\rprId}) \rprSeq \rprDec^4_{1;2} \rprSeq 
  \rprSwap{4}{1,2,3,4}{4,1,3,2} \rprSeq \rprDec^4_{1;2}$.
\qed
\end{prf}

We now turn to representing $ \CU $. The computation of the difference between a given $ z $ and the triangular number
	$ \sum_{i=0}^{k} i $, limited by $ z $, is at the core of $ \CU $.

%%%% %% % \href{http://www.math.drexel.edu/~tolya/cantorpairing.pdf}{http://www.math.drexel.edu/~tolya/cantorpairing.pdf} 

\begin{lemma}[Subtracting a triangular number]
\label{lemma:The kernel kCU  to define CU}
    For every $ x_3 $ and $ x_4 \in\mathbb{N}$, the following function exists:
    \begin{align*}
    \arraycolsep=1.4pt
    \begin{array}{rcl}
    \left. {\scriptsize 
    	\begin{array}{r} 
    	0\\ 0\\ x_3\\ x_4\\ 
    	\end{array}} 
    \right[
    & H_{3;4} &
    \left] {\scriptsize 
    	\begin{array}{l}
    	0\\ 0\\ x_3\\ x_4 - \sum_{i=0}^{x_3} i
    	\end{array} } \right.
    \end{array}
    \end{align*}
	and belongs to $ \RPP^4 $.
\end{lemma}
\begin{prf}
Let $ H_{3;4} $ be:
	\begin{align*}
	& \left (\rprSwap{4}{1,2,3,4}{3,2,1,4} 
	\rprSeq (\rprPCom{\operatorname{T3}}{\rprId})\right )
	\rprSeq \rprDec^{4}_{3;4}
	\rprSeq \rprInv{\left (\rprSwap{4}{1,2,3,4}{3,2,1,4}
		    \rprSeq (\rprPCom{\operatorname{T3}}{\rprId})\right )}
	\enspace .
	\end{align*}
By Lemma~\ref{lemma:triangular number} the proof is done.\qed
\end{prf}

We can supply $ H_{3;4} $ to the minimization $ \lpMUNoArgs $ in order for it to yield a series of ``attempts'' 
whose goal is to find when $ H_{3;4} $ becomes negative in its 4th output:
\begin{center}\vspace{-5mm}
\resizebox{\textwidth}{!}{
\begin{tabular}{ccccc}
1-st attempt & 2-nd attempt &  & $ j $-th attempt & \\
$\arraycolsep=1.4pt
\begin{array}{rcl}
\left. {\scriptsize 
	\begin{array}{r} 
	0\\ 0\\ 0\\ z\\ 
	\end{array}} 
\right[
& H_{3;4} &
\left] {\scriptsize 
	\begin{array}{l}
	0\\ 0\\ 0\\ z - \sum_{i=0}^{0} i
	\end{array} } \right.
\end{array}$
&
$\arraycolsep=1.4pt
\begin{array}{rcl}
\left. {\scriptsize 
	\begin{array}{r} 
	0\\ 0\\ 1\\ z\\ 
	\end{array}} 
\right[
& H_{3;4} &
\left] {\scriptsize 
	\begin{array}{l}
	0\\ 0\\ 1\\ z - \sum_{i=0}^{1} i
	\end{array} } \right.
\end{array}$
&
$ \cdots $
&
$\arraycolsep=1.4pt
\begin{array}{rcl}
\left. {\scriptsize 
	\begin{array}{r} 
	0\\ 0\\ j\\ z\\ 
	\end{array}} 
\right[
& H_{3;4} &
\left] {\scriptsize 
	\begin{array}{l}
	0\\ 0\\ j\\ z - \sum_{i=0}^{j} i
	\end{array} } \right.
\end{array}$
&
$ \cdots $
\end{tabular}
} % resizebox
\end{center}

\noindent
The representation of $ \CU $ relies on the search here above.

%%%%%%%%%%%%%%%%%%%%%%%%%%%%%%%%%%%%++++++++++++++++++++++++++++++++++++++++++++++++++
\begin{theorem}[Representing $ \CU: \Nat^2 \longrightarrow \Nat $ in $ \RPP $]
\label{lemma:Representing CU in RPP}
A function $ \CU\in\RPP^{8}$ exists such that, for every $ z \in\mathbb{N}$:
\begin{align*}
\arraycolsep=1.4pt
\begin{array}{rcl}
 \left. {\scriptsize \begin{array}{r} 
                       z\\[0.5mm] 0\\[0.5mm] 0\\[0.5mm] 
                        0^{5
                       	}
                     \end{array}} \right[
 & \CU &
 \left] {\scriptsize \begin{array}{l}
                       z\\ z-(\sum^{v-1}_{i=0} i ) \\ (v-1)-(z-(\sum^{v-1}_{i=0} i)) \\[0.5mm]
                       0^{5
                         }
                     \enspace ,
                     \end{array} } \right.
\end{array}
\end{align*}
where $ v $ is the least value such that $ v \leq z $ and $ z-(\sum^{v}_{i=0} i ) < 0$. 
So $ z-(\sum^{v-1}_{i=0} i ) $ is the first component of the pair that $ z $ represents under Cantor pairing
and $ (v-1)-(z-(\sum^{v-1}_{i=0} i)) $ is the second one.
\end{theorem}
\begin{prf}
Let $ \CU $  be:
$$
\begin{array}{l}
\rprSwap{8}
{1, 2, 3, 4 ,5 ,6 ,7 ,8}
{4, 2, 3, 1 ,5 ,6 ,7 ,8}
\,\rprSeq \, \rprInc^{8}_{4;8}  \,\rprSeq \lpMU{H_{3;4}} \\
\qquad \qquad 
\,\rprSeq \rprPred^{8}_{5}\,\rprSeq
\rprSwap{8}
{1, 2, 3, 4 ,5 ,6 ,7 ,8}
{1, 2, 5, 4 ,3 ,6 ,7 ,8}
\rprSeq \,( H_{3;4} \rVert \rprId^4) \,\rprSeq \, \rprDec^{8}_{4;3}
\,\rprSeq
\rprSwap{8}
{1, 2, 3, 4 ,5 ,6 ,7 ,8}
{8, 4, 3, 2 ,5 ,6 ,7 ,1}
\end{array}
$$
The \rprSComName $ \rprSwap{8}
{1, 2, 3, 4 ,5 ,6 ,7 ,8}
{4, 2, 3, 1 ,5 ,6 ,7 ,8}
\,\rprSeq \, \rprInc^{8}_{4;8} $ sets the input for $\lpMU{H_{3;4}}$ which receives the tuple
$\langle 0,0,0,z,0,0,0,z \rangle$ and yields $\langle 0,0,0,z,k,0,0,z \rangle$
where $k$ is the least value such that $k \leq z$ and  $z < \sum_{i=0}^{k} i$.
The predecessor $ \rprPred^{8}_{5}$ sets its 5th argument to the value $k-1$ which is required
to correctly apply $ \CU $ (see Definition \ref{grishpan}).
The next rewiring sets the tuple $\langle 0,0,k-1,z,0,0,0,z \rangle$, argument of $ H_{3;4} \rVert \rprId^4 $
which produces $\langle 0,0,k-1,z-\sum_{i=0}^{k - 1} i,0,0,0,z \rangle$.
The value $z-\sum_{i=0}^{k - 1} i$ is the first component of the pair of numbers we want to extract from 
the first argument of the whole $ \CU $.
The second component is $k-1-(z-\sum_{i=0}^{k - 1} i)$ we obtain
by applying $\rprDec^{8}_{4;3}$. The last rewiring rearranges the values as required.
\qed
\end{prf}

As a remark,
Theorem~\ref{theorem:CP in RPP},
Lemma~\ref{lemma:The kernel kCU  to define CU} and
Theorem~\ref{lemma:Representing CU in RPP} can be extended to functions that operate on 
$ \Int $, not only on $ \Nat $, suitably managing signs \cite{PaoliniPiccoloRoversiICTCS2015}.

%%%%%%%%%%%%%%%%%%%%%%%%%
%%% Local Variables:
%%% mode: latex
%%% TeX-master: "main.tex"
%%% ispell-local-dictionary: "american"
%%% End: