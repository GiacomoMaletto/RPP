% Template article for preprint document class `elsarticle'
% SP 2006/04/26

\documentclass{elsarticle}
%\documentclass[doublespacing]{elsart1p}

% Use the option doublespacing or reviewcopy to obtain double line spacing
% \documentclass[doublespacing]{elsarticle}

% if you use PostScript figures in your article
% use the graphics package for simple commands
% \usepackage{graphics}
% or use the graphicx package for more complicated commands
% \usepackage{graphicx}
% or use the epsfig package if you prefer to use the old commands
% \usepackage{epsfig}

% The amssymb package provides various useful mathematical symbols
%\usepackage{amssymb}
\usepackage{etex}
%\usepackage{geometry}
\usepackage[normalem]{ulem}
\usepackage{array}
\usepackage[T1]{fontenc} % correct output rendering as first packages
\usepackage{ae,aecompl}
\usepackage[latin1]{inputenc} % correct input typing as second package
%% gives \llparenthesis, \llceil, \llbracket, and right correspondent
\usepackage{stmaryrd}
\usepackage{xspace}
\usepackage{amsmath}
\usepackage{amsthm}
\usepackage{amsfonts}
\usepackage{amssymb}
%\usepackage{eufrak}
\usepackage{adjustbox}
\usepackage{graphicx}
\usepackage{alltt}
\usepackage{varwidth}
\usepackage[lutzsyntax,pdftex]{virginialake}\vlsmallbrackets\aftrianglefalse
\usepackage{todonotes}
%\usepackage{bussproofs}
\usepackage{longtable}
\usepackage{mathtools}
\usepackage{enumitem}
\usepackage{hyperref}
\usepackage{xcolor}
\hypersetup{
    colorlinks,
    linkcolor={red!50!black},
    citecolor={blue!50!black},
    urlcolor={blue!80!black}
}

\presetkeys{todonotes}{size=\scriptsize,linecolor=blue,bordercolor=red,backgroundcolor=green!30}{}
%\usepackage{marginnote}
%\usepackage[normalem]{ulem}
 %\drawLineToRightMargin
 %\marginpar{right}
\setlength{\marginparwidth}{3.8cm}
\usepackage{todonotes}
\presetkeys{todonotes}{size=\tiny,linecolor=blue,bordercolor=red,backgroundcolor=yellow!10}{}
\newcommand{\LR}[2]{\color{gray}\sout{#1}\color{blue}\  #2 \color{black}\marginpar{\qquad$\star$}}
\newcommand{\LRx}[2]{\color{gray}\ #1\ \color{blue}\  #2 \color{black}\marginpar{\qquad$\star$}}
\newcommand{\LRnm}[2]{\color{gray}\sout{#1}\color{blue}\  #2 \color{black}}
\newcommand{\LP}[2]{\color{gray}\sout{#1}\color{red}\  #2 \color{black}\marginpar{\qquad$\star$}}
\newcommand{\LPx}[2]{\color{gray}\ #1\ \color{red}\  #2 \color{black}\marginpar{\qquad$\star$}}
\newcommand{\LPnm}[2]{\color{gray}\sout{#1}\color{red}\  #2 \color{black}}

\newcommand{\myVec}[1]{\overrightarrow{{\!\!#1}}}
\newcommand{\evidenzia}[1]{\colorbox{cyan!10}{#1}}
%%%%%%%%%%%%%%%%%%%%%%%%%%%%%%%%%%%%%%%%%%%%%
%\usepackage{array}%\def\myColumn#1{\newcolumntype{L}{#1}}
\makeatletter
\renewenvironment{smallmatrix}[1][0em]{\null\,\vcenter\bgroup
  \Let@\restore@math@cr\default@tag
  \baselineskip6\ex@ \lineskip1.5\ex@ \lineskiplimit\lineskip
  \ialign\bgroup\hfil$\m@th\scriptstyle##$\hfil&&\kern#1\hfil
  $\m@th\scriptstyle##$\hfil\crcr
}{%
  \crcr\egroup\egroup\,%
}
\makeatother
\newcommand{\production}[1]{#1 &, &}
\newcommand{\commut}[2]{\production{#1}{#2}}
\makeatletter
\def\qcommut#1{\xcommut#1,\relax,}
\def\xcommut#1,{\xxcommut{#1}}
\def\xxcommut#1#2,{%
\ifx\relax#2%
#1%
\expandafter\@gobbletwo
\fi
\xxcommut{\commut{#1}{#2}}}
\makeatother
% \def\myColumn{clclclclcl}
% \newcommand{\myControl}[1]%
% {   \foreach \x in {1,2,...,#1}
%     {   \renewcommand{\myColumn}{cl\myColumn}
%     }
% }
\newcommand{\AUTOrprSwap}[3]{
%\myControl{#1}
\!\left\rmoustache\!\!%\arraycolsep=0pt
\scriptsize
\begin{smallmatrix}[0pt]%\expandafter\array\expandafter{\myColumn}%\begin{array}{#4}
\qcommut{#2}\\
\qcommut{#3}
\end{smallmatrix}%\endarray%\end{array}
\!\!\right\rmoustache^{\!\!#1}}
%%%%%%%%%%%%%%%%%%%%%%%%%%%%%%%%%%%%%%%%%%%%%
%\usepackage{etoolbox}
%\patchcmd{\subequations}{\def\theequation{\theparentequation\alph{equation}}}%
%{\def\theequation{\theparentequation.\arabic{equation}}}{}{}
\def\fcmp{\mathbin{\raise 0.6ex\hbox{\oalign{\hfil$\scriptscriptstyle \mathrm{o}$\hfil\cr\hfil$\scriptscriptstyle\mathrm{9}$\hfil}}}}



\newcommand{\loopComment}[1]{\parbox{13.8cm}{\scriptsize \slash*#1*\slash}}

\newcommand{\yBox}[1]{
\medskip\framebox{\colorbox{yellow!19}{\begin{minipage}{.89\linewidth}
#1 \end{minipage}}}\medskip
}

\newcommand{\la}{\langle}
\newcommand{\ra}{\rangle}
\renewcommand{\vec}[1]{\ensuremath{\underline{#1}}}
%\newcommand*\rectangled[1]{\tikz[baseline=(char.base)]{ \node[shape=rectangle,draw,inner sep=2pt, rounded corners=4pt, thick] (char) {\sffamily#1};}}
%\newcommand{\prrpr}[1]{\overline{#1}}
%\newcommand{\prrpr}[1]{\overbracket[.1pt][1pt]{#1}^{\scalebox{.21}{PRF}}}
%\newcommand{\prrpr}[1]{\overbracket[.1pt][1pt]{#1}^{\tikz[baseline=0em]{ \node[scale=.21]{PRF}; }  }}
%\newcommand*{\prrpr}[1]{\mathop{\!\tikz[baseline=(charluca.base)]{ \node[label={[scale=.3,label distance=-3.9mm]north:PRF}] (charluca) {$\overbracket[.1pt][1pt]{#1}$};}\!} }

\newcommand{\prrpr}[1]{\overbracket[.1pt][1pt]{\underbracket[.1pt][1pt]{#1}}}

\newcommand{\rprpr}[1]{\underline{#1}}
\newcommand{\rprm}[1]{\llparenthesis #1\rrparenthesis}
\newcommand{\eqdef}{\overset{{\tiny def}}{=}}

\newcommand{\prZero}{\mathtt{0}}
\newcommand{\prSucc}{\mathtt{S}}
\newcommand{\prPred}{\mathtt{P}}
\newcommand{\prProj}[2]{\mathtt{P}^{#1}_{#2}} % P^n_i
\newcommand{\prCom}[1]{\circ{[#1]}} % composition
\newcommand{\prRec}[2]{\mathtt{R}_{#1,#2}} % recursion

\newcommand{\rprPerm}[2]{
\left(
\!\!\!\begin{array}{c}
#1\\
#2
\end{array}\!\!\!\right)}
\newcommand{\arityI}{arity\xspace}
\newcommand{\arityO}{co-arity\xspace}

%%%%%%%%%%%%%% Terms
\newcommand{\temporarychannel} {\textit{temporary argument}\xspace}
\newcommand{\temporarychannels}{\textit{temporary arguments}\xspace}
\newcommand{\rprWea}{\mathsf{w}}
\newcommand{\rprWeaName}{\textit{weakening}\xspace}
\newcommand{\rprNeg}{\mathsf{N}}
\newcommand{\rprNegName}{\textit{sign-change}\xspace}
\newcommand{\rprId}{\mathsf{I}}
\newcommand{\rprIdz}{\rprId}
\newcommand{\rprIdName}{\textit{identity}\xspace}
\newcommand{\rprZero}{\mathsf{0}}
\newcommand{\rprZeroName}{\textit{rZero}\xspace}
\newcommand{\rprSucc}{\mathsf{S}}
\newcommand{\rprSuccName}{\textit{successor}\xspace}
\newcommand{\rprPred}{\mathsf{P}}
\newcommand{\rprPredName}{\textit{predecessor}\xspace}
\newcommand{\rprSCom}[2]{#1\fatsemi #2} % serial composition
\newcommand{\rprSComName}{\textit{series composition}\xspace}
\newcommand{\rprSeq}{\fatsemi}
\newcommand{\rprPCom}[2]{ #1 \rVert #2} % parallel composition
\newcommand{\rprPComSX}[2]{\left. #1 \right\rVert #2}
\newcommand{\rprPComName}{\textit{parallel composition}\xspace}
%%\newcommand{\rprIf}[5]{\mathsf{If}^{#1}_{#2}\mathsf{[}#3,#4,#5\mathsf{]}} % original
\newcommand{\rprIf}[5]{\mathsf{If}^{#1}_{#2}\left[#3,#4,#5\right]} 
\newcommand{\rprIfName}{\textit{selection}\xspace}
%\newcommand{\rprIt}[3]{\mathsf{It}^{#1}_{#2}\mathsf{[}#3\mathsf{]}} %% original
\newcommand{\rprIt}[3]{{\mathsf{It}^{#1}_{#2}\left[ #3 \right]}}
\newcommand{\rprItName}{\textit{finite iteration}\xspace}

\newcommand{\rprSrc}[1]{\mathsf{Src[#1]}} % source constant
\newcommand{\rprSrcName}{\textit{source constant}\xspace}
\newcommand{\rprSrcNamePlur}{\textit{source constants}\xspace}
\newcommand{\rprSnk}[1]{\mathsf{Snk[#1]}} % sink constant
\newcommand{\rprSnkName}{\textit{sink constant}\xspace}
\newcommand{\rprSnkNamePlur}{\textit{sink constants}\xspace}
\newcommand{\rprGSwap}[1]{\rprBSwap^{#1}} % swap^k generic
%\newcommand{\rprSwap}[3]{\rprBSwap^{#1}_{#2\leftrightarrow #3}} % swap^k_i,j
\newcommand{\rprSwap}[3]{\left\rmoustache\!\!
\arraycolsep=1.4pt
\scriptsize
\begin{array}{l}
#2 \\
#3
\end{array}\!\!\right\rmoustache^{\!\!#1} } % swap^k_i,j
\newcommand{\rprSwapName}{\textit{binary rewiring permutation}\xspace}
\newcommand{\rprBSwap}{\chi} % swap^2 1 and 2
\newcommand{\rprBSwapName}{\textit{binary rewiring permutation}\xspace}

\newcommand{\rprMin}[1]{\mu #1} % minimalization
\newcommand{\rprMinName}{\textit{minimalization scheme}\xspace}
\newcommand{\rprSumSet}{\mathcal{T}_{\rprSum}}
\newcommand{\rprAbs}{\mathsf{A}}
\newcommand{\rprInc}{\mathsf{inc}}

\newcommand{\rprPush}{\mathsf{push}}
\newcommand{\rprPop}{\mathsf{pop}}
\newcommand{\rprSum}{\mathsf{sum}}
\newcommand{\rprSub}{\mathsf{sub}}
\newcommand{\rprDup}{\nabla}
\newcommand{\rprCopy}{\mathsf{copy}}
\newcommand{\rprDec}{\mathsf{dec}}
\newcommand{\rprMult}{\mathsf{mult}}
\newcommand{\rprDivTwo}{\mathsf{div2}}
\newcommand{\rprBitPN}{\mathsf{bitpn}} % bit positive/negative
\newcommand{\rprLess}{\mathsf{less}} % 
\newcommand{\rprSq}{\mathsf{sq}}
\newcommand{\rprSqRoot}{\mathsf{sqroot}}
\newcommand{\rprBMu}[5]{ %
% #1 indice della variabile che contiene il minimo cercato
% #2 indice della variabile che contiene il bound
% #3 funzione in RPRF che dipende (almeno) da #1 e scrive il risultato in #4
% #4 indice della variabile col risultato di #4
\mathsf{min}_{#1;#2} ({#3}_{#4;#5})}
\newcommand{\lpMU}[1]{ %
% #1 indice della variabile che contiene il minimo cercato
% #2 indice della variabile che contiene il bound
\lpMUNoArgs ({#1})}
\newcommand{\lpMUNoArgs}{\mathsf{min}}



\newcommand{\rprrs}[1]{\mathsf{rs}_{#1}} % recursive step
\newcommand{\rprrb}[1]{\mathsf{rb}_{#1}} % recursive base
\newcommand{\rprStep}[2]{\mathsf{S}_{#1,#2}} % recursive step of translation
\newcommand{\rprBase}[1]{\mathsf{B}_{#1}} % base step of translation
\newcommand{\rprRec}[3]{\mathsf{Rec[}#1,#2,#3\mathsf{]}} % recursive scheme
\newcommand{\rprREC}[4]{{\mathsf{Rec}^{#1}[}#2,#3,#4]} % recursive scheme

\newcommand{\rprInv}[1]{{#1}^{-1}} % inverse
\newcommand{\rprInvName}{\textit{inverse}\xspace}
\newcommand{\rprrvr}{\Downarrow} % rewriting relation
\newcommand{\rprEq}{\simeq} % equivalence relation on RPRT
\newcommand{\rprQuasiEq}{\equiv} % quasi equivalence relation on RPRT
\newcommand{\rprMea}{\mathfrak{m}} % measure function of a vector
\newcommand{\rprMeaName}{\textit{measure function}\xspace} % name of the measure funciton of a vector
\newcommand{\rprSquare}[1]{\mathsf{sq}^{#1}}
\newcommand{\revCompile}{\circledR}%{\mathsf{RV}}
\newcommand{\rprNabla}{\nabla} 
\newcommand{\rprDelta}{\Delta}

%%%%%%%%%%%%%%%%%%%%%%
\makeatletter
\newcommand{\dotminus}{\mathbin{\text{\@dotminus}}}
\newcommand{\@dotminus}{%
  \ooalign{\hidewidth\raise1ex\hbox{.}\hidewidth\cr$\m@th-$\cr}%
}
\makeatother
%%%%%%%%%%%%%%%%%%%%%%
\newcommand{\etc}{etc.\xspace}
\newcommand{\ie}{i.e.\xspace}
\newcommand{\Ie}{I.e.\xspace}
\newcommand{\KR}{\mathsf{KR}}
\newcommand{\GR}{\mathsf{GR}}
%\newcommand{\PR}{\mathsf{PR}}
\newcommand{\Acker}{\operatorname{\mathsf{ Ack}}}
\newcommand{\AckerInv}{\mathsf{Ack}^{-1}}
\newcommand{\RF}{\mathsf{RF}} % recursive functions
\newcommand{\PRF}{\mathsf{P}\RF} % primitive recursive functions
\newcommand{\RPRF}{\mathsf{R}\PRF} % reversible primitive recursive functions
\newcommand{\RPP}{\mathsf{RPP}} % reversible primitive permutations
\newcommand{\ESRL}{\mathsf{ESRL}} % extended simple reversible language
\newcommand{\JMF}{\mathcal{RI}} % Jacopini Mentrasti
\newcommand{\JPRF}{\mathsf{J}\PRF}
\newcommand{\BJPRF}{\mathsf{B}\PRF}
\newcommand{\rprF}{\mathsf{F}}

\newcommand{\ZtoN}{\Psi}% morphism
\newcommand{\NtoZ}{\Phi}% morphism
\newcommand{\CP}{\operatorname{Cp}}% morphism
\newcommand{\CU}{\operatorname{Cu}}% morphism
\newcommand{\TN}{\operatorname{Tn}}% 

\newcommand{\Set}[1]{\{#1\}}
\newcommand{\Nat}{\mathbb{N}}
\newcommand{\Int}{\mathbb{Z}}
\newcommand{\cat}{\cdot}

\newcommand\tikzmark[1]{\tikz[overlay,remember picture,baseline] \coordinate (#1);}

\def\vspnodeEmpty(#1)#2#3{% vertex series parallel graph node one input
  \begin{scope}[shift={(#1)}]
    \draw (0,0) ;
    \node[circle, fill=none, minimum size=20pt,inner sep=0pt] at (.5,.7) {#3};
  \end{scope}
}

\tikzstyle{vertex}=[circle,fill=black!25,minimum size=20pt,inner sep=0pt]
%%%%%%%%%%%%%%%%%%%%%
\def\vspnodeOneInOneOut(#1)#2#3{% vertex series parallel graph node one input
  \begin{scope}[shift={(#1)}]
    \draw (0,0) ;
    \node[vertex] at (.5,.5) {#3};
%%%IN
    \draw (0,0.5) coordinate (#2 In1);
%%%OUT
    \draw (1,0.5) coordinate (#2 Out1);
  \end{scope}
}

%%%%%%%%%%%%%%%%%%%%%
\def\fPR(#1)#2#3{%
  \begin{scope}[shift={(#1)}]
    \draw (0,0) rectangle (1,1);
    \node at (.5,.5) {#3};
%%%IN
    \draw ( 0,0.8) node[right] {} -- +(-.5,0) coordinate (#2 X1);
    \draw (-.5,0.6) node[right] {$ \vdots $};
    \draw ( 0,0.2) node[right] {} -- +(-.5,0) coordinate (#2 Xk);
%%%OUT
    \draw (1,0.5) -- +(.5,0) coordinate (#2 Out);
  \end{scope}
}

%%%%%%%%%%%%%%%%%%%%%
\def\fRPR(#1)#2#3{%
  \begin{scope}[shift={(#1)}]
    \draw (0,0) rectangle (2,2.2);
%    \draw (0.5,1) -- (0.5,0);
%    \draw (0.5,0.5) -- (1,0.5);
    \node at (1,1) {#3};
%    \node at (0.75,0.25) {$\bar{Q}$};
%%%IN
    \draw ( 0,2.0) node[right] {} -- +(-.5,0) coordinate (#2 X1);
    \draw (-.5,1.8) node[right] {$ \vdots $};
    \draw ( 0,1.4) node[right] {} -- +(-.5,0) coordinate (#2 Xk);
    \draw ( 0,1.1) node[right] {} -- +(-.5,0) coordinate (#2 Zero);
    \draw ( 0,0.8) node[right] {} -- +(-.5,0) coordinate (#2 Zero1);
    \draw (-.5,0.6) node[right] {$ \vdots $};
    \draw ( 0,0.2) node[right] {} -- +(-.5,0) coordinate (#2 Zerom);
%%%OUT
    \draw (2,2.0) -- +(.5,0) coordinate (#2 Out);
    \draw (2,1.6) node[right] {} -- +(.5,0) coordinate (#2 OutX1);
    \draw (2,1.45) node[right] {$ \vdots $};
    \draw (2,1.1) node[right] {} -- +(.5,0) coordinate (#2 OutXk);
    \draw (2,0.8) node[right] {} -- +(.5,0) coordinate (#2 OutZero1);
    \draw (2,0.6) node[right] {$ \vdots $};
    \draw (2,0.2) node[right] {} -- +(.5,0) coordinate (#2 OutZerom);
  \end{scope}
}


\newtheorem{proposition}{Proposition}
\newtheorem{theorem}{Theorem}
\newtheorem{lemma}[theorem]{Lemma}
\newtheorem{corollary}[theorem]{Corollary}

\newproof{prf}{Proof} % <---- !!

\newdefinition{fact}{Fact}
\newdefinition{definition}{Definition}
\newdefinition{remark}{Remark}
\newdefinition{example}{Example}
\newdefinition{notation}{Notation}


%%%%%%%%%%%%%%%%%%%%%%%%%%%%%%
%%%%%%%%%%%%%%%%%%%%%%%%% servono ad emacs NON CANCELLARE x piacere
%%% Local Variables:
%%% mode: latex
%%% TeX-master: "main.tex"
%%% ispell-local-dictionary: "american"
%%% End:

% The lineno packages adds line numbers. Start line numbering with
% \begin{linenumbers}, end it with \end{linenumbers}. Or switch it on
% for the whole article with \linenumbers.
% \usepackage{lineno}
%\usepackage{lineno,hyperref}
% \linenumbers
\begin{document}

\begin{frontmatter}

% Title, authors and addresses

% use the thanksref command within \title, \author or \address for footnotes;
% use the corauthref command within \author for corresponding author footnotes;
% use the ead command for the email address,
% and the form \ead[url] for the home page:
% \title{Title\thanksref{label1}}
% \thanks[label1]{}
% \author{Name\corauthref{cor1}\thanksref{label2}}
% \ead{email address}
% \ead[url]{home page}
% \thanks[label2]{}
% \corauth[cor1]{}
% \address{Address\thanksref{label3}}
% \thanks[label3]{}

\title{A class of Recursive Permutations
\\ which is Primitive Recursive complete}
%%%A class of Reversible Primitive Permutations \\ which is Primitive Recursive complete (and sound)}

\author[unito]{Luca~Paolini}
\ead{paolini@di.unito.it, luca.paolini@unito.it}
\ead[urlpa]{http://www.di.unito.it/~paolini}
\author[unito]{Mauro~Piccolo}
\ead{piccolo@di.unito.it, mauro.piccolo@unito.it}
\ead[urlpi]{http://www.di.unito.it/~piccolo}
\author[unito]{Luca~Roversi}
\ead{roversi@di.unito.it, luca.roversi@unito.it}
\ead[urlr]{http://www.di.unito.it/~rover}

\address[unito]{Dipartimento di Informatica -- Universit\`a di Torino}

\begin{abstract}

We focus on total functions in the theory of reversible computational models.
We define a class of recursive permutations, dubbed
Reversible Primitive Permutations ($ \RPP $) which are
computable invertible total endo-functions on integers, 
so a subset of total reversible computations.
$ \RPP $ is generated from five basic functions (identity, 
sign-change, successor, predecessor, swap),
two notions of composition (sequential and parallel), one functional iteration 
and one functional selection.
$ \RPP $ is closed by inversion and it is expressive enough to encode Cantor pairing and the whole 
class of Primitive Recursive Functions.
\end{abstract}

\begin{keyword}
Reversible Computation \sep  Unconventional Computing Models \sep Computable Permutations 
\sep Primitive Recursive Functions \sep Recursion Theory.
%\PACS code \sep code \PACS 
\end{keyword}
\end{frontmatter}

%%%%%%%%%%%%%%%%%%%%%%%%%%%%%%%%%%%%%%%%%%%%%%%%%%%%
\section{Introduction}\label{section:Introduction}

Mainstream models of computations focus on one of the two possible directions of computation. 
We typically think how to model 
the way the computation evolves from inputs to outputs while 
we (reasonably) overlook the behavior in the opposite direction, from outputs to inputs. 
Generally speaking, models of computations are neither backward deterministic nor reversible.

For a very rough intuition about what reversible computation deals with, we start by an example.
Let us think about our favorite programming language and think of implementing the
addition between two natural numbers
$ m $ and $ n $. Very likely --- of course without absolute certainty --- we 
shall end up with some procedure \texttt{sum} which takes two arguments and which yields their sum. 
For example, \texttt{sum}($ 3 $,$ 2 $) would yield $ 5 $.
What if we think of asking for the values $ m $ and $ n $ such that \texttt{sum}($ m $,$ n $) = $ 5 $?
If we had implemented \texttt{sum} in a prolog-like language, then we could exploit its 
non deterministic evaluation mechanism to list all the pairs $ (0,5), (1,4), (2,3), (3,2), (4,1) $ and
$ (5,0) $ every of which would well be a correct instance of ($ m, n $). 
In a reversible setting we would obtain
exactly the pair we started from. \Ie, the computation would be backward deterministic.
The interest for the backward determinism arose in the sixties of the last century, 
studying the thermodynamics of computation and the connections between information theory, computability and entropy. 
Since then, the interest for the reversible computing has slowly but incessantly grown.

A forcefully non-exhaustive list of references follows.
Reversible Turing machines are defined in \cite{axelsen11lncs,bennett73ibm,jacopini90siam,lecerf63} while
\cite{axelsen11lncs,axelsen16acta,jacopini89tcs,li1996royal} 
study some of their computational theoretic properties.
Many research efforts have been focused on  boolean functions and cellular 
automata as models of the reversible
computation \cite{Morita2008101,toffoli80lncs}.
Moreover, some research focused on reversible programming languages \cite{DBLP:conf/popl/JamesS12,yokoyama08acm}.
Of course, the interest to build a comprehensive knowledge about reversible computation is wider than 
the mere interest for its computational theoretic aspects. 
The book \cite{perumalla2013chc} is a comprehensive 
introduction to the subject. It spans from low-power-consumption reversible hardware to emerging alternative computational
models like quantum  \cite{guerriniMM15,zorzi14mscs} or bio-inspired \cite{giannini2015tcs} of which reversibility is one the unavoidable building blocks.

\paragraph{Goal} 
The focus of this work is on a \emph{functional model} of reversible computation. 
Specifically, we look for a  sufficiently expressive  class of recursive permutations able to represent all
 Primitive Recursive Functions ($ \PRF $) \cite{rogers1967theory,soare1987book}
whose relevance is sometimes traced to the slogan
``programs which terminate but do not belong to $ \PRF $ are rarely of practical interest.''


We aim at formalizing a language that we identify as Reversible Primitive Permutations ($ \RPP $) and which 
retains the more interesting aspects of $\PRF$.
In analogy to $ \PRF $, our goal is to get a functional characterization of computability
-- but in a reversible setting, of course --- because functions are handier in order to compose 
algorithms. Other models, like, for example, Turing machines-oriented ones are
more convincing from an implementation view-point.
The ability to represent Cantor pairing \cite{rosenberg2009book} is one of the main properties that 
the functional characterization we are looking for must satisfy.
With Cantor pairing available it is possible to express all interesting total properties about the 
traces\footnote{Kleene's $T_n$ predicate, Kleene's normal form theorem and technical tools related to them \cite{cutland1980book,odifreddi1989book,soare1987book}.}
of Turing machines, reversible or not.
The other must-have property of our functional characterization is closure under inversion, which is something
very natural to ask for in a class of permutations and of reversible computing models.

Our quest is challenging because various negative results could have played  against it.
First of all, we remark that the class of all (total) recursive permutations 
cannot be effectively enumerated (see \cite[Exercise 4-6, p.55]{rogers1967theory}). 
In \cite{koz1972ail} a constructive generation of all the recursive permutations is given starting
from primitive recursive permutations. Since no enumeration exists for the latter, none can exist for the former.
Worst, the class of primitive recursive permutations is not closed under inversion \cite{kuznecov50sssr,PaoliniPiccoloRoversiICTCS2015,soare1987book}.
Since the above negative results hold also for the class of elementary permutations 
\cite{cannonito1969jsl,kalimullin03permutations,koz1974ail}, there is no
hope to find any effective description neither of the class of recursive 
permutations nor of the classes of primitive recursive permutations and of elementary permutations.


\paragraph{Comparison with the known literature}
Our quest to identify the reversible  analogous of $ \PRF $  must be throughly related to the following works.
\begin{itemize}
\item 
The programming language $ \mathsf{SRL} $  restricts $ \mathsf{LOOP} $, a language 
for programming $\PRF$ functions \cite{meyer1967acm}. 
Matos introduces $ \mathsf{SRL} $ and some of its variants in \cite{matos03tcs}.
He is the first using $ \Int $ --- the natural numbers with sign --- as the ground type for 
classes of reversible functions.
We share the choice with him.
The work \cite{matos03tcs} focuses on the algebraic aspects of his languages and its relations with 
matrix groups. 

%\todo{Questions 1 and 2, Reviewer 1}
The study of the classes of functions that $ \mathsf{SRL} $ variants can identify
  is part of Matos' work.
Variants of $\mathsf{SRL}$ are complete with respect to reversible boolean circuits \cite{matos2016notes}.
Proving that some given variant of $\mathsf{SRL}$ is $ \mathsf{PRF} $-complete,
i.e. that it represents all the functions in $ \mathsf{PRF} $, naturally ends up with the quest 
to encode the ``test for 0'' like in the proof that $ \mathsf{LOOP} $ and $ \mathsf{PRF} $ are equivalent \cite{meyer1967acm}. 
We \emph{conjecture} that no variant of Matos' languages exists able to simulate a conditional 
control on the flow of execution that, instead, $ \RPP $ contains by definition.
Of course, proving the conjecture false, would promote $\mathsf{SRL}$ to be (in a reasonable sense) the minimal reversible and 
$ \mathsf{PRF} $-complete language.

\item The precursor of this work is \cite{PaoliniPiccoloRoversiICTCS2015}. It introduces the class of functions $ \RPRF $ which is closed by inversion and is $\PRF$-complete. 
The completeness of $ \RPRF $ relies on \emph{built-in} Cantor pairings.
In this paper we show that we can get rid of \emph{built-in} Cantor pairings inside $ \RPP $.
With no \emph{built-in} pairing operators at hand the design of $ \RPP $ relies on operators which are more fine-grained as compared
to the ones used for the definition of $ \RPRF $.
This orients the programming style to enjoy a couple of interesting features.
Inside $ \RPP $ it is natural avoiding to save the entire history of a calculation within a single argument, i.e. into a single ancilla.
This allows to clean the garbage at the end of the simulation of any $ f\in\PRF $, something which is reminiscent of the simulation 
of Turing machines devised in \cite{bennett1989siamjc}.

\item 
Finally we discuss \cite{jacopini89tcs}. 
It introduces the class of reversible functions $ \JMF $ which is as expressive as Kleene's
partial recursive functions \cite{cutland1980book,odifreddi1989book}. 
Therefore, the focus of \cite{jacopini89tcs} is on partial reversible functions while ours  is on total ones. 
The expressiveness of $ \JMF $ is clearly stronger than the one of  $ \RPP $. 
However, we see $ \JMF $ as less abstract than $ \RPP $ for two reasons.
On one side, the primitive functions of $ \JMF $ relies on a given specific binary representation of natural numbers.
On the other, it is not evident that $ \JMF $ is the extension of a total sub-class
analogous to $ \RPP $ which should ideally correspond to $ \PRF $, but in a reversible setting.
\end{itemize}

%%%%%%%%%%%%%%%%%%%%%%%%%%%%
\paragraph{Contents}
We propose a formalism that identifies a class of functions which we call Reversible Primitive Permutations
($ \RPP $) and which is strictly included in the set of permutations.
Section~\ref{section:The class RPP of Reversible Primitive Permutations} defines $ \RPP $ in analogy
with the definition of $ \PRF $, \ie $ \RPP $ is the closure of composition schemes on basic functions.

The functions of $ \RPP $ have identical arity and co-arity and take 
$ \Int $ --- and not only $ \Nat $ --- as their 
domain because $ \Nat $ is not a group. So, $ \RPP $ is sufficiently abstract 
to avoid any reference to any specific 
encoding of numbers and strongly connects our work to Matos' one \cite{matos03tcs}.

For example, in $ \RPP $ we can define a \texttt{sum} that given the two integers $ m $ and $ n $
yields $ m+n $ and $ n $. Let us represent \texttt{sum} as:
\begin{align}
\label{align:sum example}
\arraycolsep=1.4pt
\begin{array}{rcl}
 \left. {\scriptsize \begin{array}{r} m    \\ n \end{array}} \right[
 & \texttt{sum} &
 \left] {\scriptsize \begin{array}{l} m + n\\ n \end{array} } \right.
\end{array}
\enspace .
\end{align}
The implementation of \texttt{sum} inside $ \RPP $ exploits an iteration scheme that iterates $ n $
times the successor, starting on the initial value $ m $ of the first input.
$ \RPP $ implies the existence of a (meta and) \emph{effective} procedure which produces 
the inverse $ \rprInv{f}\in\RPP $ of every given $ f\in\RPP $. 
\Ie, :
\begin{align}
\label{align:diff example}
\arraycolsep=1.4pt
\begin{array}{rcl}
 \left. {\scriptsize \begin{array}{r} p    \\ n \end{array}} \right[
 & \rprInv{\texttt{sum}} &
 \left] {\scriptsize \begin{array}{l} p - n\\ n \end{array} } \right.
\end{array}
\end{align}
belongs to $ \RPP $ and it ``undoes'' \texttt{sum} by iterating $ n $ times
the predecessor on $ p $. So if $ p = m + n $, then $ p - n = m + n - n = m$.
We remark we could have internalized the operation that yields the inverse of a 
function inside $\RPP$ like in \cite{matos03tcs}. We do not internalize it
to avoid mutually recursive definitions in $ \RPP $.

Concerning the \emph{expressiveness}, $ \RPP $ is closed under inversion and it is both $ \PRF $-complete and $ \PRF $-sound.
Completeness is the really relevant property between the two because this means that 
$ \RPP $ subsumes the class $ \PRF $.
This result is in  Section~\ref{section:Completeness of RPP}. It requires quite a lot 
of preliminary work that one can find in Sections~\ref{section:Generalizations inside RPP}, 
\ref{section:A library of functions in TRRF} and~\ref{section:Cantor pairing} 
and whose goal is to encode a bounded minimalization and Cantor pairing. 
The embedding relies on various key aspects of reversible computing. The principal
ones are:  (i) the use of ancillary variables to clone information
and (ii) the compositional programming under the pattern that Bennett's trick dictates
(cf. Section~\ref{section:Some recursion theoretic side effects of RPP}.)

%%%%%%%%%%%%%%%%%%%%%%%%%%%%
\paragraph{Contributions}
 $\RPP$ is a total class of reversible functions closed under inversion, which is $ \PRF $-complete 
and sound. We think that it can play a key role in the setting of reversible computations analogous to that played by $\PRF$ in classical recursion theory.
In particular, it can be used to devise the analogous of Kleene's normal form theorem and Kleene's predicates in the reversible setting;
it can be also used to formalize a reversible arithmetic as the primitive recursive one (a.k.a. Skolem arithmetic).

%As a last remark, we underline that the speculations leading us to the identification
%of $ \RPP $ could slightly widen our foundational perspective on the ``realm'' 
%of (classic and reversible) recursive computations and on their mathematical formalizations. 
%We present our reflections in Section~\ref{section:Some recursion theoretic side effects of RPP} 
%together with (some) possible future work.




%%%%%%%%%%%%%%%%%%%%%%%%% servono ad emacs
%%% Local Variables:
%%% mode: latex
%%% TeX-master: "main.tex"
%%% ispell-local-dictionary: "american"
%%% End:
\section{The class $ \RPP $ of Reversible Primitive Permutations}
\label{section:The class RPP of Reversible Primitive Permutations}

The identification of $ \RPP $ merges ideas and observations on Primitive Recursive Functions 
($ \PRF $) \cite{peter1967book,malcev70book,matos03tcs,odifreddi1989book}, 
Toffoli's class of boolean circuits \cite{toffoli80lncs} and Lafont's 
algebraic theory on circuits \cite{Lafont2003257}. 
We quickly recall the crucial ideas we borrowed from the above papers in order to formalize the Definition \ref{RevPrimPermutations}.

Toffoli's  boolean circuits are invertible because they avoid erasing 
information. This is why the boolean circuits in \cite{toffoli80lncs} have
identical arity and co-arity. $\RPP$ adopts this policy for the same reasons.

In analogy with $ \PRF $, the class $ \RPP $ is defined by composing 
numerical basic functions by means of suitable composition schemes. 
The will to manipulate numbers suggests that the \rprSuccName $ \rprSucc $ must be in $ \RPP $. 
So $\rprInv{\rprSucc}$ --- \ie the \rprPredName $ \rprPred $ --- must be in $ \RPP $ as well because 
we want $ f^{-1} $ to be effective. This requires that the application of $ \rprPred$ to $ 0 $ remains
meaningful. Satisfying this requirement and keeping the definition of $ \RPP $ as much natural as possible 
suggests to extend both the domain and co-domain of every function in $ \RPP$ to $ \Int $ so that 
$ \rprPred $ applied to $ 0 $ can yield $ -1 $. 
In fact, working with $ \Int $ as atomic data
it is like working on $ \Nat $ up to some existing isomorphism between $ \Int $ and $ \Nat $.

The core of the composition schemes of $ \RPP $ comes from \cite{Lafont2003257}.
The series composition of functions is ubiquitous in functional computational model so $ \RPP $ uses one. 
The parallel composition of functions is natural in presence of co-arity greater than $ 1 $ so
$ \RPP $ relies on such a scheme.

\subsection{Preliminaries}\label{subSect:Preliminaries}
Let $ \Int $ be the set of integers and let $ \Int^{n}  $ be its Cartesian product whose elements 
are $ n $-tuples, for any $ n \in \Nat $. The $ 0 $-tuple is $ \la\,\ra $. The $ 1 $-tuple
is $ \la x\ra $ or simply $ x $. In general, we name tuples with $ n $ 
elements as $ \vec{a}^n, \vec{b}^n, \ldots $ 
or simply as $ \vec{a}, \vec{b}, \ldots $ if knowing the number of their components is not 
crucial.
By definition, the concatenation $ \cat: \Int^i \times \Int^j \longrightarrow \Int^{i+j} $ 
is such that $ \la x_1,\ldots,x_j\ra \cat \la y_1,\ldots,y_k\ra = \la x_1,\ldots,x_j,y_1,\ldots,y_k\ra$.
Whenever $ j = 1  $ we prefer to write $ x_1 \cat \la y_1,\ldots,y_k\ra $ in place of 
$ \la x_1\ra \cat \la y_1,\ldots,y_k\ra $. Analogously, $ \la x_1,\ldots,x_j\ra \cat y_1$ will 
generally stand for $ \la x_1,\ldots,x_j\ra \cat \la y_1\ra$. The empty tuple is the neutral element so
$ \vec{x} \cat \la\ra = \la\ra \cat \vec{x} = \vec{x} $.
In fact, we shall generally drop the explicit use of the concatenation operator `$ \cat $'. 
For example, this means that we will often replace 
$ \vec{a} \cat x \cat \vec{b} \cat \vec{c}^n \cat \vec{d} $ by
$ \vec{a}\, x\, \vec{b}\, \vec{c}^n\, \vec{d} $. 
%Finally, given $ \vec{a}^n $, we denote 
%$ \vec{a}_{\,i} $ the $ i $-th element $ x_i $ in $ \vec{a}^n $, for any $ i \leq n $.
 
\begin{definition}[Reversible Primitive Permutations]
\label{RevPrimPermutations}
Reversible Primitive Permutations (abbreviated as $ \RPP $) is a sub-class
of Reversible Permutations\footnote{For sake of precision,
we focus on Total Reversible Recursive 
Endo-Functions on $\Int^n$ for some $n\in\mathbb N$.}. 
By definition,  $ \RPP  = \bigcup_{k\in\Nat} \RPP^{k}$  where, for every $k\in\Nat$, the set
$ \RPP^{k}$ contains functions with identical \emph{\arityI} and \emph{\arityO} 
$ k$. The classes $ \RPP^{0}, \RPP^{1}, \ldots $ are defined by mutual induction.
\begin{itemize}
\item
The \rprIdName $ \rprId $,
the \rprNegName $  \rprNeg$,
the \rprSuccName $  \rprSucc$ and 
the \rprPredName $  \rprPred$ 
belong to $ \RPP^{1}$. We formalize their semantics, which is the one we may expect,
by means of two equivalent notations. 
The first is a standard functional notation that applies the function to the input and
produces an output:
\begin{align*}
\rprId   \la x\ra & := \la     x \ra \;,&&& \rprNeg  \la x\ra & := \la    -x \ra\;,
&&& \rprSucc \la x\ra & := \la x + 1 \ra \;,& \rprPred \la x\ra & := \la x - 1 \ra
\enspace .
\end{align*}

The second notation is relational. We write the function name between square brackets;
the input is on the left hand side and the output in on the right hand side:
\begin{align*}
\arraycolsep=1.4pt
\begin{array}{rcl}
 \left. {\scriptsize \begin{array}{r} 
                       x
                     \end{array}} \right[
 & \rprId &
 \left] {\scriptsize \begin{array}{l}
                       x
                     \end{array} } \right. \;,
\end{array}
&&&
\arraycolsep=1.4pt
\begin{array}{rcl}
 \left. {\scriptsize \begin{array}{r} 
                       x
                     \end{array}} \right[
 & \rprNeg &
 \left] {\scriptsize \begin{array}{l}
                       -x
                     \end{array} } \right. \;,
\end{array}
&&&
\arraycolsep=1.4pt
\begin{array}{rcl}
 \left. {\scriptsize \begin{array}{r} 
                       x
                     \end{array}} \right[
 & \rprSucc &
 \left] {\scriptsize \begin{array}{l}
                       x+1
                     \end{array} } \right. \;,
\end{array}
&&&
\arraycolsep=1.4pt
\begin{array}{rcl}
 \left. {\scriptsize \begin{array}{r} 
                       x
                     \end{array}} \right[
 & \rprPred &
 \left] {\scriptsize \begin{array}{l}
                       x-1
                     \end{array} } \right.
\end{array}
\enspace .
\end{align*}

\item 
The \rprSwapName $ \rprBSwap $ belongs to $ \RPP^{2}$.
The functional notation is the expected one:
\begin{align*}
\rprBSwap \la x,y\ra := \la y,x\ra
% \begin{array}{rcl}
%  \left. {\scriptsize \begin{array}{r} 
%                        x\\ y
%                      \end{array}} \right[
%  & \rprSwap{2}
%            {1,2}
%            {2,1} &
%  \left] {\scriptsize \begin{array}{l}
%                        y\\[0.35mm] x
%                      \end{array} } \right.
% \end{array}
\enspace .
\end{align*}
We use two interchangeable relational notations:
\begin{align*}
\arraycolsep=1.4pt
\begin{array}{rcl}
\left. {\scriptsize \begin{array}{r} 
	x\\ y
	\end{array}} \right[  &
\rprBSwap 
& \left] {\scriptsize \begin{array}{l}
	y\\[0.35mm] x
	\end{array} } \right.
\end{array}
  \textrm{ and  }
  \textcolor{red}{ \rprSwap{2} {1,2} {2,1} } %LP non funzionava: colorato a MANO !!!
  \begin{array}{rcl}
\left. {\scriptsize \begin{array}{r} 
	x\\ y
                    \end{array}} \right[  &\rprSwap{2} {1,2} {2,1}
  & \left] {\scriptsize \begin{array}{l}
	y\\[0.35mm] x
	\end{array} } \right.
  \end{array}
\enspace .
\end{align*}
The second one explicitly represents
 (i) the arity,
(ii) the input sequence of the arguments which is the upper sequence of numbers; 
and (iii) the output sequence of arguments in the lower sequence of numbers.
\item 
Let $ f, g\in\RPP^{j}$, for some $ j $.
The \rprSComName $ (\rprSCom{f}{g})$ belongs to $ \RPP^{j} $ and is such that:
\begin{align*}
(\rprSCom{f}{g}) \, \langle x_1,\ldots,x_j\rangle 
  & =  (g\circ f) \langle x_1,\ldots,x_j\rangle
\textrm{ or }
\end{align*}
\begin{align*}
\arraycolsep=1.4pt
\begin{array}{rcl}
 \left. {\scriptsize \begin{array}{r} 
                       x_1\\ \cdots\\ x_j
                     \end{array}} \right[
 & \rprSCom{f}{g} &
 \left] {\scriptsize \begin{array}{l}
                       y_1\\ \cdots\\ y_j
                     \end{array} } \right.
\end{array}
& =
\arraycolsep=1.4pt
\begin{array}{rcl}
 \left. {\scriptsize \begin{array}{r} 
                       x_1\\ \cdots\\ x_j
                     \end{array}} \right[
 & f &
 \left] {\hspace{-1.6em}
         \phantom{\scriptsize \begin{array}{r} 
                        x_1\\ \cdots\\ x_j
                      \end{array}
                 }
        } \right.
\end{array}
\arraycolsep=1.4pt
\begin{array}{rcl}
 \left. {\hspace{-1.6em}
          \phantom{\scriptsize \begin{array}{r} 
                         x_1\\ \cdots\\ x_j
                       \end{array}
                  }
         } \right[
 & g &
 \left] {\scriptsize \begin{array}{l}
                       y_1\\ \cdots\\ y_j
                     \end{array} } \right.
\end{array}
\enspace .
\end{align*}
We remark  the use of the programming composition that applies functions rightward, in opposition to the standard 
functional  composition (denoted by $\circ$). 
\item 
Let $ f\in\RPP^{j}$ and $ g\in\RPP^{k}$,
for some $ j $ and $ k $.
The \rprPComName 
$ (\rprPCom{f}{g})$ belongs to $ \RPP^{j+k} $ and is such that:
\begin{align*}
(\rprPCom{f}{g}) \, 
\langle x_1, \ldots, x_j\rangle
\langle y_1, \ldots, y_k\rangle & =  
(f\langle x_1, \ldots, x_j\rangle)\cdot (g\,\langle y_1, \ldots, y_k\rangle)
\textrm{ or }
\end{align*}
\begin{align*}
\arraycolsep=1.4pt
\begin{array}{rcl}
 \left. {\scriptsize \begin{array}{r} 
                       x_1\\ \cdots\\ x_j\\y_1\\ \cdots\\ y_k
                     \end{array}} \right[
 & \rprPCom{f}{g} &
 \left] {\scriptsize \begin{array}{l}
                       w_1\\ \cdots\\ w_j\\z_1\\ \cdots\\ z_k
                     \end{array} } \right.
\end{array}
& =
\arraycolsep=1.4pt
\begin{array}{c}
 \begin{array}{rcl}
  \left. {\scriptsize \begin{array}{r} 
                        x_1\\ \cdots\\ x_j
                      \end{array}} \right[
  & f &
  \left] {\scriptsize \begin{array}{r} 
                        w_1\\ \cdots\\ w_j
                       \end{array}
         } \right.
 \end{array}
 \vspace{.25em}
 \\
 \vspace{.25em}
 \arraycolsep=1.4pt
 \begin{array}{rcl}
  \left. {\scriptsize \begin{array}{r} 
                          y_1\\ \cdots\\ y_k
                        \end{array}
          } \right[
  & g &
  \left] {\scriptsize \begin{array}{l}
                        z_1\\ \cdots\\ z_k
                      \end{array} } \right.
 \end{array}
\end{array}
\enspace ,
\end{align*}
where $\cdot$ is the composition of sequences (cf. Section \ref{subSect:Preliminaries}).
%%%%%%%%%%%%%%%% Finite iteration
\item 
Let $ f\in\RPP^{k}$. 
The \rprItName $ \rprIt{}{}{f} $ belongs to $\RPP^{k+1}$ and is such that:
\begin{align*}
\rprIt{}{}{f}\, (\langle x_1,\ldots, x_k \rangle \cdot x) & =  
   (\rprPCom{(\overbrace{\rprSCom{f}{\rprSCom{\ldots}{f}}}^{|x|})}
            {\rprId})\,
             (  \langle x_1,\ldots, x_k \rangle \cdot x)
\textrm{ or }
\end{align*}
\begin{align*}
\arraycolsep=1.4pt
\begin{array}{rcl}
 \left. {\scriptsize 
           \begin{array}{r}
             x_1\\ \cdots\\ x_k\\[1.5mm]
             x
           \end{array} 
         } \right[
 & \rprIt{}{}{f} &
 \left] {\scriptsize 
          \begin{array}{l}
           % x\\[-1.5mm]
           \!\!
           \left .
           \begin{array}{l}
            y_1\\ \cdots\\ y_k
           \end{array} 
           \right \} = 
              (\underbrace{\rprSCom{f}{\rprSCom{\ldots}{f}}}_{|x|})
              \, \langle x_1, \ldots, x_k \ra\\
          x
          \end{array} 
        } \right.
\end{array}
\enspace .
\end{align*}
We remark the \emph{linearity constraint} on the \rprItName.
The argument~$x$ which drives the iteration unfolding cannot be among the arguments
of the iterated function $ f $.
Moreover, $ x $ is the last argument of $ \rprIt{}{}{f} $ in order to apply $ f $ to the first $ n $ arguments
of $ \rprIt{}{}{f} $ itself.

%%%%%%%%%%%%%%%% Selection
\item 
Let $ f, g, h\in\RPP^{k}$. 
The \rprIfName
 $ \rprIf{}{}{f}{g}{h} $ belongs to $\RPP^{k+1}$ and is such that:
\begin{align*}
\rprIf{}{}{f}{g}{h} \,  ( \langle x_1, \ldots, x_k \rangle \cdot x) & =  
\begin{cases}
(\rprPCom{f}{\rprId})\, ( \langle x_1, \ldots, x_k \rangle \cdot x) & \textrm{ if } x > 0\\
(\rprPCom{g}{\rprId})\, ( \langle x_1, \ldots, x_k \rangle \cdot x) & \textrm{ if } x = 0\\
(\rprPCom{h}{\rprId})\, ( \langle x_1, \ldots, x_k \rangle \cdot x) & \textrm{ if } x < 0
\end{cases}
\quad \textrm{ or  }
\end{align*}
\begin{align*}
\arraycolsep=1.4pt
\begin{array}{rcl}
 \left. {\scriptsize 
           \begin{array}{r}
             x_1\\ \cdots\\ x_k\\[1.5mm] x
           \end{array} 
         } \right[
 & \rprIf{}{}{f}{g}{h} &
 \left] {\scriptsize 
          \begin{array}{l}
           % x \\[-1.5mm]
           \!\!
           \left .
           \begin{array}{l}
            y_1\\ \cdots\\ y_k
           \end{array} 
           \right \} = 
              \begin{cases}
               f\,\langle x_1, \ldots, x_k \rangle & \textrm{ if } x > 0\\
               g\,\langle x_1, \ldots, x_k \rangle & \textrm{ if } x = 0\\
               h\,\langle x_1, \ldots, x_k \rangle & \textrm{ if } x < 0
              \end{cases}\\
              x
          \end{array} 
        } \right.
\end{array}
\enspace .
\end{align*}
We remark the \emph{linearity constraint} imposed on the definition of \rprIfName.
The argument~$x$ which determines which among $ f, g $ and $ h $ must be used
cannot be among the arguments of $ f, g $ and $ h $.
Moreover, we choose to use $ x $ as the last argument of $ \rprIf{}{}{f}{g}{h} $ to avoid any 
re-indexing of the arguments of $ f, g $ and $ h $, once we choose one of them.
\qed
\end{itemize}
\end{definition}

%%%%%%%%%%%%%%%%
\begin{notation}
If the same value occurs in consecutive inputs or outputs, like, for example, in:
\begin{align*}
f\langle x_1, \ldots, x_m, \underbrace{0, \ldots, 0}_{r\in\Nat}, y_1,\ldots, y_n \rangle
 & = \langle w_1, \ldots, w_p, \underbrace{0, \ldots, 0}_{s\in\Nat}, z_1,\ldots, z_q \rangle
\enspace ,
\end{align*}
with $ m+r+n = p+s+q $, we will often abbreviate it as:
\begin{align*}
f\langle x_1, \ldots, x_m, 0^r, y_1,\ldots, y_n \rangle
 & = \langle w_1, \ldots, w_p, 0^s, z_1,\ldots, z_q \rangle
\textrm{ or  }
\arraycolsep=1.4pt
\begin{array}{rcl}
 \left. {\scriptsize 
            \begin{array}{r}
             x_1 \\ \cdots \\ x_m \\[1mm] 0^r \\ y_1 \\ \cdots \\ y_n
            \end{array} 
         } \right[
 & f &
 \left] {\scriptsize 
            \begin{array}{r}
             w_1 \\ \cdots \\ w_p \\[1mm] 0^s \\ z_1 \\ \cdots \\ z_q
            \end{array} 
        } \right.
\end{array}
\enspace .
\end{align*}
In particular, $ 0^0 $ means that no occurrences of $ 0 $ exist. 
\qed
\end{notation}

The following proposition certifies that identifying the inverse of a term inside $ \RPP $ 
is a simple task. I.e., given the representation of a
function $ f $ in $ \RPP $,  generating $ f^{-1} $ is \emph{effective} and 
such that $(y,x)\in f^{-1}$ if and only if $(x,y)\in f$.

%%%%%%%%%%%%%%%%
\begin{proposition}[The (syntactical) inverse $ \rprInv{f} $ of any $ f $]
\label{proposition:The function rprInv}
Let $ f\in\RPP^{j}$, for any $ j\in\Nat $. The \rprInvName of $ f$ is
$ \rprInv{f}$, belongs to $ \RPP^{j} $ and, by definition, is:
$$ 
\begin{array}{l}
\rprInv{\rprId} := \rprId   \;,\; \rprInv{\rprNeg}    := \rprNeg  
                            \;,\;  \rprInv{\rprSucc}  := \rprPred  
                            \;,\;   \rprInv{\rprPred} := \rprSucc    
                            \;,\;  \rprInv{\rprBSwap} := \rprBSwap   \;,\;  \\[1mm]
                            \rprInv{(\rprSCom{g}{f})} := \rprSCom{\rprInv{f}}{\rprInv{g}} 
                             \;,\;   \rprInv{(\rprPCom{f}{g})} := \rprPCom{\rprInv{f}}{\rprInv{g}}  \;,\;  \\[1mm]
                              \rprInv{(\rprIt{}{}{f})}       := \rprIt{}{}{\rprInv{f}}  
                            \;,\; \rprInv{(\rprIf{}{}{f}{g}{h})}  := \rprIf{}{}{\rprInv{f}}{\rprInv{g}}{\rprInv{h}}
  \enspace .
\end{array}
$$
Then $ \rprSCom{f}{\rprInv{f}} = \rprId $ and $\rprSCom{\rprInv{f}}{f} = \rprId $.
\end{proposition}
\begin{prf}
By induction on the definition of $ f $.
\qed
\end{prf}
\noindent
Proposition~\ref{proposition:The function rprInv} shows that $ \RPP $ allows for a very smooth, syntactically driven, 
definition of the operation that makes the inverse of a given $ f $ effectively available. Other methods can exist. For example, 
\cite{McC56,robinson1950pams} pursue a sort of ``brute force'' method which is not at all syntax directed.

% We emphasize that, in the course of the design of $ \RPP $, 
% we aimed at sticking as much as we could to the traditional 
% $\PRF $ formalism, in order to keep a reasonable balance between 
% conciseness and easiness of usage.
%\todo{Question 2, Reviewer 1}
	Of course, $\RPP$ is syntactically redundant.
	For example, 
	both the parallel composition of two occurrences of the unary identity and
	the series-composition of two swaps yield the identity with arity 2.
	Moreover, the selection needs not be defined on three (higher-order) arguments as we do.
	Superfluous ingredients help to make the programming with $ \RPP $ a \LP{}{bit more} reasonably easy task.
%	\LPx{Establishing which one is better among $ \RPP $ and the languages for reversible computations in 
%	\cite{matos03tcs,PaoliniPiccoloRoversiICTCS2015}, for example, requires a lot of programming exercise
%	to see how the primitives orient the design of algorithms.}{ \todo{Non ha senso confrontare linguaggi senza stabilire 
%quale qualita si vuole misurare e quale metro si vuole usare tra i possibili: ANVUR docet!}}


%%%%%%%%%%%%%%%%
\begin{proposition}[Relating \rprSComName and \rprPComName]
\label{proposition:Relating rprSComName and rprPComName}
Let $ f, g\in\RPP^{j}$ and $ f', g'\in\RPP^{k}$ with $ j, k\in\Nat $. Then:
\begin{align*}
\rprSCom{(\rprPCom{f}{f'})}{(\rprPCom{g}{g'})}
=
\rprPCom{(\rprSCom{f}{g})}{(\rprSCom{f'}{g'})}
\enspace .
\end{align*}
\end{proposition}
\begin{prf}
By definition:
\begin{align*}
\rprSCom{(\rprPCom{f}{f'})}{(\rprPCom{g}{g'})}\, \vec{a}\,\vec{b} & = 
  (\rprPCom{g}{g'}) (\rprPCom{f}{f'})\, \vec{a}\,\vec{b}\\
  & = (\rprPCom{g}{g'})\, (f\vec{a}) \, (f'\vec{b})\\
  & = (g f\vec{a})  (g'f'\vec{b})\\
  & = ((\rprSCom{f}{g})\,\vec{a}) ((\rprSCom{f'}{g'})\,\vec{b})\\
  & = (\rprPCom{(\rprSCom{f}{g})}{(\rprSCom{f'}{g'})})\,\vec{a} \,\vec{b}
  \enspace .\qed
\end{align*}
\end{prf}

%\todo{Questions 3 and 5, Reviewer 1.}
	We conclude with some comments on
	the relevance of having functions with identical input and output arity in $ \RPP $.
	Toffoli's works on reversible boolean circuits influenced our choice. 
	However, in \cite{paolini2017ngc} we extend a reversible language  with a bijective built-in map from 
	$ \Int\times\Int $ to $\Int $ such that $ \langle 0,0\rangle \mapsto 0 $.
	Clearly, the input/output symmetry breaks up and we move from permutations to isomorphisms.
	The built-in bijection  allows to generate as many ancillary arguments as needed
	for fully developing reversible computations. 
    Once finished, ancillae will be repackaged into a single argument.

%\LP{}{\todo{Questions 3 and 5, Reviewer 1}
%In this papers we focused on permutations corresponding to $\PRF$, mainly, for sake of simplicity.
%However, this restriction is essentially a contingent choice:
%indeed, in \cite{paolini2017ngc} we  developed a proposal overcoming it by involving recursive bijections not arity-respecting
%(namely, being not permutations).
%In particular, we consider Cantor-pairing like functions in $\Int\rightarrow \Int\times\Int$ that should make us able to 
%generate how many ancillary arguments we need.
%}


%%%%%%%%%%%%%%%%%%%%%%%%% servono ad emacs
%%% Local Variables:
%%% mode: latex
%%% TeX-master: "main.tex"
%%% ispell-local-dictionary: "american"
%%% End:
\section{Generalizations inside $ \RPP $}
\label{section:Generalizations inside RPP}

We introduce formal generalizations of elements in $\RPP$. This helps simplifying the 
use of $ \RPP $ as a programming notation.

\paragraph{Weakening inside $ \RPP $}
For any given $ f\in\RPP^m $ an infinite set 
$ \Set{\rprWea^n(f) \mid n\geq 1 \text{ and } \rprWea^n(f) \in \RPP^{m+n}}$ 
exists such that:
\begin{align*}
\arraycolsep=1.4pt
\begin{array}{rcl}
 \left. {\scriptsize 
           \begin{array}{r}
            \\[-2mm]
             x_1\\ \cdots\\ x_m
            \\
             x_{m+1}\\ \cdots\\ x_{m+n}
           \end{array} 
         } \right[
 & \rprWea^n(f) &
 \left] {\scriptsize 
          \begin{array}{l}
           \!\!
           \left .
           \begin{array}{l}
            y_1\\ \cdots\\ y_m
           \end{array} 
           \right \} = 
               f\,\la x_1, \ldots, x_m \ra
             \\
             x_{m+1}\\ \cdots\\ x_{m+n}
          \end{array} 
        } \right.
\end{array}
\enspace ,
\end{align*}
for every $ \rprWea^n(f)$. We call $ \rprWea^n(f) $ \rprWeaName of $ f $ which we can obviously obtain by \rprPComName
of $ f $ with $ n $ occurrences of $ \rprId $. In general, if we need some \rprWeaName of a given 
$ f $, then we shall not write $ \rprWea^n(f) $ explicitly. We shall instead use $ f $ and say that
it is the \rprWeaName of $ f $ itself.


%%%%%%%%%%%%%%%
\paragraph{Generalized \rprIdName, \rprNegName, \rprSuccName and \rprPredName}
For every $ i\leq n $, the following functions in $ \RPP^n $ exist:
\begin{align*}
 \rprId^n\, \la x_1, \ldots, x_{i-1}, x_i, x_{i+1}, \ldots, x_n \ra
 & = \la x_1, \ldots, x_{i-1}, x_i, x_{i+1}, \ldots,  x_n\ra
\\
 \rprNeg^n_i \, \la x_1, \ldots, x_{i-1}, x_i, x_{i+1}, \ldots, x_n \ra
 & = \la x_1, \ldots, x_{i-1}, -x_i, x_{i+1}, \ldots,  x_n\ra
\\
 \rprSucc^n_i \, \la x_1, \ldots, x_{i-1}, x_i, x_{i+1}, \ldots, x_n \ra
  & = \la x_1, \ldots, x_{i-1}, x_i + 1, x_{i+1}, \ldots,  x_n\ra
\\
 \rprPred^n_i \, \la x_1, \ldots, x_{i-1}, x_i, x_{i+1}, \ldots, x_n \ra
  & = \la x_1, \ldots, x_{i-1}, x_i - 1, x_{i+1}, \ldots,  x_n\ra
\enspace .
\end{align*}
They are the weakening of
$ \rprId, \rprNeg, \rprSucc, \rprPred\in\RPP^1 $. When the arity $ n $ is clear from the 
context, we allow to use $ \rprId, \rprNeg_i, \rprSucc_i $ and $ \rprPred_i $
in place of $ \rprId^n, \rprNeg^n_i, \rprSucc^n_i $ and $ \rprPred^n_i $.

%%%%%%%%%%%%%%%
\paragraph{Generalized rewiring permutations}

%\todo{Question 1. \# Editor. }
	The literature offers terminology and notation related to the reversible computing which is far from being uniform.
	Just as an example, the class \LP{od}{of} \emph{invertible functions} of \cite[p.2]{cutland1980book} differs
	from the namesake class that \cite{lane1999algebra} defines.
	From \cite{lane1999algebra} we take that a permutation on a set $S$ is a bijection on $S$, i.e. an endo-bijection
	on $ S $. Very often, the intended meaning of ``permutation'' is ``finite permutation on a given (finite) set $ S $.''
	Generally speaking, this is not our case at least because, as we will see in Section~\ref{section:Cantor pairing}, 
	we deal with pairing inside $ \RPP $. When dealing with permutations of $ \RPP $ whose only
	goal is to re-arrange the finite set $\{1, \ldots, k\}$ of indexes of their arguments, we talk about
	\emph{rewiring permutations}.

\begin{proposition}[Rewiring Permutations in $ \RPP $.]
	\label{finitePermutationGeneration}
	Each rewiring permutation with arity and co-arity equal to $k$ belongs to $\RPP^{k}$.
\end{proposition}
\begin{prf}
Fixed a rewiring permutation to represent, it is enough to 
    suitably compose \rprIdName, \rprBSwapName, \rprSComName, \rprPComName and \rprWeaName.
	\qed
\end{prf}
Let $\Set{i_1,\ldots,i_n} = \Set{1,\ldots,n}$ and $ \rho:\Nat^n\longrightarrow\Nat^n $
The notation for a rewiring permutation in $\RPP^n $, sending the $ i_1 $-th input to the $ \rho(i_1) $-th output, 
the $ i_2 $-th input to the $ \rho(i_2) $-th output \etc is:
\begin{align}
\label{align:generalized permutation}
\AUTOrprSwap{n}{i_1,\ldots,i_m}{\rho(i_1),\ldots,\rho(i_m)}
\enspace .
\end{align}

%%%%%%%%%%%%%%%
\paragraph{Generalized \rprItName}
Let $ f\in\RPP^n$ and $ m \geq n + 1$. 
Let $ \langle i\rangle$, $L=\langle i_1, \ldots, i_n \rangle$ and 
$ \langle j_1, \ldots, j_{m-n-1} \rangle$ be a partition of natural numbers in the interval   $\langle 1,\ldots,m \rangle$.
We are going to define $ \rprIt{m}{i,L}{f} $ such that 
its argument of position $ i $ drives an iteration of $ f $ from $ L $. \Ie, 
$ \rprIt{m}{i,L}{f}\, \langle x_{1}, \ldots, x_{m} \rangle = \langle y_{1}, \ldots, y_{m} \rangle $, where: 
\begin{align*}
y_i & = x_i \\
\la y_{j_1},\ldots, y_{j_{m-n-1}} \ra  & =
 \la x_{j_1},\ldots, x_{j_{m-n-1}} \ra \\
\la y_{i_1},\ldots, y_{i_n} \ra  & =
 (\underbrace{\rprSCom{f}{\rprSCom{\ldots}{f}}}_{|x_i|})\, L 
 \enspace .
\end{align*}
We observe that $ \rprIt{m}{i,L}{f}$ is the identity on $ \langle x_{j_1}, \ldots, x_{j_{m-n-1}}\rangle $. 
By definition:
\scalebox{.93}{$\rprIt{m}{i,L}{f} 
 := 
   \AUTOrprSwap{m}{1,\ldots, n  ,n+1,n+2,\ldots,m  } {   i_1,\ldots, i_n,i,j_1,\ldots,j_{m-n-1}   } 
   \rprSeq (\rprPCom{\rprIt{}{}{f}} {\rprId^{m-n-1}}) 
   \rprSeq
   \rprInv{\left(\AUTOrprSwap{m}{1,\ldots\, n  ,n+1, n+2,\ldots,m }{i_1,\ldots\, i_n,i,j_1,\ldots,j_{m-n-1} } \right) }
   \! .  $}
\\[3mm]
The condition on $ i $ preserves the linearity constraint of $ \rprIt{}{}{f} $. 
%The definition of $ \rprIt{m}{i,L}{f} $ holds for any $ m $. We shall drop it when clear from the context.

%%%%%%%%%%%%%%%%%
\paragraph{Generalizing the \rprIfName} 
Let $ f,g,h\in\RPP^n$ and $ m \geq n + 1$.  
Let $ \langle i\rangle$, $L=\langle i_1, \ldots, i_n \rangle$ and 
$ \langle j_1, \ldots, j_{m-n-1} \rangle$ be a partition of natural numbers in the interval   $\langle 1,\ldots,m \rangle$.
We are going to define $ \rprIf{m}{i,L}{f}{g}{h} $ such that its argument of position $ i $ determines
which among $ f, g $ and $ h $ must be applied to $ L $. \Ie, 
$ \rprIf{m}{i,L}{f}{g}{h} \, \la x_1, \ldots, x_m \ra = \la y_1, \ldots, y_m \ra $ where:
\begin{align*}
y_i & = x_i \\
\la y_{j_1},\ldots, y_{j_{m-n-1}} \ra  & =
 \la x_{j_1},\ldots, x_{j_{m-n-1}} \ra \\
\la y_{i_1},\ldots, y_{i_n} \ra & = 
\begin{cases}
  f\, \la x_{i_1},\ldots, x_{i_n} \ra  & \textrm{ if } x_i > 0\\
  g\, \la x_{i_1},\ldots, x_{i_n} \ra  & \textrm{ if } x_i = 0\\
  h\, \la x_{i_1},\ldots, x_{i_n} \ra  & \textrm{ if } x_i < 0
    \enspace .
\end{cases}
\end{align*}
We observe that $ \rprIf{m}{i,L}{f}{g}{h}$  is the identity on $ \la x_{j_1},\ldots, x_{j_{m-n-1}} \ra $. 
By definition:
\begin{align*}
\rprIf{m}{i,L}{f}{g}{h}
  := 
  & \AUTOrprSwap{m}{1 \ldots  n ,n+1,n+2, \ldots ,m } {i_1 \ldots i_n,i,j_1, \ldots, j_{m-n-1}}  \\
  & \rprSeq 
    (\rprPCom{\rprIf{}{}{f}{g}{h}}  {\rprId^{m-n-1}}) \\
  & \rprSeq 
    \rprInv{\left(\AUTOrprSwap{m} {1, \ldots , n ,n+1,n+2 ,\ldots ,m }{i_1, \ldots, i_n,i,j_1 ,\ldots, j_{m-n-1}} \right) }
   \enspace .
\end{align*}
The condition on $ i $ preserves the linearity constraint 
of $ \rprIf{m}{i}{f}{g}{h} $. 
%The definition of $ \rprIf{}{i,L}{f}{g}{h} $ holds for any $ m $. 
%We shall drop it when clear from the context.

%%%%%%%%%%%%%%%%%%%%%%%%%
%%% Local Variables:
%%% mode: latex
%%% TeX-master: "main.tex"
%%% ispell-local-dictionary: "american"
%%% End:
\section{A library of functions in $ \RPP $}
\label{section:A library of functions in TRRF}

% We introduce functions to  further simplify the constructions inside $\RPP$.
% Quite often, the functions we are going to introduce rely on auxiliary arguments 
% we dub as \emph{ancillary arguments\LP{/variables}{ *** se fai una ricerca in questo file NON TROVI mai la parola ``variable'', al piu trovi ``position'' che considero un eufimesmo comodo quanto rozzo, secondo me parlare di variabili alimenterebbe confusione *** }} or \emph{ancillae} in analogy to 
% \cite{thomsen2015lncs}, for instance. 
% Ancillae have two main purposes. Either they temporarily record  copies of values for later use and are set back 
% to their initial value at the end of a computation 
% or
% they eventually hold the result of a computation, if properly initialized.
% \LP{}{Adapting the nomenclature introduced in \cite{toffoli80lncs}, 
% we say that an ancilla is a \temporarychannel whenever its initial and final value are the same.}%%%
% \LR{}{*** Secondo me il commento in rosso contraddice quel che abbiamo detto giusto prima a proposito
% del fatto che le ancille servono per memorizzare il risultato di una computazione. *** }\LP{ ***
% Le ancillae sono argomenti ausiliari utili alla computazione, che hanno due usi:
% ancille-risultato e ancille-temporanee! Dato che secondo, ***}

We introduce functions to  further simplify programming inside $\RPP$ by relying on \temporarychannels,
 \emph{ancillary arguments} or \emph{ancillae}.
They are additional arguments that can:
(i) hold the result of the computation;
(ii) temporarily record copies of values for later use;
(iii) temporarily store intermediate results.
Without loss of generality, we use ancillae in a very disciplined way in order to simplify their understanding.
Tipically, ancillae are initialized to zero. We cannot forget (reversibly) the content of ancillae, 
so they have to be carefully considered in compositions albeit sometimes we neglect their contents
(tipically, just forwarding them). 
Notions that recall the meaning of ancillae are common to the studies on reversible and quantum computation. They are
``temporary'' lines, storages, channels, bits and variables \cite{toffoli80lncs}.
%Moreover, following the quantum literature sometimes the ``ancillary'' replaces the adjective ``temporary''  (e.g. see \cite{thomsen2015lncs}).
% (Some references are \cite{abdessaied16book,DeVos2010book,DeVos12rc,toffoli80lncs,thomsen2015lncs}).
%%%%%%%%%%%%%%%
\paragraph{General increment and decrement} 
Let $ i, j $ and $ m\in \Nat $ be distinct.
Two functions $ \rprInc_{j;i}, \rprDec_{j;i}\in \RPP^m $ exist such that:
\begin{align*}
\arraycolsep=1.4pt
\begin{array}{rcl}
 \left. {\scriptsize \begin{array}{r} x_1\\ \cdots \\ x_j\\ \cdots \\ x_i       \\ \cdots \\x_m\end{array}} \right[
 & \rprInc_{j;i} &
 \left] {\scriptsize \begin{array}{l} x_1\\ \cdots \\ x_j\\ \cdots \\ x_i + |x_j|\\ \cdots \\x_m \end{array} } \right.
\end{array}
&\qquad\textrm{and}&
\arraycolsep=1.4pt
\begin{array}{rcl}
 \left. {\scriptsize \begin{array}{r} x_1\\ \cdots \\ x_j\\ \cdots \\ x_i      \\ \cdots \\x_m\end{array}} \right[
 & \rprDec_{j;i} &
 \left] {\scriptsize \begin{array}{l} x_1\\ \cdots \\ x_j\\ \cdots \\ x_i - |x_j|\\ \cdots \\x_m \end{array} } \right.
\end{array}
\enspace.
\end{align*}
We define them as:
\begin{align*}
\rprInc_{j;i}  & := \rprIt{m}{j,\langle i\rangle }{\rprSucc} 
&&&
\rprDec_{j;i}  & := \rprIt{m}{j,\langle i\rangle}{\rprPred}
\enspace .
\end{align*}

%%%%%%%%%%%%%%%
\paragraph{A function that compares two integers}
Let $k,j,i, p,q$ be pairwise distinct such that $k,j,i, p,q \leq n$, for a given arity $ n\in\Nat $.
A function $ \rprLess_{i,j,p,q;k}\in\RPP^n $ exists that implements the following relation:
\begin{align*}
\arraycolsep=1.4pt
\begin{array}{rcl}
  \left. {\scriptsize 
          \begin{array}{r} 
             x_1\\ \cdots 
             \\[1.10mm]
             x_k
             \\[1.25mm] 
             \cdots \\
             x_n
          \end{array}} 
  \right[
 & \rprLess_{i,j,p,q;k} &
   \left] {\scriptsize 
          \begin{array}{l}
             x_1\\ \cdots 
           \\[-1mm]
            x_k
            + \begin{cases}
                1 & \textrm{ if } x_i <    x_j\\
                0 & \textrm{ if }      x_i \geq x_j
                \enspace .
            \end{cases}
           \\[-1.10mm]
           \cdots \\ x_n
          \end{array} 
           } 
   \right.
\end{array}
\end{align*}

The function $\rprLess_{i,j,p,q;k}$ is expected to behave in accordance with the intended meaning whenever the 
arguments $ x_p, x_q $ are initially contains $0$, viz. they are ancillae.
Tipically, $x_k$ will be used as ancilla initialized to zero and it is not a temporary argument.
If the value of $x_k$ is initially $ 0 $, then $\rprLess_{i,j,p,q;k}$ returns  $ 0 $ or $ 1 $ in $ x_k $ 
depending on the result of comparing the values $ x_i $ and $ x_j $.  
Both $x_p$ and $x_q$ serve to duplicate the values of $x_i$ and $x_j$, respectively.
The copies allow to circumvent the linearity constraints in presence of nested selections.

Let  $ \Set{j_1,\ldots, j_{n-5}} = \Set{1,\ldots, n}\setminus \Set{k,j,i, p,q} $. We define
% \begin{subequations}
%   \begin{align}
%     \rprLess_{i,j,p,q;k} := &\ \AUTOrprSwap{n}{k,p,q,i,j,j_1,\ldots,j_{n-5}}
%                                             {1,2,3,4,5,6  ,\ldots,n } \rprSeq\
%     \rprInc_{5;3} \ \rprSeq\  \rprInc_{4;2}    \      \rprSeq
%     \label{less:subeq1}  \\
%     &   (\mathsf{F} \  \rVert \ \rprId^{n-5})  \ \rprSeq 
%       \label{less:subeq2}\\
%     &
%     \rprInv{\big(\rprInc_{5;3} \ \rprSeq\  \rprInc_{4;2} \big) }
%     \rprSeq\ \rprInv{\left( \AUTOrprSwap{n}{k,p,q,i,j,j_1,\ldots,j_{n-5}}
%                                             {1,2,3,4,5,6  ,\ldots,n } \right)} 
%     \label{less:subeq3}
%     \\
%    \nonumber      \mbox{where}\qquad\quad\\  
%    \nonumber
%     \mathsf{F}     := &\
%     \mathsf{If}^{5}_{5,\langle1,2,3,4\rangle}   [
%           \mathsf{If}^{4}_{4,\langle1,2,3\rangle}  [\mathsf{SameSign}, \rprSucc^3_1,  \rprSucc^3_1 ]  \\
%     \nonumber   & \phantom{\mathsf{If}^{5}_{5,\langle1,2,3,4\rangle}\ }
%     ,\mathsf{If}^{4}_{4,\langle1,2,3\rangle}   [ \rprId^3 , \rprId^3 ,    \rprSucc^3_1   ]  \\
%     \nonumber   &\phantom{\mathsf{If}^{5}_{5,\langle1,2,3,4\rangle}\ }
%     ,\mathsf{If}^{4}_{4,\langle1,2,3\rangle} [\rprId^3 ,   \rprId^3  ,\mathsf{SameSign}]] \\[5mm]
%     \nonumber%\mathsf{BothPos}
%     \mathsf{SameSign} := &\ \rprDec_{2;3}
%                            \rprSeq \mathsf{If}^{3}_{3,\langle1,2\rangle}[\rprSucc^2_1,\rprId^2,\rprId^2]
%                            \rprSeq  \rprInv{(\rprDec_{2;3})}
%     % \nonumber
%     % \mathsf{BothNeg} := &\ %\rprInc_{2;3} 
%     %                        \rprDec_{2;3}
%     %                        \rprSeq \mathsf{If}^{3}_{3,\langle1,2\rangle}[\rprSucc^2_1, \rprId^2, \rprId^2 ] 
%     %                        \rprSeq \rprInv{ (\rprDec_{2;3}) }
%     \enspace .
%   \end{align}
% \end{subequations}

  \begin{align}
    \rprLess_{i,j,p,q;k} := &\ \AUTOrprSwap{n}{k,p,q,i,j,j_1,\ldots,j_{n-5}}
                                            {1,2,3,4,5,6  ,\ldots,n } \rprSeq\
    \rprInc_{5;3} \ \rprSeq\  \rprInc_{4;2}    \      \rprSeq
    \label{less:subeq1}  \\
    &   (\mathsf{F} \  \rVert \ \rprId^{n-5})  \ \rprSeq 
      \label{less:subeq2}\\
    &
    \rprInv{\big(\rprInc_{5;3} \ \rprSeq\  \rprInc_{4;2} \big) }
    \rprSeq\ \rprInv{\left( \AUTOrprSwap{n}{k,p,q,i,j,j_1,\ldots,j_{n-5}}
                                            {1,2,3,4,5,6  ,\ldots,n } \right)} 
    \label{less:subeq3}
      \end{align}
 where $\mathsf{F} $ is
$$\mathsf{If}^{5}_{5,\langle1,2,3,4\rangle}   \big[
          \mathsf{If}^{4}_{4,\langle1,2,3\rangle}  [\mathsf{SameSign}, \rprSucc^3_1,  \rprSucc^3_1 ]
            \pmb,\mathsf{If}^{4}_{4,\langle1,2,3\rangle}   [ \rprId^3 , \rprId^3 ,    \rprSucc^3_1   ] 
            \pmb,\mathsf{If}^{4}_{4,\langle1,2,3\rangle} [\rprId^3 ,   \rprId^3  ,\mathsf{SameSign}]\big]$$
and $\quad \mathsf{SameSign} := \ \rprDec_{2;3}
                           \rprSeq \mathsf{If}^{3}_{3,\langle1,2\rangle}[\rprSucc^2_1,\rprId^2,\rprId^2]
                           \rprSeq  \rprInv{(\rprDec_{2;3})}\quad$.
% \begin{align}
%     \mathsf{F}     := &\
%     \mathsf{If}^{5}_{5,\langle1,2,3,4\rangle}   [
%           \mathsf{If}^{4}_{4,\langle1,2,3\rangle}  [\mathsf{SameSign}, \rprSucc^3_1,  \rprSucc^3_1 ]  \\
%     \nonumber   & \phantom{\mathsf{If}^{5}_{5,\langle1,2,3,4\rangle}\ }
%     ,\mathsf{If}^{4}_{4,\langle1,2,3\rangle}   [ \rprId^3 , \rprId^3 ,    \rprSucc^3_1   ]  \\
%     \nonumber   &\phantom{\mathsf{If}^{5}_{5,\langle1,2,3,4\rangle}\ }
%     ,\mathsf{If}^{4}_{4,\langle1,2,3\rangle} [\rprId^3 ,   \rprId^3  ,\mathsf{SameSign}]] \\[5mm]
%     \nonumber%\mathsf{BothPos}
%     \mathsf{SameSign} := &\ \rprDec_{2;3}
%                            \rprSeq \mathsf{If}^{3}_{3,\langle1,2\rangle}[\rprSucc^2_1,\rprId^2,\rprId^2]
%                            \rprSeq  \rprInv{(\rprDec_{2;3})}
%     % \nonumber
%     % \mathsf{BothNeg} := &\ %\rprInc_{2;3} 
%     %                        \rprDec_{2;3}
%     %                        \rprSeq \mathsf{If}^{3}_{3,\langle1,2\rangle}[\rprSucc^2_1, \rprId^2, \rprId^2 ] 
%     %                        \rprSeq \rprInv{ (\rprDec_{2;3}) }
%     \enspace .
%   \end{align}

\noindent
The \rprSComName at the lines \eqref{less:subeq1} and \eqref{less:subeq3} are one the inverse of the other.
The leftmost function at line \eqref{less:subeq1} moves the five relevant 
arguments in the first five positions\footnote{
The order of the first five elements has been carefully devised in order to avoid index changes of 
arguments due to the removal of intermediate arguments to satisfy the  linearity constraint of 
the $\mathsf{If}$.}.
The rightmost function at line \eqref{less:subeq1} duplicates $x_i$ and $x_j$ into the ancillae
$x_p$ and $x_q$, respectively. The functions at line \eqref{less:subeq3} undo what 
those ones at line \eqref{less:subeq1} have done. In between them,
at line \eqref{less:subeq2} 
the function $ \mathsf{F} \  \rVert \ \rprId^{n-5} $ receives a given 
$\langle x_k,x_p+|x_i|,x_q+|x_j|,x_i,x_j,x_{j_1},\ldots,x_{j_{n-5}}\rangle$ 
as argument and operates on its first five elements.
We explain $ \mathsf{F}\in\RPP^5$ by the following case analysis scheme:
\begin{align*}
\begin{array}{rrcccl||l}
                 &                 & x_i > 0         & x_i = 0     & x_i < 0       &       \\
                 \hline\hline  % &                 &                 &             &                &  &         \\
\mathsf{If}^{5\phantom{|}}_{5,\langle1,2,3,4\rangle} 
      [&\mathsf{If}^{4}_{4,\langle1,2,3\rangle} [&\mathsf{SameSign},&  \rprSucc^3_1,   &   \rprSucc^3_1    &] & x_j > 0 \\
      ,&\mathsf{If}^{4}_{4,\langle1,2,3\rangle} [& \rprId^3,    &  \rprId^3,    &  \rprSucc^3_1   &] & x_j = 0 \\
      ,&\mathsf{If}^{4}_{4,\langle1,2,3\rangle} [&   \rprId^3,  &  \rprId^3,     &\mathsf{SameSign}&]]& x_j < 0 
\end{array}
\enspace .
\end{align*}
For example, 
let us assume $x_j <0$ and $x_j < 0$. The nested selections of $ \mathsf{F} $ eventually apply
$ \mathsf{SameSign} $ to a tuple of three elements $ \langle x_k, x_p+|x_j|, x_q+|x_i|\rangle $.
The effect of  $\rprDec_{2;3} $ is to update the third element of the 
tuple yielding $ \langle x_k, x_p+|x_j|, x_q+|x_i|-x_p-|x_j|\rangle $. So, whenever the initial
value of the temporary arguments $ x_p $ and $ x_q $ is $0$, we can deduce which among
$ x_i $ and $ x_j $ is greater than the other. The value of $ x_k $ is incremented only when
the difference is positive. $ \mathsf{SameSign} $ concludes by unfolding its local subtraction.


%%%%%%%%%%%%%%%
\paragraph{A function that multiplies two integers}
Let $k,j,i$ be pairwise distinct such that $k,j,i\leq n$, for any $ n\in\Nat $.
A function $ \rprMult_{k,j;i}\in\RPP^n  $ exists such that:
%\begin{align*}
%\arraycolsep=1.4pt
%\begin{array}{rcl}
%  \left. {\scriptsize 
%          \begin{array}{r} 
%             x_1\\ \cdots 
%             \\[1.10mm]
%             x_i
%             \\[1.25mm] 
%             \cdots \\
%             x_n
%          \end{array}} 
%  \right[
% & \rprMult_{k,j;i} &
%   \left] {\scriptsize 
%          \begin{array}{l}
%             x_1\\ \cdots 
%           \\[-3mm]
%            x_i +
%             \begin{cases}
%               \underbrace{(\phantom{-} x_j + \ldots + x_j)}_{\mid x_k\mid }
%                      & \textrm{ if } x_k, x_j \textrm{ have same sign}\\
%               \overbrace{(-x_j - \ldots - x_j)}
%                      & \textrm{ if } x_k, x_j \textrm{ have different sign}
%             \end{cases}
%           \\[-3mm]
%           \cdots \\ x_n
%          \end{array} 
%           } 
%   \right.
%\enspace.
%\end{array}
%\end{align*}
\begin{align*}
\arraycolsep=1.4pt
\begin{array}{rcl}
\left. {\scriptsize 
	\begin{array}{r} 
	x_1\\ \cdots 
	\\[1.10mm]
	x_i
	\\[1.25mm] 
	\cdots \\
	x_n
	\end{array}} 
\right[
& \rprMult_{k,j;i} &
\left] {\scriptsize 
	\begin{array}{l}
	x_1\\ \cdots 
	\\[-1mm]
	  (x_i + |x_k| \times |x_j| ) \times
	  \begin{cases}
	  \phantom{-}1  & \textrm{ if } x_k, x_j \textrm{ have same sign}\\
	  -1  &\textrm{ if } x_k, x_j \textrm{ have different sign}
	  \end{cases}
	\\[-1mm]
	\cdots \\ x_n
	\end{array} 
} 
\right.
\enspace.
\end{array}
\end{align*}
The function $ \rprMult_{k,j;i} $ behaves in accordance with the intended meaning whenever 
the ancilla $ x_i$ is initially set to $0$. In that case
$ \rprMult_{k,j;i} $ yields $ |x_j|\times |x_k| $ in position $ i $ which is multiplied by $ -1 $
if, and only if, $ x_j $ and $ x_k $ have different sign. We define:
\begin{align*}
\rprMult_{k,j;i} 
& = 
\rprIt{n}{k,\langle i,j\rangle}{ % \rprInc_{1;2}
	                            \rprInc_{2;1}
                               }
\rprSeq
\rprIf{n}{k,\langle i,j\rangle}
      {\rprIf{2}{1,\langle 2\rangle}
             {\rprId}{\rprId}{\rprNeg}}
      {\rprIf{2}{1,\langle 2\rangle}
             {\rprId}{\rprId}{\rprId}}
      {\rprIf{2}{1,\langle 2\rangle}
             {\rprNeg}{\rprId}{\rprId}} 
\end{align*}
which gives the result by first nesting the iterations of $ \rprInc_{2;1} $ inside an explicit iteration 
(acting on 3 arguments) and then setting the correct sign.

%%%%%%%%%%%%%%%%%%%%%%%%%%%%%%%%%%%%%%%%%%%%%%%%%%%%%%%%%%%%%%%%%%%%%%%%%%%%%%%%%%%%%%%%%%%%%%%%%%%%%%%%%%%%%
\paragraph{A function that encodes a bounded minimization}
Let $\mathsf{F}_{i;j} \in \RPP^n$ with $ i \neq j $ and $ i, j\leq n $. 
For any $\langle x_1,\ldots,x_i,\ldots, x_n\rangle$ and
any $y\in\Nat$, which we call \emph{range}, we look for the minimum integer 
$v$ such that, both $ v\geq 0$  and
$\mathsf{F}_{i;j}\langle x_1,\ldots, x_{i-1}, x_i + v, x_{i+1}, \ldots, x_n\rangle$ 
returns a negative value in position $j$, when a such $ v $ exists.
If $ v $ does not exist we  simply return $ |y| $.          
Summarizing, $ \lpMU{\mathsf{F}_{i;j}}\in\RPP^{n+4} $ is:
\par
{\hspace{-.05\textwidth}
\scalebox{.95}{\parbox{\linewidth}{%
\begin{align*}
    \arraycolsep=1.4pt
    \begin{array}{rcl}
      \left. {\scriptsize 
              \begin{array}{r} 
                 x_1\\ \cdots  \\ x_{n}
                 \\
                 x_{n+1}
                 \\
                 0 \\  0 \\  x_{n+4} 
              \end{array}} 
      \right[
     & \rotatebox{90}{$\hspace{-5mm}\lpMU{\mathsf{F}_{i;j}}$} &
       \left] {\scriptsize 
              \begin{array}{l}
                 x_1\\ \cdots  \\ x_{n}
               \\[-9mm]
                x_{n+1}
                + 
         \begin{cases}
           v
           & \textrm{whenever } 0\leq v\! <\!|x_{n+4}| \textrm{ is such that }\\
           &  \mathsf{F}_{i;j}\langle \ldots,x_{i-1},x_i + v,x_{i+1},\ldots \rangle 
                              = \langle \ldots,y_{j-1},z' ,y_{j+1},\ldots \rangle  \textrm{ and } z' < 0 \\
           & \textrm{and, for all } u \textrm{ such that } 0 \!\leq \! u\! < \! v ,\\
           &  \hfill \mathsf{F}_{i;j} \langle \ldots,x_{i-1},x_i +u,x_{i+1},\ldots \rangle
                              = \langle \ldots,y_{j-1},z,y_{j+1},\ldots \rangle \textrm{ and } z \geq 0; \\[3mm]
           | x_{n+4} |
           & \textrm{otherwise.}\\
         \end{cases}
               \\[-9mm]
                  0 \\  0 \\  x_{n+4} 
              \end{array}
               } 
       \right.
    \end{array}
    \end{align*}}}}
\par\noindent
The function $ \lpMU{\mathsf{F}_{i;j}} $ behaves in accordance with the here above specification when:
(i) the arguments $x_{n+1},x_{n+2}$ and $ x_{n+3} $, that we use as temporary arguments, are set to $ 0 $, and
(ii) $x_{n+4}$ is set to contain the range $ y $.
We plan to use the ancilla $x_{n+3}$ as flag,  the ancilla $x_{n+2}$ as a counter from $0$ to $x_{n+4}$ and,
the ancilla $x_{n+1}$ as store for $v$ (or $0$).
We define: % $\lpMU{\mathsf{F}_{i;j}}$ as follows:
     $$\begin{array}{l}
    \lpMU{\mathsf{F}_{i;j}} :=\\[3mm]
    \qquad
     \rprIt{}
           {}
           {  (\rprPCom{\mathsf{F}_{i;j}}{ \rprId^{3}}) \rprSeq
             \mathsf{If}^{n+3}
                   _{j\!,\scalebox{.59}{$\langle n\!+\!1\!,\!n\!+\!2\!,n\!+\!3\rangle$}\!}
              \big[{\rprId^3\!},
                   {\rprId^3\!},
                   {(\rprIf{}
                          {}
                          {\rprId^2\!}
                          {\rprInc_{2;1}}
                          {\rprId^2\!}
                    \rprSeq
                    \rprSucc_3^{3})
                   }\big] \rprSeq
             \rprInv{(\rprPCom{\mathsf{F}_{i;j}}{ \rprId^{3}})} \rprSeq
             (\rprSucc_i^{n} \rVert % \rprSucc_{n+2}
                                   \rprSucc^{3}_{2}
             )
             }
   \rprSeq  \\[3mm]
    \qquad
%     &
     \rprIf{n+4}
           {n+3,\langle n+1,n+4\rangle}
           {\rprId}
           {\rprInc_{2;1}\,}
           {\rprId}
  \rprSeq    \\[3mm]
%     & \ 
    \qquad
\rprInv{\left(
   \rprIt{}
           {}
           {  (\rprPCom{\mathsf{F}_{i;j}}{ \rprId^{3}}) \rprSeq
             \mathsf{If}^{n+3}
                   _{j\!,\scalebox{.59}{$\langle n\!+\!1\!,\!n\!+\!2\!,n\!+\!3\rangle$}\!}
              \big[{\rprId^3\!},
                   {\rprId^3\!},
                   { \rprSucc_3^{3})
                   }\big] \rprSeq
             \rprInv{(\rprPCom{\mathsf{F}_{i;j}}{ \rprId^{3}})} \rprSeq
             (\rprSucc_i^{n} \rVert \rprSucc^{3}_{2}
             
             )
             }
  \right)
}
    \enspace .
    %\end{aligned}
    \end{array}$$
 %  \end{minipage}
 % } % scalebox
 % \end{center}


\noindent The first line of the here above definition iterates its whole argument as many times as specified by 
the value of the range in $ x_{n+4} $:
\par
\resizebox{.98\textwidth}{!}{
\hspace*{-5mm}\begin{minipage}{\textwidth}
\begin{align*}
&
(\rprPCom{\mathsf{F}_{i;j}}{ \rprId^{3}}) \rprSeq
\mathsf{If}^{n+3}
_{j\!,\scalebox{.59}{$\langle n\!+\!1\!,\!n\!+\!2\!,n\!+\!3\rangle$}\!}
\big[{\rprId^3\!},
{\rprId^3\!},
{(\rprIf{}
	{}
	{\rprId^2\!}
	{\rprInc_{2;1}}
	{\rprId^2\!}
	\rprSeq
	\rprSucc_3^{3})
}\big] \rprSeq
\rprInv{(\rprPCom{\mathsf{F}_{i;j}}{ \rprId^{3}})} \rprSeq
(\rprSucc_i^{n} \rVert % \rprSucc_{n+2}
\rprSucc^{3}_{2}
) & 1\textrm{-st step,} \\
& \vdots \\
&
\rprSeq(\rprPCom{\mathsf{F}_{i;j}}{ \rprId^{3}}) \rprSeq
\mathsf{If}^{n+3}
_{j\!,\scalebox{.59}{$\langle n\!+\!1\!,\!n\!+\!2\!,n\!+\!3\rangle$}\!}
\big[{\rprId^3\!},
{\rprId^3\!},
{(\rprIf{}
	{}
	{\rprId^2\!}
	{\rprInc_{2;1}}
	{\rprId^2\!}
	\rprSeq
	\rprSucc_3^{3})
}\big] \rprSeq
\rprInv{(\rprPCom{\mathsf{F}_{i;j}}{ \rprId^{3}})} \rprSeq
(\rprSucc_i^{n} \rVert % \rprSucc_{n+2}
\rprSucc^{3}_{2}
)  & k\textrm{-th step,} \\
& \vdots \\
&
\rprSeq(\rprPCom{\mathsf{F}_{i;j}}{ \rprId^{3}}) \rprSeq
\mathsf{If}^{n+3}
_{j\!,\scalebox{.59}{$\langle n\!+\!1\!,\!n\!+\!2\!,n\!+\!3\rangle$}\!}
\big[{\rprId^3\!},
{\rprId^3\!},
{(\rprIf{}
	{}
	{\rprId^2\!}
	{\rprInc_{2;1}}
	{\rprId^2\!}
	\rprSeq
	\rprSucc_3^{3})
}\big] \rprSeq
\rprInv{(\rprPCom{\mathsf{F}_{i;j}}{ \rprId^{3}})} \rprSeq
(\rprSucc_i^{n} \rVert % \rprSucc_{n+2}
\rprSucc^{3}_{2}
)  & x_{n+4}\textrm{-th step;} \\
\end{align*}
\end{minipage}
} %resizebox
\par\noindent
let us assume that the $ x_j $ is negative at step $ k$.
We have two cases. 
\begin{itemize}
	\item 
The first case is with $ x_{n+3} $ equal to 0.
By design, this means that $ x_j $ has never become negative before. 
So, $ x_i $ contains the value $ v $ we are looking for. The firs step of:
\begin{align}
\label{align:min inner if}
\rprIf{}{}{\rprId^2\!}{\rprInc_{2;1}}{\rprId^2\!}\rprSeq\rprSucc_3^{3}
\end{align}
is to store $ x_{n+2} $ into $ x_{n+1} $. 
The reason is that, by design, the increments applied to  $ x_{n+2} $  and  $ x_i $ are   always stepwise aligned.
The second step of~\eqref{align:min inner if} is to increment $ x_{n+3} $, preventing
any further change of the value of $ x_{n+1} $ by the subsequent steps of the iteration.
	\item 
The second case is with $ x_{n+3} $ different from 0.
By design, this means that $ x_j $ has become negative in some step before the current $ k $-th one.
So, \eqref{align:min inner if} skips $ 	\rprInc_{2;1} $. The increment $ \rprSucc^{3}_{2} $
that operates on $ x_{n+3} $ is harmless because it just reinforces the idea that the value
$ v $ we were looking for is ``frozen'' in $ x_{n+1} $.
\end{itemize}
After any occurrence of~\eqref{align:min inner if} it is necessary to unfold the last application of 
$\mathsf{F}_{i;j}$ by means of $ \rprInv{\mathsf{F}_{i;j}} $.
Then, increasing $ x_i $ and $ x_{n+2} $ supplies a new value to $\mathsf{F}_{i;j}$ 
and keeps $ x_{n+2} $ aligned with $ x_i $.

\bigskip

After the first iteration completes, the evaluation follows with the second line
$\quad %\begin{align*}
\rprIf{n+4}
{n+3,\langle n+1,n+4\rangle}
{\rprId}
{\rprInc_{2;1}\,}
{\rprId}
\rprSeq
\quad$ %\end{align*}
in the definition of $ \lpMU{\mathsf{F}_{i;j}} $. It takes care of two cases.
\begin{itemize}
	\item 
	The value of $ x_{n+3} $ is not zero: somewhere in the course of the iteration
    $\mathsf{F}_{i;j}$ became negative. We have set $ x_{n+1} $ in accordance with that.	
	\item 
	The value of $ x_{n+3} $ is zero: $\mathsf{F}_{i;j}$  became negative in the course of the iteration.
	By definition, $ \rprInc_{2;1} $ sets $ x_{n+1} $ to the value of the range in $ x_{n+4} $.
\end{itemize}
The last line of $ \lpMU{\mathsf{F}_{i;j}} $ undoes what the first line did,
without altering the values of $ x_{n+1} $. This is why 
$ \rprIf{}{}{\rprId^2\!}{\rprInc_{2;1}}{\rprId^2\!} $ is missing.
In fact, the last line of $ \lpMU{\mathsf{F}_{i;j}} $
is an application of Bennett's trick \cite{bennett73ibm} in our functional programming setting.
 % \LR

%%%\par
%%%We conclude by formalizing a notation for a simple generalization of $ \lpMU{\mathsf{F}_{i;j}} $. 
%%%Let $ k,r \leq m \in\Nat $, with $ m \geq n+4 $ and $ r\neq k $.
%%%By definition $ \rprBMu{r}{k}{\mathsf{F}}{i}{j}\in\RPP^{m}$ is:
%%%    \begin{align*}
%%%    \arraycolsep=1.4pt
%%%    \begin{array}{rcl}
%%%      \left. {\scriptsize 
%%%              \begin{array}{r} 
%%%                 x_1\\ \cdots \\[6.5mm] x_{k} \\[7.5mm] \cdots  \\ x_{m}
%%%              \end{array}} 
%%%      \right[
%%%     & \rotatebox{90}{$\hspace{-5mm}\rprBMu{r}{k}{\mathsf{F}}{i}{j}$} &
%%%       \left] {\scriptsize 
%%%              \begin{array}{l}
%%%                 x_1\\ \cdots
%%%                    \\[-5mm] x_{k}
%%%                + 
%%%                \begin{cases}
%%%                  v
%%%                  & \textrm{whenever } 0\leq v\! <\!|x_{r}| \textrm{ is such that}\\
%%%                  &  \mathsf{F}_{i;j}\langle \ldots,x_{i-1},x_i + v,x_{i+1},\ldots \rangle 
%%%                                     = \langle \ldots,y_{j-1},z' ,y_{j+1},\ldots \rangle 
%%%                  \textrm{ and } z' < 0  \\
%%%                  & \textrm{and, for all } u \textrm{ such that } 0 \!\leq \! u\! < \! v ,\\
%%%                  &   \mathsf{F}_{i;j} \langle \ldots,x_{i-1},x_i + u,x_{i+1},\ldots \rangle
%%%                                     = \langle \ldots,y_{j-1},z,y_{j+1},\ldots \rangle
%%%                                     \textrm{ and } z \geq 0; \\\\\\
%%%                  | x_{r} |
%%%                  & \textrm{otherwise \enspace .}
%%%                \end{cases}
%%%               \\[-2mm] \cdots \\
%%%               x_{m}
%%%              \end{array}
%%%               } 
%%%       \right.
%%%    \end{array}
%%%    \end{align*}
%%%In it, the value $ v $ needs not to be stored in the argument just after the $ n $ arguments of $ \mathsf{F}_{i;j} $.
%%%Any $ x_k $ can contain it. Moreover, the argument that contains the range is not necessary the last one, as in the definition
%%%of $ \mathsf{F}_{i;j} $.

%%%%%%%%\paragraph{Replication trees in $ \RPP $}
%%%%%%%%For every $ n\geq 1 $ and $ x $, a function $ \rprNabla^n\in\RPP^{n+1}$ exists such that:
%%%%%%%%\begin{align*}
%%%%%%%%\arraycolsep=1.4pt
%%%%%%%%\begin{array}{rcl}
%%%%%%%% \left. {\scriptsize \begin{array}{r} 
%%%%%%%%                       x_1 \\ \ldots \\ x_n  \\ x
%%%%%%%%                     \end{array}} \right[
%%%%%%%% & \rprNabla^n &
%%%%%%%% \left] {\scriptsize \begin{array}{l}
%%%%%%%%                       x_1 + x\\ \ldots \\ x_n + x \\ x
%%%%%%%%                     \end{array} } \right.
%%%%%%%% \enspace .
%%%%%%%%\end{array}
%%%%%%%%\end{align*}
%%%%%%%%With $ x_1,\ldots,x_n $ set to $ 0 $, then $ \rprNabla^n \la x_1,\ldots,x_n\ra $ yields
%%%%%%%%$ n $ replicas of the value $ x $. By definition:
%%%%%%%%\begin{align*}
%%%%%%%%\rprNabla^n & =
%%%%%%%%\begin{cases}
%%%%%%%%\rprInc^{2}_{2;1}
%%%%%%%%& \text{if } n = 1
%%%%%%%%\\\\
%%%%%%%%(\rprPCom{\rprInc^{2}_{2;1}}
%%%%%%%%         {\rprId^{n-1}}
%%%%%%%%) \rprSeq \cdots
%%%%%%%%\\
%%%%%%%%\cdots \rprSeq
%%%%%%%%(\rprPCom{\rprId^{i}}
%%%%%%%%         {\rprPCom{\rprInc^{2}_{2;1}}
%%%%%%%%                  {\rprId^{n-i}}}
%%%%%%%%) \rprSeq \cdots 
%%%%%%%%\\
%%%%%%%%\qquad \qquad 
%%%%%%%%\cdots \rprSeq
%%%%%%%%(\rprPCom{\rprId^{n-1}}
%%%%%%%%         {\rprInc^{2}_{2;1}}
%%%%%%%%) 
%%%%%%%%& \text{if } n \geq 2, \text{for every } 1 \leq i < n
%%%%%%%%\enspace .
%%%%%%%%\end{cases}
%%%%%%%%\end{align*}




%%%%%%%%%%%%%%%%%%%%%%%%% LucaP:emacs-configuration
%%% Local Variables:
%%% mode: latex
%%% TeX-master: "main.tex"
%%% ispell-local-dictionary: "american"
%%% End:
%%%%%%%%%%%%%%%%%%%%%%%%% LucaP:emacs-configuration
\section{Cantor pairing functions}
\label{section:Cantor pairing}
Pairing functions provide a mechanism to uniquely encode two natural numbers into a 
single natural number \cite{rosenberg2009book}. 
We show how to represent Cantor pairing functions as functions of $\RPP$ 
restricted on natural numbers,
albeit it is possible to re-formulate the pairing function on integers (see \cite{PaoliniPiccoloRoversiICTCS2015}.)

\begin{definition}\label{grishpan}
Cantor pairing is a pair of isomorphisms $ \CP: \Nat^2 \longrightarrow \Nat $ and 
$ \CU: \Nat \longrightarrow \Nat^2 $ which embed $ \Nat^2 $ into $ \Nat$:
\begin{align}
\label{align: CP definition}
\CP(x,y) = \, & x + \sum_{i=0}^{x+y} i\\
\nonumber
\CU(z) =\, &  \left\langle z - \sum_{i=0}^{k - 1} i\;\pmb,\;(k - 1) 
                          - (z - \sum_{i=0}^{k - 1} i)
                          \right\rangle 
         % \text{ such that }
         %  \left\{ \begin{array}{l}
         %    x = z - (\sum_{i=0}^{k - 1} i),\\
         % y = (k - 1) - x
         %  \end{array} \right.
\ \  \begin{minipage}{.4\linewidth}
 where $k$  is the least value s.t. 
$k \leq z$ and  $z < \sum_{i=0}^{k} i$.
  \end{minipage}
\end{align}
%\todo{\texttt{http://www.math.drexel.edu/~tolya/cantorpairing.pdf}}
\end{definition}

The pairing functions in the above Equation~\eqref{align: CP definition} rely on the notion of triangular number $T_n= \sum_{i=0}^{n} i$, for any $ n\in\Nat $.

%%%%%%%%%%%%%%%%%%%%%%%%%%%%%%%%%%%%%%
\begin{lemma}
\label{lemma:triangular number}
For every $ x_1, x_2 $ and $ x_3 \in\mathbb{N}$, the two following functions exist:
$$\arraycolsep=1.4pt
\begin{array}{rcl}
\left. {\scriptsize \begin{array}{r} x_1\\ x_2 \end{array}} \right[
& \operatorname{T2} &
\left] {\scriptsize \begin{array}{l} x_1 + 1\\ x_2+x_1+1 \end{array} } \right.
\end{array}\qquad
\arraycolsep=1.4pt
\begin{array}{rcl}
\left. {\scriptsize \begin{array}{r} x_1\\ x_2\\ x_3 \end{array}} \right[
& \operatorname{T3} &
\left] {\scriptsize \begin{array}{l} x_1\\ x_2 + x_1\\ x_3 + x_2 x_1 + \sum^{x_1}_{i=0} i \end{array} } \right. 
\enspace .
\end{array}$$
and belong to $ \RPP^2 $ and $ \RPP^3 $, respectively.
\end{lemma}
\begin{proof}
Let $\operatorname{T2} $ be $\rprSCom{\rprSucc^2_1} {\rprInc^2_{1;2}}$
and let $\operatorname{T3}\in\RPP^3 $ be $\rprIt{3}{1,\langle 2,3\rangle}{\operatorname{T2}}$.
The result is immediate because we are interested in the behaviour of the operator only on positive numbers,
viz.  $ x_1, x_2 , x_3 \in\mathbb{N}$.
\end{proof}

We shall systematically neglect to specify the relational behavior of
the coming definitions of functions on negative input values.
 
%%%%%%%%%%%%%%%%%%%%%%%%%%%%%%%%%%%%%%%++++++++++++++++++++++++
\begin{theorem}%[$ \CP: \Nat^2 \longrightarrow \Nat $ and Triangular numbers in $ \RPP $]
\label{theorem:CP in RPP}
For every $x\in\mathbb{N}$, the following function is in $\RPP^4$:
\begin{align*}
%\arraycolsep=1.4pt
%\begin{array}{rcl}
% \left. {\scriptsize \begin{array}{r} x\\ 0\\[0.75mm] 0 \end{array}} \right[
% & \operatorname{Tn} &
% \left] {\scriptsize \begin{array}{l}
%                       x\\ 0\\[0.75mm] (\sum^{x}_{i=0} i) \\
%                     \end{array} } \right.
%\end{array}
%&\textrm{ and }&
\arraycolsep=1.4pt
\begin{array}{rcl}
 \left. {\scriptsize \begin{array}{r} x\\ y\\[1mm] 0\\[1mm] 0  \end{array}} \right[
 & \CP &
 \left] {\scriptsize \begin{array}{l}
                       x\\ y\\[1mm]  (\sum^{x+y}_{i=0} i ) + x\\[0.75mm] 0
                     \end{array} } \right.
\end{array}
\enspace .
\end{align*}
\end{theorem}
\begin{prf}
Let $\CP$ be 
$\rprInc^{4}_{1;2} \rprSeq\rprInc^{4}_{1;4} \rprSeq \rprSwap{4}{1,2,3,4}{2,3,1,4} 
  \rprSeq (\rprPCom{\operatorname{T3}}{\rprId}) \rprSeq \rprDec^4_{1;2} \rprSeq 
  \rprSwap{4}{1,2,3,4}{4,1,3,2} \rprSeq \rprDec^4_{1;2}$.
\qed
\end{prf}

We now turn to representing $ \CU $. The computation of the difference between a given $ z $ and the triangular number
	$ \sum_{i=0}^{k} i $, limited by $ z $, is at the core of $ \CU $.

%%%% %% % \href{http://www.math.drexel.edu/~tolya/cantorpairing.pdf}{http://www.math.drexel.edu/~tolya/cantorpairing.pdf} 

\begin{lemma}[Subtracting a triangular number]
\label{lemma:The kernel kCU  to define CU}
    For every $ x_3 $ and $ x_4 \in\mathbb{N}$, the following function exists:
    \begin{align*}
    \arraycolsep=1.4pt
    \begin{array}{rcl}
    \left. {\scriptsize 
    	\begin{array}{r} 
    	0\\ 0\\ x_3\\ x_4\\ 
    	\end{array}} 
    \right[
    & H_{3;4} &
    \left] {\scriptsize 
    	\begin{array}{l}
    	0\\ 0\\ x_3\\ x_4 - \sum_{i=0}^{x_3} i
    	\end{array} } \right.
    \end{array}
    \end{align*}
	and belongs to $ \RPP^4 $.
\end{lemma}
\begin{prf}
Let $ H_{3;4} $ be:
	\begin{align*}
	& \left (\rprSwap{4}{1,2,3,4}{3,2,1,4} 
	\rprSeq (\rprPCom{\operatorname{T3}}{\rprId})\right )
	\rprSeq \rprDec^{4}_{3;4}
	\rprSeq \rprInv{\left (\rprSwap{4}{1,2,3,4}{3,2,1,4}
		    \rprSeq (\rprPCom{\operatorname{T3}}{\rprId})\right )}
	\enspace .
	\end{align*}
By Lemma~\ref{lemma:triangular number} the proof is done.\qed
\end{prf}

We can supply $ H_{3;4} $ to the minimization $ \lpMUNoArgs $ in order for it to yield a series of ``attempts'' 
whose goal is to find when $ H_{3;4} $ becomes negative in its 4th output:
\begin{center}\vspace{-5mm}
\resizebox{\textwidth}{!}{
\begin{tabular}{ccccc}
1-st attempt & 2-nd attempt &  & $ j $-th attempt & \\
$\arraycolsep=1.4pt
\begin{array}{rcl}
\left. {\scriptsize 
	\begin{array}{r} 
	0\\ 0\\ 0\\ z\\ 
	\end{array}} 
\right[
& H_{3;4} &
\left] {\scriptsize 
	\begin{array}{l}
	0\\ 0\\ 0\\ z - \sum_{i=0}^{0} i
	\end{array} } \right.
\end{array}$
&
$\arraycolsep=1.4pt
\begin{array}{rcl}
\left. {\scriptsize 
	\begin{array}{r} 
	0\\ 0\\ 1\\ z\\ 
	\end{array}} 
\right[
& H_{3;4} &
\left] {\scriptsize 
	\begin{array}{l}
	0\\ 0\\ 1\\ z - \sum_{i=0}^{1} i
	\end{array} } \right.
\end{array}$
&
$ \cdots $
&
$\arraycolsep=1.4pt
\begin{array}{rcl}
\left. {\scriptsize 
	\begin{array}{r} 
	0\\ 0\\ j\\ z\\ 
	\end{array}} 
\right[
& H_{3;4} &
\left] {\scriptsize 
	\begin{array}{l}
	0\\ 0\\ j\\ z - \sum_{i=0}^{j} i
	\end{array} } \right.
\end{array}$
&
$ \cdots $
\end{tabular}
} % resizebox
\end{center}

\noindent
The representation of $ \CU $ relies on the search here above.

%%%%%%%%%%%%%%%%%%%%%%%%%%%%%%%%%%%%++++++++++++++++++++++++++++++++++++++++++++++++++
\begin{theorem}[Representing $ \CU: \Nat^2 \longrightarrow \Nat $ in $ \RPP $]
\label{lemma:Representing CU in RPP}
A function $ \CU\in\RPP^{8}$ exists such that, for every $ z \in\mathbb{N}$:
\begin{align*}
\arraycolsep=1.4pt
\begin{array}{rcl}
 \left. {\scriptsize \begin{array}{r} 
                       z\\[0.5mm] 0\\[0.5mm] 0\\[0.5mm] 
                        0^{5
                       	}
                     \end{array}} \right[
 & \CU &
 \left] {\scriptsize \begin{array}{l}
                       z\\ z-(\sum^{v-1}_{i=0} i ) \\ (v-1)-(z-(\sum^{v-1}_{i=0} i)) \\[0.5mm]
                       0^{5
                         }
                     \enspace ,
                     \end{array} } \right.
\end{array}
\end{align*}
where $ v $ is the least value such that $ v \leq z $ and $ z-(\sum^{v}_{i=0} i ) < 0$. 
So $ z-(\sum^{v-1}_{i=0} i ) $ is the first component of the pair that $ z $ represents under Cantor pairing
and $ (v-1)-(z-(\sum^{v-1}_{i=0} i)) $ is the second one.
\end{theorem}
\begin{prf}
Let $ \CU $  be:
$$
\begin{array}{l}
\rprSwap{8}
{1, 2, 3, 4 ,5 ,6 ,7 ,8}
{4, 2, 3, 1 ,5 ,6 ,7 ,8}
\,\rprSeq \, \rprInc^{8}_{4;8}  \,\rprSeq \lpMU{H_{3;4}} \\
\qquad \qquad 
\,\rprSeq \rprPred^{8}_{5}\,\rprSeq
\rprSwap{8}
{1, 2, 3, 4 ,5 ,6 ,7 ,8}
{1, 2, 5, 4 ,3 ,6 ,7 ,8}
\rprSeq \,( H_{3;4} \rVert \rprId^4) \,\rprSeq \, \rprDec^{8}_{4;3}
\,\rprSeq
\rprSwap{8}
{1, 2, 3, 4 ,5 ,6 ,7 ,8}
{8, 4, 3, 2 ,5 ,6 ,7 ,1}
\end{array}
$$
The \rprSComName $ \rprSwap{8}
{1, 2, 3, 4 ,5 ,6 ,7 ,8}
{4, 2, 3, 1 ,5 ,6 ,7 ,8}
\,\rprSeq \, \rprInc^{8}_{4;8} $ sets the input for $\lpMU{H_{3;4}}$ which receives the tuple
$\langle 0,0,0,z,0,0,0,z \rangle$ and yields $\langle 0,0,0,z,k,0,0,z \rangle$
where $k$ is the least value such that $k \leq z$ and  $z < \sum_{i=0}^{k} i$.
The predecessor $ \rprPred^{8}_{5}$ sets its 5th argument to the value $k-1$ which is required
to correctly apply $ \CU $ (see Definition \ref{grishpan}).
The next rewiring sets the tuple $\langle 0,0,k-1,z,0,0,0,z \rangle$, argument of $ H_{3;4} \rVert \rprId^4 $
which produces $\langle 0,0,k-1,z-\sum_{i=0}^{k - 1} i,0,0,0,z \rangle$.
The value $z-\sum_{i=0}^{k - 1} i$ is the first component of the pair of numbers we want to extract from 
the first argument of the whole $ \CU $.
The second component is $k-1-(z-\sum_{i=0}^{k - 1} i)$ we obtain
by applying $\rprDec^{8}_{4;3}$. The last rewiring rearranges the values as required.
\qed
\end{prf}

As a remark,
Theorem~\ref{theorem:CP in RPP},
Lemma~\ref{lemma:The kernel kCU  to define CU} and
Theorem~\ref{lemma:Representing CU in RPP} can be extended to functions that operate on 
$ \Int $, not only on $ \Nat $, suitably managing signs \cite{PaoliniPiccoloRoversiICTCS2015}.

%%%%%%%%%%%%%%%%%%%%%%%%%
%%% Local Variables:
%%% mode: latex
%%% TeX-master: "main.tex"
%%% ispell-local-dictionary: "american"
%%% End:
\section{Expressiveness of $ \RPP $}
\label{section:Completeness of RPP}

We start to show how to represent stacks as natural numbers in $ \RPP $.
\Ie, given $ x_1, x_2, \ldots, x_n  \in \Nat$
we can encode them into $ \la   x_n, \ldots, \la x_2, x_1 \ra \ldots \ra \in \Nat$
by means of a sequence of $ \rprPush$ on a suitable ancillae, while a corresponding
sequence of $ \rprPop $ can decompose it as expected.

%%%%%%%%%%%%%%%
\begin{proposition}[Representing stacks in $ \RPP $]
\label{proposition:Representing stacks in RPP}
Functions $ \rprPush, \rprPop\in\RPP^{10}$ exist such that:
\begin{align*}
\arraycolsep=1.4pt
\begin{array}{rcl}
 \left. {\scriptsize \begin{array}{r} 
                       s\\ x\\ 0^{8}
                     \end{array}} \right[
 & \rprPush &
 \left] {\scriptsize \begin{array}{l}
                       \CP(s,x)\\ 0\\ 0^{8}
                     \end{array} } \right.
\end{array}
&&
\arraycolsep=1.4pt
\begin{array}{rcl}
 \left. {\scriptsize \begin{array}{r} 
                       \CP(s,x)\\ 0\\ 0^{8}
                     \end{array}} \right[
 & \rprPop &
 \left] {\scriptsize \begin{array}{l}
                       s\\ x\\ 0^{8}
                     \end{array} } \right.
\end{array}
\enspace ,
\end{align*}
for every value $ s \in\mathbb{N}$ (the ``stack'') and $ x \in\mathbb{N}$ (the element 
one has to push on or pop out the stack.)
\end{proposition}
\begin{prf}
The function 
$\operatorname{zClean}:= 
(\rprPCom{\rprId^2}{\CU})\, 
 \rprSeq \rprDec^{10}_{4;1}\, 
 \rprSeq \rprDec^{10}_{5;2}\, 
 \rprSeq \rprInv{(\rprPCom{\rprId^2}{\CU})} 
$ 
is such that:
$$\begin{array}{rcl}
 \left. {\scriptsize \begin{array}{r} 
                       s\\ x\\ \CP(s,x)\\ \\[-1mm] 0^{7}
                     \end{array}} \right[
 & \operatorname{zClean} &
 \left] {\scriptsize \begin{array}{l}
                       0\\ 0\\ \CP(s,x)\\  \\[-1mm] 0^{7}
                     \end{array} } \right.
\end{array}$$
so,
$\rprPush : =
(\rprPCom{\CP} {\rprId^{6}})\, 
          \rprSeq \operatorname{zClean}
       \rprSeq (\rprPCom{\rprSwap{3}{1,2,3}{3,2,1}}{\rprId^{6}})$ 
and 
$\rprPop  := \rprInv{\rprPush}$.
\qed
\end{prf}



\subsection{$ \RPP $ is $ \PRF $-complete} $ \RPP $ is expressive enough to represent 
the class $ \PRF $ of Primitive Recursive Functions \cite{cutland1980book,odifreddi1989book}, which we recall for easy 
of reference. $ \PRF $ is the smallest class of functions on natural numbers that:
\begin{itemize}
\item 
 contains the functions $\prZero^n(x_1,\ldots,x_n):=0$, the successors 
$\prSucc^n_i(x_1,\ldots,x_n) := x_i + 1$  and  the projections 
$ \prProj{n}{i}(x_1,\ldots,x_n) := x_i $ for all $1\leq i\leq n$;

\item 
 is closed under composition, viz. $ \PRF $  includes the function $f(\myVec{x}) := h(g_1(\myVec{x}), \ldots , g_m(\myVec{x}))$)
whenever there are  $g_1, \ldots, g_m, h\in\PRF$ of suitable arity; and,

\item 
 is closed under primitive recursion, viz. $ \PRF $ includes the function $f$ defined   by means of the schema $f(\myVec{x},0):= g(\myVec{x})$ and 
$f(\myVec{x},y+1):= h(f(\myVec{x},y),\myVec{x},y)$ whenever there are  $g,h\in\PRF$  of suitable arity.
\end{itemize}


In the following, $\PRF^n$ denotes the class of functions $f\in \PRF$ with arity $n$.

%%%%%%%%%%%%%
\begin{definition}[$ \RPP $-definability of any $ f\in\PRF $]
\label{definition:RPP-definability of any f in PRF}
Let $n, a\in\mathbb N$ and $ f\in\PRF^n $. We say that $ f $ is $ \RPP^{n+a+1} $-definable  whenever
there is a function  $ d_f\in\RPP^{n+a+1} $  such that, for every $x, x_1, \ldots, x_n\in\Nat$:
\begin{align*}
\arraycolsep=1.4pt
\begin{array}{rcl}
 \left. {\scriptsize 
            \begin{array}{r}
             x \\ x_1 \\ \cdots \\ x_n
             \\[1mm]
              0^a
            \end{array} 
         } \right[
 & d_f &
 \left] {\scriptsize 
            \begin{array}{l}
             x + f (x_1, \ldots, x_n) \\ x_1 \\ \cdots \\ x_n
             \\[1mm]
              0^a
            \enspace .
            \end{array} 
        } \right.
\end{array}
\end{align*}
\end{definition}

Three observations are worth doing.
Definition~\ref{definition:RPP-definability of any f in PRF} relies on $a+1$ arguments used as ancillae.
If $ f\in\PRF^n $ is $ \RPP^{n+a+1} $-definable, then $f$ is also
$ \RPP^{n+k} $-definable for any $k\geq a+1$ by using \rprWeaName (cf. Section \ref{section:Generalizations inside RPP}).
Definition \ref{definition:RPP-definability of any f in PRF} improves 
the namesake one in \cite[Def.3.1, p.236]{PaoliniPiccoloRoversiICTCS2015} because it gets rid of an output
line:  this line was a ``waste bin'' containing part of the computation trace that, eventually, was useless. 
The encoding proposed in this paper does not need the ``waste bin'' anymore.
%%%%%%%%%%%%%%%%%%%%%%%%%%%%%%%%%%%%%%%%%%%%%%%%%%%%%%%%%%%%%%%%%%%
%%%%%%%%%%%%%%%%%%%%%%%%%%%%%%%%%%%%%%%%%%%%%%%%%%%%%%%%%%%%%%%%%%%
%%%%%%%%%%%%%%%%%%%%%%%%%%%%%%%%%%%%%%%%%%%%%%%%%%%%%%%%%%%%%%%%%%%
%%%%%%%%%%%%%   VECCHIA COMPLETEZZA INIZIO
%%%%%%%%%%%%%\begin{definition}[Embedding $ \prrpr{(\cdot)} $ of $ \PR $ into $ \RPP $]
%%%%%%%%%%%%%The function $\prrpr{(\cdot)}: \PR \longrightarrow \RPP $ is inductively defined
%%%%%%%%%%%%%on the structure of its argument. 
%%%%%%%%%%%%%\begin{itemize}
%%%%%%%%%%%%%\item 
%%%%%%%%%%%%%The basic cases are as follows:  
%%%%%%%%%%%%%\begin{align*}
%%%%%%%%%%%%%%%%%%%% 
%%%%%%%%%%%%%\prrpr{\prZero^n}
%%%%%%%%%%%%%& = \rprId^2
%%%%%%%%%%%%%\\
%%%%%%%%%%%%%%%%%%%% 
%%%%%%%%%%%%%\prrpr{\prSucc}
%%%%%%%%%%%%%& = (\rprPCom{\rprId^1}
%%%%%%%%%%%%%            {\rprSucc}) \rprSeq \rprInc^2_{2;1} \rprSeq \rprInv{(\rprPCom{\rprId^1}
%%%%%%%%%%%%%                                                                     {\rprSucc})}
%%%%%%%%%%%%%\\
%%%%%%%%%%%%%\prrpr{(\prProj{k}{i})}
%%%%%%%%%%%%%& = \rprInc^{k+1}_{i;1} 
%%%%%%%%%%%%%\enspace
%%%%%%%%%%%%%\end{align*}
%%%%%%%%%%%%%where $ \prrpr{\prZero}, \prrpr{\prSucc} \in\RPP^{2} $
%%%%%%%%%%%%%and $ \prrpr{(\prProj{k}{i})} \in\RPP^{k+1} $.
%%%%%%%%%%%%%
%%%%%%%%%%%%%\item 
%%%%%%%%%%%%%%%%%%%%%%%%%%%%%% composition
%%%%%%%%%%%%%Let $ g_1,\ldots,g_k\in\PR^{n} $ and $ h\in\PR^{k} $ in $ \prCom{h,g_1,\ldots,g_k}\in\PR^{n}$.
%%%%%%%%%%%%%By inductive hypothesis and weakening
%%%%%%%%%%%%%we can assume that $ \prrpr{g_1},\ldots,\prrpr{g_k}\in\PR^{n+p} $ and 
%%%%%%%%%%%%%$ \prrpr{h}\in\PR^{k+q} $ exist, for some $ p $ and $ q $ big enough.
%%%%%%%%%%%%%Then $ \prrpr{\prCom{h,g_1,\ldots,g_k}}\in\RPP^{1+n+m} $ is:
%%%%%%%%%%%%%\begin{align*}
%%%%%%%%%%%%%\prrpr{\prCom{h,g_1,\ldots,g_k}}
%%%%%%%%%%%%%& =
%%%%%%%%%%%%%{H}
%%%%%%%%%%%%%\rprSeq
%%%%%%%%%%%%%(\rprPCom{\rprId}{\rprPCom{\prrpr{h}}{\rprId^{(n+p)k}}}) \rprSeq
%%%%%%%%%%%%%(\rprPCom{\rprInc^2_{2;1}}{\rprId^{k+q+(n+p)k}}) \rprSeq
%%%%%%%%%%%%%\rprInv{H}
%%%%%%%%%%%%%\enspace ,        
%%%%%%%%%%%%%\end{align*}
%%%%%%%%%%%%%where:
%%%%%%%%%%%%%\begin{align*}
%%%%%%%%%%%%%H
%%%%%%%%%%%%%& =
%%%%%%%%%%%%%(\rprPCom{\rprId}{\rprPCom{\rprGSwap{nk}_1}{\rprPCom{\rprId^{(p+1)k}}{\rprId^{1+q}}}})\rprSeq
%%%%%%%%%%%%%(\rprPCom{\rprId}
%%%%%%%%%%%%%         {\rprPCom{\overbrace{\rprPCom{\rprNabla^{k-1}}{\rprPCom{\cdots}{\rprNabla^{k-1}}}}^{n}}
%%%%%%%%%%%%%                  {\rprPCom{\rprId^{(p+1)k}}
%%%%%%%%%%%%%                           {\rprId^{1+q}}}})\rprSeq
%%%%%%%%%%%%%\\ & \phantom{= }\ \
%%%%%%%%%%%%%(\rprPCom{\rprId}{\rprPCom{\rprGSwap{(n+p+1)k}_2}{\rprId^{1+q}}})\rprSeq
%%%%%%%%%%%%%(\rprPCom{\rprId}
%%%%%%%%%%%%%         {\rprPCom{\rprPCom{\prrpr{g_1}}{\rprPCom{\cdots}{\prrpr{g_k}}}}
%%%%%%%%%%%%%                  {\rprId^{1+q}}})\rprSeq
%%%%%%%%%%%%%\\ & \phantom{= }\ \
%%%%%%%%%%%%%(\rprPCom{\rprId}{\rprGSwap{(n+p+1)k+1+q}_3})\rprSeq
%%%%%%%%%%%%%(\rprPCom{\rprId}{\rprPCom{\prrpr{h}}{\rprId^{(n+p)k}}})
%%%%%%%%%%%%%\\
%%%%%%%%%%%%%\rprGSwap{}_1
%%%%%%%%%%%%%& =
%%%%%%%%%%%%%\rprSwap
%%%%%%%%%%%%%{nk}
%%%%%%%%%%%%%{1,2 ,\ldots,n ,n+1,\ldots,n+k-1,\ldots,n+(n-1)(k-1)+1,\ldots,nk}
%%%%%%%%%%%%%{k,2k,\ldots,nk,1  ,\ldots,k-1  ,\ldots,(n-1)k+1      ,\ldots,nk-1}
%%%%%%%%%%%%%%\\
%%%%%%%%%%%%%%\rprGSwap{}_2
%%%%%%%%%%%%%%& =
%%%%%%%%%%%%%%\rprSwap
%%%%%%%%%%%%%%{(n+1+p)k}
%%%%%%%%%%%%%%{1,\ldots,k,\ldots,(n-1)k+1,\ldots,nk,
%%%%%%%%%%%%%%  nk+1,\ldots,(n+1)k,
%%%%%%%%%%%%%%   (n+1)k+1,\ldots,(n+1)k+p,\ldots,(n+1)k+(k-1)p,\ldots,(n+1)k+kp
%%%%%%%%%%%%%%}
%%%%%%%%%%%%%%{2,\ldots,(k-1)(n+1+p)+2,\ldots,n+1,\ldots,(k-1)(n+1+p)+(n+1),
%%%%%%%%%%%%%%  1,\ldots,(k-1)(n+1+p)+1,
%%%%%%%%%%%%%%   n+1+1,\ldots,n+1+p,\ldots,
%%%%%%%%%%%%%%    (k-1)(n+1+p)+(n+1)+1,\ldots,(k-1)(n+1+p)+(n+1)+p
%%%%%%%%%%%%%%}
%%%%%%%%%%%%%%\\
%%%%%%%%%%%%%%\rprGSwap{}_3
%%%%%%%%%%%%%%& =
%%%%%%%%%%%%%%\rprSwap
%%%%%%%%%%%%%%{(n+p+1)k+1+q}
%%%%%%%%%%%%%%{1,2      ,\ldots,1+n    ,2+n    ,\ldots,1+n+p,
%%%%%%%%%%%%%%  \ldots,
%%%%%%%%%%%%%%   (k-1)(1+n+p)+1,\ldots,k(1+n)+(k-1)p,
%%%%%%%%%%%%%%    k(1+n)+(k-1)p+1      ,\ldots,k(1+n+p),
%%%%%%%%%%%%%%     k(1+n+p)+1,k(1+n+p)+2,\ldots,k(1+n+p)+1+q
%%%%%%%%%%%%%%}
%%%%%%%%%%%%%%{2,1+k+q+1,\ldots,1+k+q+n,2+k+q+n,\ldots,1+k+q+n+p,
%%%%%%%%%%%%%%  \ldots,
%%%%%%%%%%%%%%   1+k           ,\ldots,1+k+q+(k-1)(n+p)+n),
%%%%%%%%%%%%%%    1+k+q+(k-1)(n+p)+n+1),\ldots,1+k+q+(k-1)(n+p)+n+p),
%%%%%%%%%%%%%%     1         ,k+2       ,\ldots,1+k+q
%%%%%%%%%%%%%%}
%%%%%%%%%%%%%\end{align*}
%%%%%%%%%%%%%
%%%%%%%%%%%%%%\begin{align*}
%%%%%%%%%%%%%%\arraycolsep=1.4pt
%%%%%%%%%%%%%%\begin{array}{rcl}
%%%%%%%%%%%%%% \left. {\scriptsize 
%%%%%%%%%%%%%%         \begin{array}{r} 
%%%%%%%%%%%%%%           x_{1}\\ x_{2} \\ \ldots\\ x_{n}\\x_{n+1}\\\ldots\\x_{n+k-1}\\
%%%%%%%%%%%%%%             \ldots\\ \ldots\\ x_{n+(n-1)(k-1)+1} \\ \ldots\\ x_{n+(n-1)(k-1)+(k-1)}
%%%%%%%%%%%%%%         \end{array}} 
%%%%%%%%%%%%%%\right[
%%%%%%%%%%%%%% & \rprGSwap{nk}_1 &
%%%%%%%%%%%%%% \left] {\scriptsize 
%%%%%%%%%%%%%%         \begin{array}{l}
%%%%%%%%%%%%%%           x_{k}\\ x_{2k} \\ \ldots\\ x_{nk}\\x_{1}\\\ldots\\x_{k-1}\\
%%%%%%%%%%%%%%             \ldots\\ \ldots\\ x_{(n-1)k+1} \\ \ldots\\ x_{(n-1)k+(k-1)}
%%%%%%%%%%%%%%         \end{array} } \right.
%%%%%%%%%%%%%%\end{array}
%%%%%%%%%%%%%%\end{align*}
%%%%%%%%%%%%%\begin{align*}
%%%%%%%%%%%%%\arraycolsep=1.4pt
%%%%%%%%%%%%%\begin{array}{rcl}
%%%%%%%%%%%%% \left. {\scriptsize 
%%%%%%%%%%%%%         \begin{array}{r} 
%%%%%%%%%%%%%           x_{1}\\ \ldots\\ x_{k} \\\ldots\ldots\\x_{(n-1)k+1}\\\ldots\\x_{nk}\\
%%%%%%%%%%%%%             x_{nk+1} \\ \ldots\\ x_{(n+1)k}\\ 
%%%%%%%%%%%%%              x_{(n+1)k+1}\\ \ldots\\ x_{(n+1)k+p}\\\ldots\ldots\\
%%%%%%%%%%%%%               x_{(n+1)k+(k-1)p}\\ \ldots\\ x_{(n+1)k+kp}
%%%%%%%%%%%%%         \end{array}} 
%%%%%%%%%%%%%\right[
%%%%%%%%%%%%% & \rprGSwap{(n+1+p)k}_2 &
%%%%%%%%%%%%% \left] {\scriptsize 
%%%%%%%%%%%%%         \begin{array}{l}
%%%%%%%%%%%%%           x_{2} \\ \ldots\\ x_{(k-1)(n+1+p)+2}\\\ldots\ldots\\ x_{n+1}\\\ldots\\ x_{(k-1)(n+1+p)+(n+1)}\\
%%%%%%%%%%%%%             x_{1}\\ \ldots\\ x_{(k-1)(n+1+p)+1}\\ 
%%%%%%%%%%%%%              x_{n+1+1}\\ \ldots\\ x_{n+1+p}\\\ldots\ldots\\
%%%%%%%%%%%%%                x_{(k-1)(n+1+p)+(n+1)+1}\\ \ldots\\ x_{(k-1)(n+1+p)+(n+1)+p}
%%%%%%%%%%%%%         \end{array} } \right.
%%%%%%%%%%%%%\end{array}
%%%%%%%%%%%%%\end{align*}
%%%%%%%%%%%%%\begin{align*}
%%%%%%%%%%%%%\arraycolsep=1.4pt
%%%%%%%%%%%%%\begin{array}{rcl}
%%%%%%%%%%%%% \left. {\scriptsize 
%%%%%%%%%%%%%         \begin{array}{r} 
%%%%%%%%%%%%%           x_{1}\\ x_{2}       \\ \ldots \\x_{1+n}  \\x_{1+n+1}    \\\ldots\\x_{1+n+p}
%%%%%%%%%%%%%             \\ \ldots\\ \ldots\\ 
%%%%%%%%%%%%%           x_{(k-1)(1+n+p)+1}\\ x_{(k-1)(1+n+p)+1+1} \\ \ldots \\ x_{k(1+n)+(k-1)p}\\
%%%%%%%%%%%%%           x_{k(1+n)+(k-1)p+1} \\\ldots\\x_{k(1+n+p)}\\
%%%%%%%%%%%%%           x_{k(1+n+p)+1}\\x_{k(1+n+p)+1+1}\\\ldots\\x_{k(1+n+p)+1+q}
%%%%%%%%%%%%%         \end{array}} 
%%%%%%%%%%%%%\right[
%%%%%%%%%%%%% & \rprGSwap{(n+p+1)k+1+q}_3 &
%%%%%%%%%%%%% \left] {\scriptsize 
%%%%%%%%%%%%%         \begin{array}{l}
%%%%%%%%%%%%%           x_{2}\\ x_{1+k+q+1} \\ \ldots \\x_{1+k+q+n}\\x_{1+k+q+n+1}\\\ldots\\x_{1+k+q+n+p}
%%%%%%%%%%%%%             \\ \ldots\\ \ldots\\ 
%%%%%%%%%%%%%           x_{1+k}          \\ x_{1+k+q+(k-1)(n+p)+1)} \\ \ldots \\ x_{1+k+q+(k-1)(n+p)+n)}\\
%%%%%%%%%%%%%                          x_{1+k+q+(k-1)(n+p)+n+1)}\\\ldots\\x_{1+k+q+(k-1)(n+p)+n+p)}\\
%%%%%%%%%%%%%           x_{1} \\ x_{1+k+1}\\\ldots\\x_{1+k+q}
%%%%%%%%%%%%%         \end{array} } \right.
%%%%%%%%%%%%%\end{array}
%%%%%%%%%%%%%\end{align*}
%%%%%%%%%%%%%\begin{remark}
%%%%%%%%%%%%%Some of the expressions that identify the index of an input or of an output in $ \rprGSwap{(n+1+p)k}_2$
%%%%%%%%%%%%%or $ \rprGSwap{(n+p+1)k+1+q}_3 $ can be simplified. We leave them
%%%%%%%%%%%%%in an explicit form to easy the reconstruction of the pattern for their generation.
%%%%%%%%%%%%%
%%%%%%%%%%%%%The intuition behind the structure of $ \prrpr{\prCom{h,g_1,\ldots,g_k}} $ is the one
%%%%%%%%%%%%%we may expect, up to the required number of \temporarychannels. 
%%%%%%%%%%%%%The point is to supply enough arguments to $ \prrpr{h} $ which are the
%%%%%%%%%%%%%results we get from $ \prrpr{g_1}, \ldots, \prrpr{g_k} $. Every $ \prrpr{g_i} $ needs its 
%%%%%%%%%%%%%own copy of the initial arguments $ x_1,\ldots,x_n $ which we produce by means of
%%%%%%%%%%%%%$ n $ instances of $ \rprNabla^{k-1} $. Every among 
%%%%%%%%%%%%%$ \rprGSwap{nk}_1, \rprGSwap{(n+1+p)k}_2$ and
%%%%%%%%%%%%%$ \rprGSwap{(n+p+1)k+1+q}_3 $ rearranges the inputs as required by the inductive
%%%%%%%%%%%%%hypothesis on the structure of $ \prrpr{g_1}, \ldots, \prrpr{g_k} $ and $ \prrpr{h} $.
%%%%%%%%%%%%%\end{remark}
%%%%%%%%%%%%%%%%%%%%%%%%%%%%%%%
%%%%%%%%%%%%%
%%%%%%%%%%%%%\item 
%%%%%%%%%%%%%%%%%%%%%%%%%%%%%% iteration
%%%%%%%%%%%%%Let $ g\in\PR^{k} $ and $ h\in\PR^{k+2} $ in $ \prRec{g}{h}\in\PR^{k+1}$.
%%%%%%%%%%%%%By inductive hypothesis and weakening
%%%%%%%%%%%%%we can assume that $ \prrpr{g}\in\PR^{1+k+m} $ and $ \prrpr{h}\in\PR^{1+(k+2)+m} $,
%%%%%%%%%%%%%for some $ m \geq 14 $, exist. 
%%%%%%%%%%%%%Then $ \prrpr{\prRec{h}{g}} \in \RPP^{k+m+5} $ is:
%%%%%%%%%%%%%\begin{align*}
%%%%%%%%%%%%%\prrpr{\prRec{h}{g}}
%%%%%%%%%%%%%& =
%%%%%%%%%%%%% \rprGSwap{} \rprSeq
%%%%%%%%%%%%% (\rprPCom{\rprId^4}{G}) \rprSeq
%%%%%%%%%%%%% (\rprPCom{\rprId^2}{\rprIt{4+k+m}{1}{H,Z,\rprId}})\, \rprSeq
%%%%%%%%%%%%%\\ & \phantom{= }\
%%%%%%%%%%%%% \rprInc^{5+k+m}_{5+k;1}\,\rprSeq
%%%%%%%%%%%%%\\ & \phantom{= }\
%%%%%%%%%%%%%  \rprInv{\left(
%%%%%%%%%%%%%           \rprGSwap{} \rprSeq
%%%%%%%%%%%%%           (\rprPCom{\rprId^4}{G}) \rprSeq
%%%%%%%%%%%%%           (\rprPCom{\rprId^2}{\rprIt{4+k+m}{1}{H,Z,\rprId}})
%%%%%%%%%%%%%          \right)}
%%%%%%%%%%%%%\end{align*}
%%%%%%%%%%%%%where:
%%%%%%%%%%%%%\begin{align*}
%%%%%%%%%%%%%\rprGSwap{}
%%%%%%%%%%%%%& =
%%%%%%%%%%%%%\rprSwap{5+k+m}
%%%%%%%%%%%%%        {1,2,3,\ldots,3+k-1,3+k,4+k,5+k,6+k,\ldots,5+k+m}
%%%%%%%%%%%%%        {1,2,6,\ldots,6+k-1,4  ,5  ,3  ,6+k,\ldots,5+k+m}
%%%%%%%%%%%%%\\
%%%%%%%%%%%%%G
%%%%%%%%%%%%%& =
%%%%%%%%%%%%%\rprPCom{\rprId^4}
%%%%%%%%%%%%%        {\prrpr{g}}
%%%%%%%%%%%%%\\
%%%%%%%%%%%%%Z
%%%%%%%%%%%%%& = 
%%%%%%%%%%%%%\rprPCom{\rprId^2}
%%%%%%%%%%%%%        {\rprSwap{3+k+m}
%%%%%%%%%%%%%                 {1,2,3  ,4,\ldots,4+k,5+k,\ldots,3+k+m}
%%%%%%%%%%%%%                 {1,2,3+k,2,\ldots,2+k,5+k,\ldots,3+k+m}}
%%%%%%%%%%%%%\\
%%%%%%%%%%%%%H
%%%%%%%%%%%%%& =
%%%%%%%%%%%%%\left(
%%%%%%%%%%%%%\rprPCom{\rprId^1}
%%%%%%%%%%%%%        {\rprPCom{\rprSucc}
%%%%%%%%%%%%%                 {\rprPCom{\rprSwap{k+1}
%%%%%%%%%%%%%                                   {1  ,2,\ldots,k+1}
%%%%%%%%%%%%%                                   {k+1,1,\ldots,k  }}
%%%%%%%%%%%%%                          {\rprId^{m}}}}\right) 
%%%%%%%%%%%%%\rprSeq 
%%%%%%%%%%%%%\\&\phantom{= } \ \
%%%%%%%%%%%%%\prrpr{h} \rprSeq (\rprPCom{\rprId^{k+2}}{\rprPush})\, \rprSeq 
%%%%%%%%%%%%%\\&\phantom{= } \ \
%%%%%%%%%%%%%\rprSwap{3+k+m}
%%%%%%%%%%%%%        {1  ,2,\ldots,2+k,2+k+1,2+k+2,5+k,\ldots,3+k+m}
%%%%%%%%%%%%%        {3+k,2,\ldots,2+k,2+k+2,1    ,5+k,\ldots,3+k+m}
%%%%%%%%%%%%%\enspace .
%%%%%%%%%%%%%\end{align*}
%%%%%%%%%%%%%%%%%%%%%%%%%%%%%%%
%%%%%%%%%%%%%
%%%%%%%%%%%%%\begin{remark}
%%%%%%%%%%%%%Using $ \rprPush $ and $ \rprPop $ inside $ \prrpr{\prRec{h}{g}} $ requires that we can supply
%%%%%%%%%%%%%at least 14 inputs set to the value $ 0 $ in accordance with 
%%%%%%%%%%%%%Proposition~\ref{proposition:Representing stacks in RPP}. This is why we require $ m\geq 14 $
%%%%%%%%%%%%%which is a condition we can satisfy by using the possibility of \rprWeaName functions inside 
%%%%%%%%%%%%%$ \RPP $.
%%%%%%%%%%%%%
%%%%%%%%%%%%%The intuition behind the structure of $ \prrpr{\prRec{h}{g}} $ is that we use 
%%%%%%%%%%%%%the \emph{second} argument to drive a \rprSComName of the translation of $ h\in\PRF $ which must be preceded 
%%%%%%%%%%%%%by a call to the translation of $ g\in\PRF $. The function $ Z $ serves to correctly re-organize the
%%%%%%%%%%%%%outputs of $ \prrpr{g} $ whenever the \rprSComName is $ 0 $-steps long.
%%%%%%%%%%%%%
%%%%%%%%%%%%%The use of $ \rprPush $ is crucial to pile up in the right order the intermediate values that
%%%%%%%%%%%%%$ \prrpr{h} $ produces and which depend on the length of the \rprSComName. Once got and stored 
%%%%%%%%%%%%%the result, $ \rprPop $ extracts every intermediate value to revert every step of the \rprSComName.
%%%%%%%%%%%%%\qed
%%%%%%%%%%%%%\end{remark}
%%%%%%%%%%%%%\end{itemize}
%%%%%%%%%%%%%\end{definition}
%%%%%%%%%%%%%
%%%%%%%%%%%%%
%%%%%%%%%%%%%%%%%%%%
%%%%%%%%%%%%%\begin{theorem}[$\RPP $ is $ \PRF $-complete]
%%%%%%%%%%%%%\label{theorem:RPP is PRF-complete}
%%%%%%%%%%%%%If $ f\in\PR $, then $ \prrpr{f}$ is $ \RPP $-definable.
%%%%%%%%%%%%%\end{theorem}
%%%%%%%%%%%%%\begin{prf}
%%%%%%%%%%%%%It is enough to check that $ \prrpr{f} $ satisfies the constraints
%%%%%%%%%%%%%of Definition~\ref{definition:RPP-definability of any f in PRF}, proceeding
%%%%%%%%%%%%%by structural induction on $ f $. In particular,
%%%%%%%%%%%%%if $ f $ is $ \prCom{h,g_1,\ldots,g_k} $:
%%%%%%%%%%%%%\begin{align*}
%%%%%%%%%%%%%\arraycolsep=1.4pt
%%%%%%%%%%%%%\begin{array}{rcl}
%%%%%%%%%%%%% \left. {\scriptsize 
%%%%%%%%%%%%%            \begin{array}{r}
%%%%%%%%%%%%%             x \\[0.5mm] \myVec{x}
%%%%%%%%%%%%%             \\[0.5mm] 0^{n} 
%%%%%%%%%%%%%             \\ n
%%%%%%%%%%%%%           \left \{
%%%%%%%%%%%%%           \begin{array}{l}
%%%%%%%%%%%%%             0^{k-1}\\ \cdots \\[0.5mm] 0^{k-1} 
%%%%%%%%%%%%%           \end{array} 
%%%%%%%%%%%%%           \right .
%%%%%%%%%%%%%           \\ 
%%%%%%%%%%%%%           n \left \{
%%%%%%%%%%%%%           \begin{array}{l}
%%%%%%%%%%%%%             0^{j}\\ \cdots \\[0.5mm] 0^{j} 
%%%%%%%%%%%%%           \end{array} 
%%%%%%%%%%%%%           \right .
%%%%%%%%%%%%%            \end{array} 
%%%%%%%%%%%%%         } \right[
%%%%%%%%%%%%% & \prrpr{\prCom{h,g_1,\ldots,g_k}} &
%%%%%%%%%%%%% \left] {\scriptsize 
%%%%%%%%%%%%%            \begin{array}{l}
%%%%%%%%%%%%%             x+f(g_1(\myVec{x}),\ldots,g_n(\myVec{x})) \\[0.5mm] \myVec{x}
%%%%%%%%%%%%%             \\[0.5mm] 0^{n} 
%%%%%%%%%%%%%             \\ \!\!
%%%%%%%%%%%%%           \left .
%%%%%%%%%%%%%           \begin{array}{l}
%%%%%%%%%%%%%             0^{k-1}\\ \cdots \\[0.5mm] 0^{k-1} 
%%%%%%%%%%%%%           \end{array} 
%%%%%%%%%%%%%           \right \} n
%%%%%%%%%%%%%             \\ \!\!
%%%%%%%%%%%%%           \left .
%%%%%%%%%%%%%           \begin{array}{l}
%%%%%%%%%%%%%             0^{j}\\ \cdots \\[0.5mm] 0^{j} 
%%%%%%%%%%%%%           \end{array} 
%%%%%%%%%%%%%           \right \} n
%%%%%%%%%%%%%            \end{array} 
%%%%%%%%%%%%%        } \right.
%%%%%%%%%%%%%\end{array}
%%%%%%%%%%%%%\end{align*}
%%%%%%%%%%%%%and if $ f $ is $ \prRec{g}{h} $, then:
%%%%%%%%%%%%%\begin{align*}
%%%%%%%%%%%%%\arraycolsep=1.4pt
%%%%%%%%%%%%%\begin{array}{rcl}
%%%%%%%%%%%%% \left. {\scriptsize 
%%%%%%%%%%%%%            \begin{array}{r}
%%%%%%%%%%%%%             x \\ y \\ \myVec{x}
%%%%%%%%%%%%%             \\[0.5mm] 0^{3+m} 
%%%%%%%%%%%%%            \end{array} 
%%%%%%%%%%%%%         } \right[
%%%%%%%%%%%%% & \prrpr{\prRec{g}{h}} &
%%%%%%%%%%%%% \left] {\scriptsize 
%%%%%%%%%%%%%            \begin{array}{l}
%%%%%%%%%%%%%             x+
%%%%%%%%%%%%%             h(\myVec{x},y-1,
%%%%%%%%%%%%%                h(\myVec{x},y-2,\ldots
%%%%%%%%%%%%%                  h(\myVec{x},1,g(\myVec{x}))\ldots)) 
%%%%%%%%%%%%%             \\ y \\ \myVec{x}
%%%%%%%%%%%%%             \\[0.5mm] 0^{3+m} 
%%%%%%%%%%%%%            \end{array} 
%%%%%%%%%%%%%        } \right.
%%%%%%%%%%%%%\end{array}
%%%%%%%%%%%%%\enspace ,
%%%%%%%%%%%%%\end{align*}
%%%%%%%%%%%%%where $ \myVec{x} $ stands for $x_1, \ldots, x_k$. 
%%%%%%%%%%%%%\qed
%%%%%%%%%%%%%\end{prf}
%%%%%%%%%%%%%}{
%%%%%%%%%%%%%   VECCHIA COMPLETEZZA FINE
%%%%%%%%%%%%%%%%%%%%%%%%%%%%%%%%%%%%%%%%%%%%%%%%%%%%%%%%%%%%%%%%%%%
\begin{theorem}[$\RPP $ is $ \PRF $-complete]
\label{theorem:RPP is PRF-complete}
If $ f\in\PRF^n $ then $ \prrpr{f}$ is $ \RPP^{n+a+1} $-definable, for some $a\in\mathbb{N}$.
\end{theorem}
\begin{prf}
\begin{table}%%%%%%%%%%%%%%%%%%%%%%%%%%%%%%%%%%%%%%%%%%%%%%%%%%%%%%%%%%%%%%%%%%%%%%%%%%%%%%%%%%%%%%
\hrule   
$$\begin{array}{c}
\arraycolsep=1.4pt
\begin{array}{rcl}
 \left. {\scriptsize 
            \begin{array}{r}
             x \\ x_1 \\ \cdots \\ x_n
             \\[1mm]
              0^m\\
           k\left\{
              \begin{array}{r}
             r_1 \\ \cdots\\ r_{i-1}\\[1.5mm] 0\\[.5mm] r_{i+1}\\\cdots \\ r_k
              \end{array}
              \right.
            \end{array} 
         } \right[
 & g_i^{*} &
 \left] {\scriptsize 
            \begin{array}{l}
             x  \\ x_1 \\ \cdots \\ x_n
             \\[1mm]
             0^m\\
            r_1\\ \cdots \\ r_{i-1} \\[1mm]
          g_i(x_1,\ldots,x_n) \\ 
          r_{i+1}  \\\cdots\\ r_k 
            \end{array} 
        } \right.
\end{array}
\hspace{1cm} %%%%%%%%%%%%%%%%%%%%%%%%%%%%%%%
\begin{array}{rcl}
 \left. {\scriptsize 
            \begin{array}{r}
             x \\ x_1 \\ \cdots \\ x_n
             \\[1mm]
              0^m\\[1mm]
           k\left\{
              \begin{array}{c}
             0 \\[1mm] \cdots \\[1mm] 0
              \end{array}
              \right.\!\!\!\!\!
            \end{array} 
         } \right[
 & h^* &
 \left] {\scriptsize 
            \begin{array}{l}
             x  +\prCom{h,g_1,\ldots,g_k}(x_1,\ldots,x_n)\\ x_1 \\ \cdots \\ x_n
             \\[1mm]
          0^m\\
          g_1(x_1,\ldots,x_n) \\[1mm] \cdots\\[1mm] g_k(x_1,\ldots,x_n)
            \end{array} 
        } \right.
\end{array}
\\\\
%%%%%%%%%%%%%%%%%%%%%%%%%%***************************************
\begin{array}{rcl}
 \left. {\scriptsize 
            \begin{array}{r}
             x \\ x_1 \\ \cdots \\ x_n
             \\[1mm]
              0^m\\[1mm]
           k\left\{
              \begin{array}{r}
             0 \\[1mm] \cdots \\[1mm] 0
              \end{array}
              \right.
            \end{array} 
         } \right[
 & \!\!\!G\!\!\! &
 \left] {\scriptsize 
            \begin{array}{l}
             x  \\ x_1 \\ \cdots \\ x_n
             \\[1mm]
             0^m\\
          g_1(x_1,\ldots,x_n) \\[1mm] \cdots\\[1mm] g_k(x_1,\ldots,x_n)
            \end{array} 
        } \right.
\end{array}
\hspace{-3mm}
\begin{array}{rcl}
 \left. {\scriptsize 
            \begin{array}{r}
              x \\ x_1 \\ \cdots \\ x_n
             \\[1mm]
             0^m\\
             0^k
            \end{array} 
         } \right[
 & \!\!\!H\!\!\! &
 \left] {\scriptsize 
            \begin{array}{l}
             x  +\prCom{h,g_1,\ldots,g_k}(x_1,\ldots,x_n)\\ x_1 \\ \cdots \\ x_n
             \\[1mm]
          0^m\\
          0^k
            \end{array} 
        } \right.
\end{array}
\end{array}$$
\hrule
\caption{Composition component relations}\label{compositionComponentRelations}
\end{table}

The proof is given by induction on the definition of the primitive recursive function $f$. 
%For sake of simplicity, we present some of its cases in an exemplified form.
\begin{itemize}[leftmargin=5mm]
\item  
If $f$ is $\prZero^n$ then let $\prrpr{\prZero^n}:= \rprId^{n+1}$, where $a=0$.
\item 
If $f$ is $\prSucc^n_i$ then let $\prrpr{\prSucc^n_i}:= \rprInc^{n+1}_{i+1;1} \rprSeq (\rprPCom{\rprSucc}{\rprId^n}) $, where $a=0$.
\item 
If $f$ is $ \prProj{n}{i} $ ($1\leq i\leq n$) then, let  $\prrpr{ \prProj{n}{i} }:= \rprInc^{n+1}_{i+1;1}$, where $a=0$.
\item 

Let $f=\prCom{h,g_1,\ldots,g_k}$ where $ g_1,\ldots,g_k\in\PRF^{n} $ and $ h\in\PRF^{k} $, for some $k\geq 1$.
 %Let $ g_1,\ldots,g_k\in\PRF^{n} $ and $ h\in\PRF^{k} $ in $ \prCom{h,g_1,\ldots,g_k}\in\PRF^{n}$.
Therefore, by inductive hypothesis,  there are $ \prrpr{g_1} \in \RPP^{n+a_1+1}$, $\ldots$, $\prrpr{g_k}\in\RPP^{n+a_k+1} $ and 
$ \prrpr{h}\in\RPP^{k+a_0+1} $ for $ a_0,\ldots,a_k\in\mathbb{N}$.

 Let $m=\max\{ a_0,\ldots,a_k\}$.
By using $\prrpr{g_i}$, it is easy to build $g_i^{*}\in \PRF^{n+m+k+1}$ computing 
the reversible permutation described in the top-left of Table \ref{compositionComponentRelations}.
The sequential composition can be used to compute  the permutation $G$ described in the bottom-left of Table \ref{compositionComponentRelations}.
By using $G$ and $h$, it is easy to compute $h^*$ in the top-right of Table \ref{compositionComponentRelations}.
Last, $H\in\RPP^{n+(m+k)+1}$ in the bottom-right of Table \ref{compositionComponentRelations} is 
 $\prrpr{\prCom{h,g_1,\ldots,g_k}}$ and we define it as $h^* \rprSeq \rprInv{G}$.

$H$ improves the representation of composition sketched  in \cite{PaoliniPiccoloRoversiICTCS2015} because
it reduces the number of ancillae by re-using them to compute one after the other $ g_1,\ldots,g_k\in\PRF^{n} $ and $ h\in\PRF^{k} $.

\item 
Let $f\in\PRF^{n}$ where $n\geq 1$ be defined by means of the primitive recursion on $ g\in\PRF^{n-1} $ and $ h\in\PRF^{n+1} $.
By inductive hypothesis there are $ \prrpr{g} \in \RPP^{(n-1)+a_g+1}$ and $ \prrpr{h}\in\RPP^{(n+1)+a_h+1} $ for 
$ a_g,a_h\in\mathbb{N}$.

The definition of $\prrpr{f}$ requires:
(i) a temporary argument to store the result of the previous recursive call;
(ii) a temporary argument  to stack  intermediate results;
(iii) a temporary argument  to index the current iteration step;
(iv) a temporary argument to contain the final result which is not modified when the computation is undone for cleaning temporary values; and
(v) in different times, as many temporary arguments as $a_g$ to compute $\prrpr{g}$, 
                        as many temporary arguments as $a_h$ to compute $\prrpr{h}$ and
                        $ 10 $ 
                        ancillae to $\rprPush$ and $\rprPop$ elements in the course of the computation
                        (see Proposition \ref{proposition:Representing stacks in RPP}.)


\begin{table}\hrule\centering
 $$\hspace{-3mm}
\begin{array}{c}
\arraycolsep=1.4pt
\begin{array}{rcl}
      \left. {\scriptsize 
          \begin{array}{r}
            x \\ x_1 \\ \cdots \\ x_n
            \\[1.2mm]
              0 \\ \cdots \\ 0\\[1mm]
              0^4
          \end{array}
        } \right[
      & \prrpr{f} &
      \left] {\scriptsize 
          \begin{array}{l}
            f(x_1,\ldots,x_n)\\[1pt] x_1 \\ \cdots \\ x_n
            \\[2pt]
             \!\!\left. 
              \begin{array}{l}
                0\\ \cdots\\ 0
              \end{array}\right\}\max\{a_g,a_h,10\}\\
             0^4
          \end{array} 
        } \right.
    \end{array}\hspace{8mm}
\begin{array}{rcl}
      \left. {\scriptsize 
          \begin{array}{r}
           0\\ r \\ x_1 \\ \cdots \\ x_{n-1}\\ i
            \\[1.2mm]
            0^{a-4}
            \\[1mm]
            \mathcal{S}
          \end{array}
        } \right[
      &  h_\text{step} &
      \left] {\scriptsize 
          \begin{array}{l}
           0\\ \prrpr{h}(r,x_1,\ldots,x_{n-1},i)
            \\ x_1 \\ \cdots \\ x_{n-1}\\ i+1
            \\[1mm]
            0^{a-4}
            \\[1mm]
             \langle r, \mathcal{S} \rangle
          \end{array} 
        } \right.
    \end{array}\\
%$\;$\\   
  \end{array}$$
\hrule
  \caption{Primitive Recursion Components}\label{fig:pimRecComponents}
\end{table}

Let $a:=4+max\{a_g,a_h, 10\}$. Our goal is to define $\prrpr{f}\in\RPP^{n+a+1}$ which 
behaves as described in the left of Table~\ref{fig:pimRecComponents}:
\begin{enumerate}[leftmargin=5mm]
%\setlength{\itemindent}{-.5in}
\item 
A first block of operations which we call $F_1$ reorganizes the inputs of $\prrpr{f}$ in Table~\ref{fig:pimRecComponents}
to obtain $0, 0, x_1,\ldots, x_{n-1}, \overbrace{0,\ldots,  0}^{a-2}, x_n,x$.

\item
The result of the previous step becomes the input of $\rprPCom{\rprId^1}{\rprPCom{\prrpr{g}}{\rprId^{a-a_g}}}$ 
which yields $0,g(x_1,\ldots, x_{n-1}), x_1,\ldots, x_{n-1}, \overbrace{0,\ldots, 0}^{a-2}, x_n,x$.  
We notice that $x_n$ is the argument that drives the iteration.

\item 
The previous point supplies the arguments to the $x_n$-times applications of the recursive step:
$$h_\text{step} := (\rprPCom{\prrpr{h}}{\rprId^{a-(a_h+2)}})\rprSeq \rprSucc^{n+a-1}_{n+2}
   \rprSeq \rprPush^{n+a-1}_{2;n+a-1} \rprSeq (\rprPCom{\rprBSwap}{\rprId^{n+a-3}})$$
by means of $\rprPCom{\rprIt{}{}{h_{\text{step}}}}{\rprId^1}$.
The relation that $h_\text{step}$ implements is depicted in Table~\ref{fig:pimRecComponents}.
Thus, $h_\text{step}$ takes $n+a-1$ input arguments only among the $ n+a+1 $ available
because the first one serves to the identity and the other one drives the iteration.
The function of $ h_\text{step} $ first applies $\prrpr{h}$ and then re-organizes arguments for the next iteration. Specifically, 
(i) it increments the step-index in the argument of position $(n+2)$,
(ii) it pushes the result of the previous step, which is in position $2$, 
on top of the stack , which is in its last ancilla,
(iii) it exchanges the first two arguments, the first one containing the result of the last 
iterative step and the second one containing a fresh zero produced by $\rprPush$.
\item 
We get $0, f(x_1,\ldots, x_{n}), x_1,\ldots, x_{n-1}, x_n, \overbrace{0,\ldots,  0}^{a-2}, x_n,x$
from the previous point. We add the result to the last line by means of 
$\rprInc_{2;n+a+1}$ by yielding 
$0, f(x_1,\ldots, x_{n}), x_1,\ldots, x_{n-1}, x_n, \overbrace{0,\ldots,  0}^{a-2}, x_n,x+f(x_1,\ldots, x_{n})$.

\item 
We then conclude by unwinding the first three steps.
\end{enumerate}
\end{itemize}
\par\noindent
Summing up, $\prrpr{h}$ is:
$$ F_1 \rprSeq \left(\rprPCom{\rprId^1}{\rprPCom{\prrpr{g}}{\rprId^{a-a_g}}}\right) \rprSeq \left(\rprPCom{\rprIt{}{}{h_{\text{step}}}}{\rprId^1}\right)
\rprSeq \rprInc_{2;n+a+1} \rprSeq \rprInv{\left(\rprPCom{\rprIt{}{}{h_{\text{step}}}}{\rprId^1}\right)} \rprSeq 
   \rprInv{\left(\rprPCom{\rprId^1}{\rprPCom{\prrpr{g}}{\rprId^{a-a_g}}}\right)}  \rprSeq \rprInv{F_1} 
\enspace . $$
\qed
\end{prf}

\subsection{$ \RPP $ is $ \PRF $-sound}
The mere intuition should support the evidence that every $ f\in\RPP $ has a representative inside $ \PRF $ we
can obtain via the bijection which exists between $ \Int $ and $ \Nat $. 
More precisely, every $ \RPP $-permutation can be represented  as a $ \PRF $-endofunction on tuples of natural 
numbers which encode integers. Details on how formalizing the embedding of $ \RPP $ into $ \PRF $ are in \cite{PaoliniPiccoloRoversiICTCS2015}.

%%%%%%%%%%%%%%%%%%%%%%%%% servono ad emacs
%%% Local Variables:
%%% mode: latex
%%% TeX-master: "main.tex"
%%% ispell-local-dictionary: "american"
%%% End:
\section{Conclusions}
\label{section:Some recursion theoretic side effects of RPP}

% \todo{Question 3, Reviewer 1}
In this paper we introduce the class $\RPP$ of functions which
(i) is closed under inversion,
(ii) is both $ \PRF $-complete and $ \PRF $-sound,
(iii) provides a reasonable balance between conciseness and easiness of usage, suitable for recursion theoretical analysis,
(vi) is a good candidate for extensions able to encompass recursive bijections  which are not strictly total endo-functions 
--- and recursive partial functions	and 
(v) which is a good starting point to formalize classical results about the recursion theory in reversible settings.

%\medskip

We add some final comments on intensional aspects of $ \RPP $.

\subsection{$ \RPP $ and Ackermann}
Let $ A $ be the Ackermann function, example of total computable function with two arguments 
that cannot belong to $ \PRF $ because its growth rate is too high. 
 Kuznecov shows that a primitive recursive  function $ F $ exists with input arity $ 1 $ such that $ \rprInv{F}(x) = A(x,x) $.
  %{A reasonably detailed and accessible version of Kuznecov's proof is \cite{Simpson2009}.}
It is worth to remark that the inverse of $F$ is not primitive recursive, because $A$ is not.
References are not immediate, because the original result \cite{kuznecov50sssr} is in Russian:
some details about Kuznecov's proof are in \LPx{\cite{PaoliniPiccoloRoversiICTCS2015,Simpson2009}}{\cite{paolini2017ngc,Simpson2009}}.
Moreover in \cite[Exercise 5.7, p.25]{soare1987book} Kuznecov's result is slightly reformulated
by the statement saying that 
``primitive recursive functions do not form a group under composition.''

By Theorem~\ref{theorem:RPP is PRF-complete}, the function:
\begin{align*}
%\label{align:00}
\arraycolsep=1.4pt
\begin{array}{rcl}
 \left. {\scriptsize 
         \begin{array}{r} 
           w  \\ z  \\[1mm] 0^{k}
         \end{array}} \right[
 & \prrpr{F} &
 \left] {\scriptsize 
         \begin{array}{l} 
           w + F(z)\\ z \\[1mm] 0^{k}
         \end{array}} \right.
\end{array}
\end{align*}
exists and belongs to $\RPP $, for some $ k $. 
Proposition~\ref{proposition:The function rprInv} implies that $ \rprInv{\prrpr{F}} $ is in 
$ \RPP $, is computable and is such that:
\begin{align*}
%\label{align:01}
\arraycolsep=1.4pt
\begin{array}{rcl}
 \left. {\scriptsize 
         \begin{array}{r} 
           w  \\ z  \\[1mm] 0^{k}
         \end{array}} \right[
 & \prrpr{F} &
 \left] {\scriptsize 
         \begin{array}{c} 
           w + F(z)\\ z \\[1mm] 0^{k}
         \end{array}} 
 \right.
\end{array}
\arraycolsep=1.4pt
\hspace{-0.6cm}
\begin{array}{rcl}
 \left. {\phantom{
         \scriptsize 
         \begin{array}{r} 
           w  \\ z  \\[1mm] 0^{k}
         \end{array}
         }} \right[
 & \rprInv{\prrpr{F}} &
 \left] {\scriptsize 
         \begin{array}{l} 
           w \\ z \\[1mm] 0^{k}
         \end{array}} 
 \right.
\end{array}
\enspace .
\end{align*}
This highlights the strongly intensional nature of the reversible functions inside $ \RPP $.
The inversion of a permutation $p$ undoes what $ p $ executes, so
$ \rprInv{\prrpr{F}} $ algorithmically searches the value $ x $ such that $ A(x,x) = z $,  
for any given argument $ z $ we can pass to $ F $, by using $F(z)$ as bound for the iteration.

\subsection{$ \RPP$ and no-cloning}
A form of no-cloning theorem holds in the setting of reversible computing in analogy with the no-colning theorem for the quantum computing.

\begin{theorem}[No-cloning theorem \cite{matos2016notes}]
\label{theorem:No cloning theorem ESRL}
Let $\Int^{k+2}\rightarrow \Int^{k+2}$ represent the class of permutations with arity $ k+2 $, for any fixed 
$k\in \mathbb N$. 
Let $i \neq j$ belong to the initial segment $[1,k+2]$ of $ \mathbb N $ such that
$ i $ and $ j $ identify two arguments of the permutations in $\Int^{k+2}\rightarrow \Int^{k+2}$.
No permutation can always return the same value on the arguments of position $i$ and $j$,  
independently of their values.
\end{theorem}
\begin{proof}
Permutations are bijections. So they are surjections which necessarily range over the whole co-domain.
\end{proof}
\noindent
Quoting from \cite[p.530]{nielsen2011book}:
``\ldots the no-cloning theorem states that quantum mechanics does not allow unknown quantum states to be copied exactly, and places severe
limitations on our ability to make approximate copies \dots [\textit{however}] the no-cloning
theorem does not prevent all quantum states from being copied, it simply says that non-orthogonal quantum states cannot be copied.''

The moral is that cloning is not available if we deal with quantum computations in general.
However, in some cases programming strategies exist to circumvent Theorem~\ref{theorem:No cloning theorem ESRL}
and we can use them for cloning when we move inside reversible computing.
Cloning strategies boil down to using ancillary variables constrained to assume specific values.
As a simple instance, the general increment $ \rprInc_{j;i} $ of Section \ref{section:A library of functions in TRRF}
clones --- builds an exact a copy of --- the $ j $-th argument in its $ i $-th argument exactly when 
this latter initially assumes value 0.

\subsection{Ancillae in $ \RPP $} 
%\todo{Question 4, Reviever 1} 
In the previous subsection we refer to ancillae as tools to circumvent no-cloning.
For remarking once more that ancillae are harmless as far as reversible computing is concerned 
we focus on reversible computation as formalized by means of reversible Turing-machines ($\mathsf{RTM}$)
\cite{axelsen11lncs,axelsen16acta,bennett73ibm,jacopini90siam}. 
This is like saying that a function is reversible when, and only when, some $\mathsf{RTM}$ exists that computes it.
Any $\mathsf{RTM}$ crucially relies on an infinite tape that  contains infinite blank cells:  
%This seemingly unavoidable tape 
they supply an unbounded amount of ancillae that the computations can use at will.
%The ancillary tape 
An $\mathsf{RTM}$ allows to duplicate data at the cost of using states to recall where and how many copies are generated,
in order to revert the computation. This does not break the property of being in front of reversible computations.
%\LR{Interestingly, \ldots discuss how to taking advantage from limited irreversible erasures in the reversible simulation of irreversible computations.}
By the way, interesting discussions exist on how taking advantage from limited irreversible erasures on the ancillary tape 
when simulating irreversible computations inside reversible ones \cite{bennett1989siamjc,buhrman2001lncs,li1996royal}.

Turning our focus back on $\RPP$ we emphasize that our assumptions are analogous to those ones just recalled when working with $ \mathsf{RTM} $
for simulating standard Turing-machines.
We recall that $ \RPP $ contains permutations only (Definition \ref{RevPrimPermutations}) and  
that no zero-constant function exist in it.
%So, the simulation of $\PRF$ via $\RPP$ necessarily relies on permutations which a Turing-machine as well as
%a $\mathsf{RTM}$ can compute.
The formalization of how the simulation works follows a path which is standard when comparing computational models 
\cite{cutland1980book,malcev70book,odifreddi1989book}.
For example, representing a function by a Turing-machine requires to fix how supplying data on the initial tape 
and where initially positioning its head.
Concerning $ \RPP $, its simulation of $ \PRF $ exploits permutations purposefully devised to behave as required when we supply the value
0 to some specific arguments, the ancillae. 
Summarizing, we define restriction-less permutations which behave as required as soon as the simulation conventions are satisfied.

%In analogy with the reversible boolean circuits we have ancillary wires which assure that circuits are reversible.
%Setting an ancillary wire with a specific value, the circuit behaves as required.

%nevertheless, we have to formalize the encoding (a conventional agreement) that the simulation has to respect (cf. Definition \ref{definition:RPP-definability of any f in PRF}).
%This is the standard approach taken comparing computational models \cite{cutland1980book,malcev70book,odifreddi1989book},
%e.g. to represent a function in Turing-machine we have to establish how data are supplied on the initial tape and where the head is positioned.
%To be more explicit, we paraphrase the  Definition \ref{definition:RPP-definability of any f in PRF}: 
%the permutations devised for the encoding of $\PRF$ are purposeful for the simulation only when used with zeroes supplied to some specific arguments, 
%while different use of devised permutations are not pertinent (to our simulation). 
%Therefore, we do not define permutations working only when suitable zeroes are provided, but we are defining permutations without restrictions;
%nevertheless, the behaviour of $\PRF$-encoding permutations is relevant only when the simulation conventions are satisfied.
% Concluding, we recall that ancillae are typically used in the reversible circuit model.
%The reversible circuit model shares many aspects with $\RPP$, and in particular the use of ancillary wires do not break the reversibility,
%at least, until we do not assume the availability of ancillary bits (zero or one) for them.
%More explicitly, the circuit (with ancillary wires but without ancillary bits) can be reversed by obtaining a reversible circuit,
%the inversion issues arises in presence of ancillary bits. 

%%%%%%%%%%%%%%%%%%%%%%%%% servono ad emacs
%%% Local Variables:
%%% mode: latex
%%% TeX-master: "main.tex"
%%% ispell-local-dictionary: "american"
%%% End:
\section{Future work}
\label{section:Related and future work}
We briefly discuss some of the possible directions we can follow to develop this work.

The formalization of $ \RPP $ adhere to a standard recursion theory approach to identify computational classes.
We see $ \RPP $ as a formalism which lies at the same level of abstraction as the imperative programming languages
$ \mathsf{SRL}$, and similar, that Matos introduces for dealing with reversible computations \cite{matos03tcs,matos2016notes}.
So, we are focused on aspects more closely related to the design of paradigmatic programming languages. 
For example, an open question is whether $ \RPP $ and $ \mathsf{SRL} $ are equivalent.
Answering it amounts to show that the selection --- a built-in axiom of $ \RPP $ --- can be simulated in $ \mathsf{SRL} $.

Despite $ \RPP $ is not a real programming language it can be a good place to start from for conceiving one. 
Even though the design of $ \RPP $ does not start with the aim of computing with isomorphisms among types,
like the point-free languages of \cite{James2014TheseusAH,10.1007/978-3-540-27764-4_16}, most of the programming examples 
we give with it are point-free, especially when we iterate or select functions. 
Moreover, multiple input and output arity can be at the base of the representation and manipulation of immutable data structures 
like arrays. In the lines of \cite{thomsen2015lncs}, no obstacle seems to exist against the introduction of primitives that 
hide ancillae with the aim of simplifying programming.
Finally, $ \RPP $ looks flexible enough to incorporate specific bijections with different input and output arity
like in \cite{DBLP:conf/rc/YokoyamaAG11}. This would extend $ \RPP $ to a class of functions which are not necessarily 
permutations.

%Thirdly, of course, we are working on a natural extension of $ \RPP $ which is complete
%with respect to Kleene partial recursive functions.

On the model theoretic point of view, it is obvious that $ \RPP $ and \LP{of}{} Lafont's circuit classes in \cite{Lafont2003257} 
share the same construction principles. Besides basic functions and two composition schemes, $ \RPP $ has iteration 
and selection schemes by means of which, for example, we can give compressed descriptions of Lafont's circuits.
%It follows that models of $ \RPP $ are instances of monoidal categories with 
%natural numbers as objects, and whose maps compose by means of series and parallel compositions. 
%\todo{Questions 2 and 3. \#Editor (spostato qui come da ipotesi iniziale.)}
Since $ \RPP $ is $ \PRF $-complete a relation with Burroni's category of primitive recursive functions 
in \cite{burroni1986} has to exist. Formalizing it will require to investigate \LP{on}{} the link between the formalizations of 
the natural numbers that the two approaches pursue. The categorical approach in \cite{burroni1986} relies on a
Peano-Lawvere axiom \cite{Lawvere1506}. Instead, $ \RPP $ see numbers in a Peano style,
even though $ \RPP $ does not contain any function which is constantly equal to 0.

Of course, the categorical structure in \cite{burroni1986} is not the only one we can focus on
to say what a model of $ \RPP $ is. Models of $ \RPP $ might well be instances of reversible $ \mathsf{PRO} $ already used
to study feedback-free reversible circuits \cite{burroni1986,burroni1993tcs,lafont1995rta,Lafont2003257,lafont2013sc}.
Also, we would not be surprised we could use the category $ \mathsf{PInj} $ of sets and functional injective relations as model
of $ \RPP $, already used in \cite{paolini2017lipics} as a model of $ \mathsf{Janus} $, a reversible imperative programming language 
\cite{Lutz86,Yokoyama:2007:RPL:1244381.1244404}.

Finally there is a quite close resemblance between the stack mechanism we implement to show that $ \RPP $ is $ \PRF $-complete
and the pebble game used to assess how efficient the simulation of a reversible computation by means of an irreversible one is
\cite{li1996royal,bennett1989siamjc}. Given that $ \RPP $ suggests a programming style where a computation can be undone as 
soon as the result it produces is at hand, and observing that information is piled up by coding it numerically, 
we plan to study if this features keep the overhead of the $ \PRF $ simulation by means of $ \RPP $ within an
interesting range of space complexity.

%[Vitani] We conjectured that all reversible simulations of
%an irreversible computation can essentially be represented as the pebble game
%de ned below, and that consequently the lower bound of Corollary 2 applies
%to all reversible simulations of irreversible computations.
	
%	\item [\textbf{Lafont}] %%% From Lafont
%	Boolean circuits are used to represent programs on nite data. Reversible Boolean circuits and
%	quantum Boolean circuits have been introduced to modelize some physical aspects of compu-
%	tation. Those notions are essential in complexity theory, but we claim that a deep mathematical
%	theory is needed to make progress in this area. For that purpose, the recent developments of
%	knot theory is a major source of inspiration.
%	Following the ideas of Burroni, we consider logical gates as generators for some algebraic
%	structure with two compositions, and we are interested in the relations satis3ed by those gen-
%	erators. For that purpose, we introduce canonical forms and rewriting systems. Up to now, we
%	have mainly studied the basic case and the linear case, but we hope that our methods can be
%	used to get presentations by generators and relations for the (reversible) classical case and for
%	the (unitary) quantum case.
%	
	
	
%	%%% Yokoyama
%	\item [\textbf{Yokoyama}]
%	\cite{DBLP:conf/rc/YokoyamaAG11}
%	For exploring the theoretical foundations of reversible computing, we focus on
%	a purely-reversible and side-effect free first-order functional language. There exist
%	several imperative and pseudo-functional reversible languages. As far as we know,
%	Janus [12,20] is the first reversible language, and is imperative. We regard Gries'
%	invertible language [11], R and PISA [8] as also belonging to this category. Baker
%	proposed Ψ-Lisp [4], reversible linear Lisp; due to the use of a state and a hidden
%	history stack, it is neither purely reversible nor purely functional. Mu, Hu, and
%	Takeichi proposed INV [16], a point-free functional language with relational se-
%	mantics, which essentially includes both forward and backward nondeterminism.
%	Bowman, James, and Sabry have proposed Π [6], another point-free reversible
%	functional language. Because of the nature of point-free languages, these do not
%	have powerful pattern matching, so e.g. overlapping branch patterns in loops is
%	prohibited. While the above examples all have reversible language features, a
%	main contribution here is to separate reversibility in functional languages from
%	other features.
	

%\todo{Citare pebble games, semantica dei giochi, geometria dell'interazione altre parrocchie.
%	Referenze pebble games \cite{li1996royal} and among others  \cite{buhrman2001JPh,li1998pdnp}.
%	Dobbiamo ricollegarci al nostro precedente lavoro \cite{paolini2017lipics} sulla semantica categoriale.
%}


%%%%%%%%%%%%%%%%%%%%%%%%% servono ad emacs
%%% Local Variables:
%%% mode: latex
%%% TeX-master: "main.tex"
%%% ispell-local-dictionary: "american"
%%% End:


%%%%%%%%%%%%%%%%%%%%%%
\bibliographystyle{abbrv}
%\bibliographystyle{myrefs}
%\bibliographystyle{alphaurl}
%\bibliographystyle{rsl}
\section*{\refname}
\bibliography{bibliography}


%\begin{thebibliography}{00}

% \bibitem{label}
% Text of bibliographic item

% notes:
% \bibitem{label} \note

% subbibitems:
% \begin{subbibitems}{label}
% \bibitem{label1}
% \bibitem{label2}
% If there is a note, it should come last:
% \bibitem{label3} \note
% \end{subbibitems}

%\bibitem{}

%\end{thebibliography}
%%%%%%%%%%%%%%%%%%%%%%%%%%%%%%%%%%%%%%%%
%%%%%%%%%%%%%%%%%%%%%%%%%%%%%%%%%%%%%%%%
%%%%%%%%%%%%%%%%%%%%%%%%%%%%%%%%%%%%%%%%
%%%%%%%%%%%%%%%%%%%%%%%%%%%%%%%%%%%%%%%%
%%%%%%%%%%%%%%%%%%%%%%%%%%%%%%%%%%%%%%%%
%%%%%%%%%%%%%%%%%%%%%%%%%%%%%%%%%%%%%%%%
% The Appendices part is started with the command \appendix;
% appendix sections are then done as normal sections

%%%%%%%%%%%%%%%%%%%%%%%%%%%%%%%%%%%%%
%%%%%%%%%%%%%%%%%%%%%%%%%%%%%%%%%%%%%
%%\todo[inline]{Credo proprio che questa appendice rimarr\`a nascosta.}


\appendix

\section{Some further functions in $ \RPP $}
\label{Some further functions in RPP}

%%%%%%%%%%%%%%%
\paragraph{Halving a number}
Our goal is to define a function $ \rprDivTwo $ that takes an integer $ n $ and yields 
$ \lfloor \frac{n}{2} \rfloor $, \ie the best approximation from below of $ \frac{n}{2} $.

\ldots

We shall define a class of functions where the algorithm just described
exists in the form of a reversible function

Let $n, j, i, p, q\in\Nat$ such that $ j < i < p < q \leq n$.
We define:
\begin{align*}
\rprDivTwo^{n}_{j,p,q;i} =
 \rprIt{n}{j}{\rprSucc_p \circ \rprSwap{}{p}{q}}\circ
 \rprInc_{p;i} \circ
 \rprInv{\left(\rprIt{n}{j}{\rprSucc_p \circ \rprSwap{}{p}{q}}\right)} \circ
 \rprIf{n}{j}{\rprId}{\rprId}{\rprNeg_i} 
\end{align*}
Given $ \vec{a}^{n} $, we have 
$ \rprDivTwo^n_{j,p,q;i}\, \vec{a}^n = \vec{b}^n$ such that:
\begin{align*}
\la \vec{b}_1,\ldots,\vec{b}_{\,i-1},\vec{b}_{\,i+1},\ldots,\vec{b}_{\,n}\ra
 & = \la \vec{a}_1,\ldots,\vec{a}_{\,i-1},\vec{a}_{\,i+1},\ldots,\vec{a}_{\,n}\ra
\enspace ,
\\
\vec{b}_i & = 
\begin{cases}
\vec{a}_i + \vec{a}_p 
   + \underbrace{1 + \ldots + 1}_{\left| \lfloor\vec{a}_j/2 \rfloor\right|}
   & \textrm{ if } \vec{a}_j \textrm{ even } \\
\vec{a}_i + \vec{a}_q 
   + \overbrace{1 + \ldots + 1}^{}
   & \textrm{ if } \vec{a}_j \textrm{ odd }
   \enspace .
\end{cases}
\end{align*}
So, if $ \vec{a}_p, \vec{a}_p$ and $ \vec{a}_q $ are initially set to $ 0 $, then 
$ \rprDivTwo^n_{j,p,q;i} $ eventually sets $ \vec{a}_i $ to the value $ \left\lfloor\vec{a}_j/2 \right\rfloor $.
We get the result by first alternatively adding $ 1 $ as many times as to $ \vec{a}_p $ and $\vec{a}_q $. The global number
of $ 1 $s we add amounts to the absolute value of $ \vec{a}_j $. Finally, if $ \vec{a}_j $ is negative, then $ \vec{a}_i $
must be negative as well.

%%%%%%%%%%%%%%%
\paragraph{Squaring an integer}
Let $n, k, j, i\in\Nat$ such that $ k < j < i \leq n$.
We define:
\begin{align*}
\rprSq^n_{k,j;i} & = 
\rprInc_{j;k} \circ 
\rprIf{n}{j}{\rprId}{\rprId}{\rprNeg} \circ 
\rprMult^{n}_{k,j;i} \circ 
\rprInv{(\rprIf{n}{j}{\rprId}{\rprId}{\rprNeg})}\circ
\rprInv{(\rprInc_{j;k})}
\enspace .
\end{align*}
Given $ \vec{a}^{n} $, we have 
$ \rprSq^n_{k,j;i}\, \vec{a}^n = \vec{b}^n$ such that:
\begin{align*}
\la \vec{b}_1,\ldots,\vec{b}_{\,i-1},\vec{b}_{\,i+1},\ldots,\vec{b}_{\,n}\ra
 & = \la \vec{a}_1,\ldots,\vec{a}_{\,i-1},\vec{a}_{\,i+1},\ldots,\vec{a}_{\,n}\ra
\enspace ,
\\
\vec{b}_i & = \vec{a}_i + \underbrace{\vec{a}_j + \ldots + \vec{a}_j}_{\mid \vec{a}_k +\vec{a}_j \mid}
\enspace .
\end{align*}
So, if $ \vec{a}_k $ and $ \vec{a}_i $ are initially set to $ 0 $, then 
$ \rprSq^n_{k,j;i} $ eventually sets $ \vec{a}_i $ to the value $ (\vec{a}_j)^2 $.
We get the result by first adding the absolute value of $ \vec{a}_j $ to the initial value of 
$ \vec{a}_k $. If needed, we let $ \vec{a}_j $ positive. After the multiplication, of
$ \vec{a}_j $ by itself, we undo everything while keeping the result in $ \vec{a}_i $.


%%%%%%%%%%%%%%%
\paragraph{An integer square root of a positive integer}
Let $n, j, i, p, q, r\in\Nat$ such that $ j < i < p < q < r \leq n$. We define:
\begin{align*}
\rprSqRoot^n_{j,p,q,r;i} & =
\rprBMu{p<j}{i}
       { (\rprSq^n_{r,p;q}\circ
          \rprDec_{j;q}
         )
       }
       {p}{q}
\circ 
\rprPred_i
\enspace .
\end{align*}
Given $ \vec{a}^{n}$, we have 
$ \rprSqRoot^n_{j;i}\, \vec{a}^n = \vec{b}^n$ such that:
\begin{align*}
\la \vec{b}_1,\ldots,\vec{b}_{\,i-1},\vec{b}_{\,i+1},\ldots,\vec{b}_{\,n}\ra
 & = \la \vec{a}_1,\ldots,\vec{a}_{\,i-1},\vec{a}_{\,i+1},\ldots,\vec{a}_{\,n}\ra
\enspace ,
\\
\vec{b}_i & = 
\vec{a}_i + \lfloor \sqrt{\vec{a}_j}\rfloor 
\enspace .
\end{align*}
So, if $ \vec{a}_i, \vec{a}_p, \vec{a}_q$ and $ \vec{a}_r$ are initially set to $ 0 $, then 
$ \rprSqRoot^n_{j,p,q,r;i}\, \vec{a}^n $ eventually sets $ \vec{a}_i $ to  $ \lfloor \sqrt{x_j}\rfloor $.
We get the result by first looking for the least integer not greater than $ \vec{a}_j $ that we generate in 
$ \vec{a}_p $ and such that $ \vec{a}_p^2 > \vec{a}_j  $. Once found it, the best approximation of $ \sqrt{\vec{a}_j} $ 
from below is $ \vec{a}_p - 1$.


%%%%%%%%%%%%%%%%%%%%%%%%%%%%%%%%%%%%%%%%%%%%%%%%%%%%
\section{$ \RPP $ is $ \ESRL $-complete}
\label{section:RPP is ESRL-complete}

\todo[inline]{Lo mettermo in un altro lavoro, penso.}

Let $ P $ be a program of $ \ESRL $ \cite{matos03tcs} defined on a finite set $ \mathcal{R} $ of registers $ r_1, r_2, \dots $.
We can map $ P $ into a function $ \llbracket  P \rrbracket \in\RPP$ which we define recursively on the structure of $ P $:
\begin{align*}
\llbracket \mathtt{INC}\ r_i\rrbracket & = \rprSucc^{|\mathcal{R}|}_{r_i}
\\
\llbracket \mathtt{DEC}\ r_i\rrbracket & = \rprPred^{|\mathcal{R}|}_{r_i}
\\
\llbracket -r_i\rrbracket & = \rprNeg^{|\mathcal{R}|}_{r_i}
\\
\llbracket P_1; P_2\rrbracket & = \rprSCom{\llbracket P_1\rrbracket}
                                          {\llbracket P_2\rrbracket}
\\
\llbracket \mathtt{FOR}\ r_i\ \mathtt{\{} P \mathtt{ \} }\rrbracket & = 
\rprIf{|\mathcal{R}|}
      {r_i}
      {\llbracket  P \rrbracket}
      {\rprId^{|\mathcal{R}|}}
      {\rprInv{\llbracket  P \rrbracket}}
\enspace .
\end{align*}

\begin{lemma}[$ \RPP $ is $ \ESRL $-complete]
\label{lemma:RPP is ESRL-complete}
Let $ P $ any program of $ \ESRL $ on a set of registers $ \mathcal{R} $. 
Let $\mathcal{R}' $ denote the final values of the register after a run of $ P $.
If $ \vec{a}^{|\mathcal{R}|} $ is the initial tuple of values in $ \mathcal{R} $ and
$ \vec{b}^{|\mathcal{R}|} $ is the final tuple of values in $ \mathcal{R} $ after a run of $ P $,
Then $ \llbracket  P \rrbracket\, \vec{a}^{|\mathcal{R}|} = \vec{b}^{|\mathcal{R}|}$.
\end{lemma}



 

%%%%%%%%%%%%%%%%%%%%%%%%%%%%%%%%%%%%%
%%%%%%%%%%%%%%%%%%%%%%%%%%%%%%%%%%%%%

%%%%%%%%%%%%%%%%%%%%%%%%%%%%%%%%%%%%%%%%%%%%%%%%%%%%%%%%%%%%%%%%%%%%%%%%%%%%%%%%%%5
%%%%%%%%%%%%%%%%%%%%%%%%%%%%%%%%%%%%%%%%%%%%%%%%%%%%%%%%%%%%%%%%%%%%%%%%%%%%%%%%%%5
%%%%%%%%%%%%%%%%%%%%%%%%%%%%%%%%%%%%%%%%%%%%%%%%%%%%%%%%%%%%%%%%%%%%%%%%%%%%%%%%%%5
%\begin{center}
$\vspace{1cm}$\\
\rule{0.5\textwidth}{.4pt}\\
{\Huge *** FILE pLuca ***}
\end{center}

\section{Commenti}

\LP{}{Commenti Tecnici per LucaR} 

\begin{enumerate}
\item \LR{ Vogliamo chiamare le nuove funzioni \textbf{Reversible Primitive Permutations}? Reversible gia sott'intende calcolabili!
MACRO-PROBLEMA: servono 2 diverse macro: una per le nuove-RPRF, una per le vecchie-RPRF. }{Fatto. C'\`e
RPP per le nuove e RPRF per le vecchie.}

\item 
\LR{Bisogner\`a togliere il pacchetto geometry, se non vogliamo  falsificare lo stile ``elsarticle''.}{Tolto}

\item Se non ricordo male, se tu che hai il brutto vizio di  cambiare nome alle label: per favore non farlo, \LR{}{OK. Per\`o: di quali label parli?}
oppure modificalo consistentemente dappertutto (io uso fgrep sul vecchio nome-label).

\item \LR{Perche vuoi forzare uno stile non-Elsevier nei titoli dei teoremi? A me sembra di navigare controcorrente inutilmente e probabilmente in maniera controproducente ...}{In effetti non volevo. Ora uso
le varie macro \texttt{\\newtheorem}, etc. del pacchetto, come indicato nella guida }
\item 
Cambiamo la notazione delle finite-permutations? \LR{}{S\`i.} Io trovo la mia proposta la piu naturale: 
\`e una abbreviazione della array-notation. \LR{}{Anche io userei quella: si rischiano meno errori di adattamento, rispetto a quello che c'\`e gi\`a scritto, e rimane molto leggibile. Non abbiamo bisogno di 
tutta la compressione notazionale della cyclic notation. Per\`o ora non la faccio.}

\item \LR{Al posto di $f^*$ mettiamo $f^{-1}$ che viene normalmente usato per invertire le permutazioni? (In quantum c'e qualche alternativa ... ). }{Fatto.}
\end{enumerate}


\LP{}{Commenti per LucaP} 

\LR{}{Ho messo i commenti che (secondo te) erano pr te, nella Section~\ref{section:Some recursion theoretic side effects of RPP}}


\section{Proposte di Aggiunta}

\LR{Propongo di aggiungere la Proposition seguente dopo la Proposition \ref{proposition:Relating rprSComName and rprPComName}}
{Fatto. Ho tolto il testo da qui sotto e l'ho messo dove scritto.}

\LR{I seguenti teoremi potrebbero andare nell'introduzione. }{Fatto. Ho messo il testo che seguiva nella intro.}

\section{Expresiveness of *SLR}


We show how to simulate  \rprIfName in the programming language $\ESRL$.
%The \rprIfName of one among $ f, g $ and $ h $ is $ \rprIf{}{}{f}{g}{h} $, it belongs to $\RPRF^{k+1}$ and is such that:
%$\mathtt{INC}$ $$\mathtt{FOR}\ r_i\ \mathtt{\{} P \mathtt{ \} }$$


In the following, we will use some special registers that are safely hidden w.r.t. the program input/output. More precisely, we consider registers that are used by the programmer as temporary stores: we assume that these registers are suitably initialized by the programmer, not by the program user.
We do not want undefined programs, thus we assume a programmer duty to restore the input in each hidden register before the computation ends (thus, hidden registers are still hidden registers when a computation is reverted). If a program contain an hidden variable and does not respect the above assumption then the variable is not an hidden one. While register are ranged over $r$ with or without index and/or apex, for sake of simplicity we denote hidden registers by special letters. A list of useful hidden register is the following.
\begin{description}
\item[Zero registers] are registers initialized to zero (at the beginning of the computation and restored to zero before the program ends). 
We range over them by $z$ with or without index and/or apex.
\item[Logical Pairs] are pairs of registers initialized to $0,1$ respectively at the beginning of the computation.
We range over them by a pair of indexed letters $b_0,b_1$ with or without the same apex, e.g. $b'_0,b'_1$ and $b^3_0,b^3_1$ are two different logical register pairs.
\end{description}


\subsection*{Selection emulation}
Our aim is to define a program driven by the register $r_0$ that must operate a selection between three programs $P1, P2$ and $ P3$ (that cannot modify $r_0$).  W.l.o.g. we assume that $\mathtt{P1}, \mathtt{P2}$ and $ \mathtt{P3}$ do not affect our hidden registers.
We describe our solution by incrementally providing  and discussing pieces of code.

We start by checking if $r_0$ is a positive or not. 
If $r_0$ is a positive odd number then the goal is reached by the instruction in line 1 of the next code.  
If $r_0$ is a positive even number then, the instruction in line 1, became as the identity.
We circumvent the even case, by verifying the if the successor of  $r_0$  is a positive odd number.\\
 

\newcounter{rowcount}
\setcounter{rowcount}{0}
  \begin{tabular}{@{\stepcounter{rowcount}{\tiny\therowcount}\hspace*{3mm}}l}
    $\mathtt{FOR}\,r_0\{ \mathtt{SWAP} (b_0,b_1);   \mathtt{FOR}\,b_0\{ \mathtt{INC}\, z^{+}_0 \}; \mathtt{FOR}\,b_1\{ \mathtt{DEC}\, z^{+}_0 \}  \}$\\
\loopComment{ if $r_0$ is even then the above instructions leave all registers unchanged;
 if $r_0$ is odd then the logical pair $b_0,b_1$ is inverted and $z^{+}_0$ is modified:
 (i) if $r_0 \geq 0$ then $z^{+}_0$ became $+1$, (ii) if $r_0 < 0$ then $z^{+}_0$ became $-1$.  }\\
$\mathtt{FOR}\,b_1\{ $  \qquad \loopComment{ This code is executed only if $r_0$ is even} \\
 $\hspace{11mm}\mathtt{INC}\, r_0;$ \qquad\loopComment{ we consider the (odd) successor of $r_0$} \\
 $\hspace{11mm} \mathtt{FOR}\,r_0\{ \mathtt{SWAP} (b'_0,b'_1);   \mathtt{FOR}\,b'_0\{ \mathtt{INC}\, z^{+}_0 \}; \mathtt{FOR}\,b'_1\{ \mathtt{DEC}\, z^{+}_0 \}  \};$\\
$\hspace{11mm}$\loopComment{ If $r_0+1 \geq 0$ then $z^{+}_0$ became $+1$, otherwise became $-1$.  }\\
$\hspace{11mm} \mathtt{DEC}\, r_0 \}$\\
\loopComment{  Indipendently from the parity of $r_0$, if $r_0 \geq 0$ then $z^{+}_0=+1$ otherwise  $z^{+}_0=-1$; moreover, one of the two used logical pairs is inverted. }\\
$\mathtt{FOR}\,b_1\{  $  \qquad \loopComment{ This code is executed only if $r_0$ is even} \\
$\hspace{11mm} \mathtt{SWAP} (b'_0,b'_1);\}$ \loopComment{ restore the logical pair to its starting value}\\[5mm]
\end{tabular}

At the end of the above pieces of code we reach the following situation: (i) $r_0$ is unchanged; (ii)  if $r_0$ is even then the pair $b_0,b_1$ is unchanged, otherwise is inverted;
(iii) the pair $b'_0,b'_1$ is unchaged;  (iv) if $r_0\geq 0$ then $z^{+}_0$ contains $+1$, otherwise $z^{+}_0$ contains $-1$.

Next steps is the dual, in some sense: if $r_0\leq 0$ we aim to put $+1$ in a hidden variable $z^{-}_0$, otherwise $-1$. Sometimes we re-use available information.\\

\begin{tabular}{@{\stepcounter{rowcount}{\tiny\therowcount}\hspace*{3mm}}l}
$\mathtt{FOR}\,b_0\{ $  \qquad \loopComment{This code is executed only if $r_0$ is odd} \\
 $\hspace{11mm}\mathtt{FOR}\,r_0\{\mathtt{SWAP} (b'_0,b'_1);\mathtt{FOR}\,b'_0\{\mathtt{DEC}\, z^{-}_0 \};\mathtt{FOR}\,b'_1\{\mathtt{INC}\, z^{-}_0 \}\}$\\
$\hspace{11mm}$\loopComment{ If $r_0 \leq 0$ then $z^{-}_0$ became $+1$, otherwise became $-1$.  }\\
$\hspace{11mm}\}$\\
$\mathtt{FOR}\,b_1\{ $  \qquad \loopComment{This code is executed only if $r_0$ is even} \\
 $\hspace{11mm}\mathtt{DEC}\, r_0;$\\
$\hspace{11mm} \mathtt{FOR}\,r_0\{  \mathtt{SWAP} (b'_0,b'_1); \mathtt{FOR}\,b'_0\{ \mathtt{DEC}\, z^{-}_0 \}; \mathtt{FOR}\,b'_1\{ \mathtt{INC}\, z^{-}_0 \} \};$\\
$\hspace{11mm} \mathtt{INC}\, r_0;$\\
$\hspace{11mm}$\loopComment{ If $r_0 \leq 0$ then $z^{-}_0$ became $+1$, otherwise became $-1$.  }\\
$\hspace{11mm}\}$\\
$\mathtt{SWAP} (b'_0,b'_1);$ \qquad\loopComment{ Restore the logical pair to $0,1$ }\\[5mm] 
  \end{tabular}

At the end of the above pieces of code we reach the following situation: (i) $r_0$ is unchanged; (ii)  if $r_0$ is even then the pair $b_0,b_1$ is unchanged, otherwise is inverted;
(iii) the pair $b'_0,b'_1$ is unchanged;  (iv) if $r_0\geq 0$ then $z^{+}_0$ contains $+1$, otherwise contains $-1$, (v) if $r_0\leq 0$ then $z^{-}_0$ contains $+1$, otherwise contains $-1$.

\medskip

We are ready to realize a selection agreeing with that of Definition \ref{RevPrimPermutations} by combining  information in $z^{+}_0$ and $z^{-}_0$, 
summarized in the next table
$$\begin{array}{r||c|c|c|}
  r & \leq 0 & = 0 & \geq 0 \\
\hline
z^{+}_0 & +1 & +1 & -1 \\
z^{-}_0 & -1 & +1 & +1 \\
\hline\hline
 z^{+}_0 + z^{-}_0 & 0 & +2 & 0\\
 z^{+}_0 - z^{-}_0 +1 & +3 & 1 & -1\\
z^{-}_0 - z^{+}_0 +1 & -1 & 1 & +3\\
\end{array}$$


The case ``$r_0$ is zero'' is based on the fact that $z^{+}_0+z^{-}_0=0$ whenever $r_0\neq 0$, otherwise is $z^{+}_0+z^{-}_0=2$.
%To record this sum we use an extra zero-variable $z^*_0$.
The integer-division allow us two executes $\mathtt{P1}$ when and only when $r_0= 0$.\\


\begin{tabular}{@{\stepcounter{rowcount}{\tiny\therowcount}\hspace*{3mm}}l}  
$\mathtt{FOR}\,z^{-}_0 \{\mathtt{INC}\, z^+_0 \};$\qquad \loopComment{Put in $z^{+}_0$ the sum of $z^{+}_0$ and $z^{-}_0$} \\
$\mathtt{FOR}\,z^{+}_0\{\mathtt{SWAP}(b'_0,b'_1);\mathtt{FOR}\,b'_0\{ \mathtt{P1} \} \};$ \qquad\loopComment{ $\mathtt{P1}$ is executed once iff $r_0=0$}\\
$\mathtt{FOR}\,z^{-}_0 \{\mathtt{DEC}\, z^+_0 \};$\qquad \loopComment{Restore $z^{+}_0$ to its previous value} \\[5mm]
\end{tabular}

At the end of the above pieces of code the content of our control-register (i.e. $r_0$, $b_0,b_1$, $b'_0,b'_1$, $z^{+}_0$ and $z^{-}_0$) is not chaged w.r.t. the starting situation. If $\mathtt{P1}$ has been executed then, it could affect without restriction all other  registers available to our program.

In order to implement the selection of $\mathtt{P2}$ when and only when $r_0\geq 0$, 
we use $ z^{+}_0 - z^{-}_0 +1$ and a trick on a logical pair. We observe the second component of the pair after one and three swapping.
If swap of a logical pair (initialized to $0,1$) is executed one time then the second component become (definitively) $0$. 
If swap of a logical pair (initialized to $0,1$) is executed three times, then the second component become $1$ only after the second swap.\\

\begin{tabular}{@{\stepcounter{rowcount}{\tiny\therowcount}\hspace*{3mm}}l}  
$\mathtt{FOR}\,z^{-}_0 \{\mathtt{DEC}\, z^+_0 \}; \mathtt{INC}\, z^+_0;$\qquad \loopComment{$ z^{+}_0 - z^{-}_0 +1$ is stored in $z^{+}_0$} \\
$\mathtt{FOR}\,z^{+}_0\{\mathtt{SWAP}(b'_0,b'_1);$\qquad\loopComment{This swap is executed 1 or 3 times}\\ 
$\hspace{11mm}\mathtt{FOR}\,b'_1\{ \mathtt{P1} \};$ \qquad\loopComment{ the test of $b'_1$ came across $1$ only if the swap is executed $3$ times}\\
$\hspace{11mm} \};$ \qquad\loopComment{ $\mathtt{P2}$ is executed once iff $r_0\geq 0$}\\
$\mathtt{SWAP}(b'_0,b'_1);$\qquad \loopComment{Restore  the pair $(b'_0,b'_1)$ to $0,1$ } \\
$\mathtt{DEC}\, z^+_0; \mathtt{FOR}\,z^{-}_0 \{\mathtt{INC}\, z^+_0 \}; $\qquad \loopComment{Restore $z^{+}_0$ to its previous value} \\[5mm]
\end{tabular}

At the end of the above pieces of code, the situation of our control-register is not changed (it has been carefully restored).  
If $\mathtt{P2}$ has been executed then it could affect without restriction all other  registers available to our program.
In order to implement the selection of $\mathtt{P3}$ when and only when $r_0\leq 0$, we adapt the same trick.\\

\begin{tabular}{@{\stepcounter{rowcount}{\tiny\therowcount}\hspace*{3mm}}l}  
$\mathtt{FOR}\,z^{+}_0 \{\mathtt{DEC}\, z^-_0 \}; \mathtt{INC}\, z^-_0;$\qquad \loopComment{$ z^{-}_0 - z^{+}_0 +1$ is stored in $z^{-}_0$} \\
$\mathtt{FOR}\,z^{-}_0\{\mathtt{SWAP}(b'_0,b'_1);\mathtt{FOR}\,b'_1\{ \mathtt{P3} \} \};$ \qquad\loopComment{ $\mathtt{P3}$ is executed once iff $r_0\leq 0$}\\
$\mathtt{SWAP}(b'_0,b'_1);$\qquad \loopComment{Restore  the pair $(b'_0,b'_1)$ to $0,1$ } \\
$\mathtt{DEC}\, z^-_0; \mathtt{FOR}\,z^{+}_0 \{\mathtt{INC}\, z^-_0 \}; $\qquad \loopComment{Restore $z^{-}_0$ to its previous value} \\[5mm]
\end{tabular}

At the end of the above pieces of code, the situation of our control-register is not changed (it has been carefully restored). 
Patently, if $\mathtt{P2}$ has been executed then it could affect without restriction all other  registers available to our program.
We remind the following situation: (i) $r_0$ is unchanged; (ii)  if $r_0$ is even then the pair $b_0,b_1$ is unchanged, otherwise is inverted;
(iii) the pair $b'_0,b'_1$ is unchanged;  (iv) if $r_0\geq 0$ then $z^{+}_0$ contains $+1$, otherwise contains $-1$, (v) if $r_0\leq 0$ then $z^{-}_0$ contains $+1$, otherwise contains $-1$.

In order to conclude our program, we must correctly restore all our hidden variables. We start by restoring $z^{+}_0,z^{-}_0$ by using the parity information (about $r_0$) stored in the logical pair $b_0,b_1$. First, we consider the case $r_0$ is odd.\\

\begin{tabular}{@{\stepcounter{rowcount}{\tiny\therowcount}\hspace*{3mm}}l}
$\mathtt{FOR}\,b_0\{ $  \qquad \loopComment{This code is executed only if $r_0$ is odd} \\
 $\hspace{11mm}\mathtt{FOR}\,r_0\{\mathtt{SWAP} (b'_0,b'_1);\mathtt{FOR}\,b'_0\{\mathtt{DEC}\, z^{+}_0 \};\mathtt{FOR}\,b_1\{\mathtt{INC}\, z^{+}_0\}  \};$\\
$\hspace{11mm}\mathtt{SWAP} (b'_0,b'_1);$\qquad\loopComment{ Restore the logical pair to $0,1$ }\\
 $\hspace{11mm}\mathtt{FOR}\,r_0\{\mathtt{SWAP} (b'_0,b'_1);\mathtt{FOR}\,b'_0\{\mathtt{INC}\, z^{-}_0 \};\mathtt{FOR}\,b'_1\{\mathtt{DEC}\, z^{-}_0 \}\};$\\
$\hspace{11mm}\mathtt{SWAP} (b'_0,b'_1);$\qquad\loopComment{ Restore the logical pair to $0,1$ }\\
$\hspace{11mm}\}$\\[5mm]
 \end{tabular}

If $r_0$ is odd then all register contents, but the logical pair $b_0,b_1$, have been re-establish to the program-starting situation.
We consider the case $r_0$ is even.\\

\begin{tabular}{@{\stepcounter{rowcount}{\tiny\therowcount}\hspace*{3mm}}l}
$\mathtt{FOR}\,b_1\{ $  \qquad \loopComment{This code is executed only if $r_0$ is even} \\
$\hspace{11mm}\mathtt{INC}\, r_0;$\qquad \loopComment{we are driven by the successor of $r_0$}\\
 $\hspace{11mm}\mathtt{FOR}\,r_0\{  \mathtt{SWAP} (b'_0,b'_1); \mathtt{FOR}\,b'_0\{ \mathtt{DEC}\, z^{+}_0 \}; \mathtt{FOR}\,b'_1\{ \mathtt{INC}\, z^{+}_0 \} \};$\\
$\hspace{11mm}\mathtt{SWAP} (b'_0,b'_1);$\qquad\loopComment{ Restore the logical pair to $0,1$ }\\
$\hspace{11mm}\mathtt{DEC}\, r_0 ; \mathtt{DEC}\, r_0;$ \qquad \loopComment{ we are driven by the predecessor of $r_0$ }\\
 $\hspace{11mm} \mathtt{FOR}\,r_0\{  \mathtt{SWAP} (b'_0,b'_1); \mathtt{FOR}\,b'_0\{ \mathtt{INC}\, z^{-}_0 \}; \mathtt{FOR}\,b'_1\{ \mathtt{DEC}\, z^{-}_0 \} \};$\\
$\hspace{11mm}\mathtt{SWAP} (b'_0,b'_1);$\qquad\loopComment{ Restore the logical pair to $0,1$ }\\
$\hspace{11mm}\mathtt{INC}\, r_0;$\\
$\hspace{11mm}\}$\qquad \loopComment{ we restore $r_0$ }\\[5mm] 
  \end{tabular}

The control register situation is the following: (i) $r_0$ is unchanged; (ii)  if $r_0$ is even then the pair $b_0,b_1$ is unchanged, otherwise is inverted;
(iii) the pair $b'_0,b'_1$ is unchanged;  (iv) $z^{+}_0$ contains $0$; (v)  $z^{-}_0$ contains $0$.

Finally, we restore the logical $b_0,b_1$ to $0,1$. The trick is that the double of each number is even, so swapping $r_0$ times more the logical pair we can restore its original value.\\

\begin{tabular}{@{\stepcounter{rowcount}{\tiny\therowcount}\hspace*{3mm}}l}
$\mathtt{FOR}\,r_0\{ \mathtt{SWAP} (b'_0,b'_1) \}$\\[5mm] 
 \end{tabular}
%%%%%%%%%%%%%%%%%%%%%%%%%%%%%%%%%%%%%%%%%%%%%%%%%%%%%%%%%%%%%%%%%%%%%%%%%%%%%%%%%%5
%%%%%%%%%%%%%%%%%%%%%%%%%%%%%%%%%%%%%%%%%%%%%%%%%%%%%%%%%%%%%%%%%%%%%%%%%%%%%%%%%%5

\end{document}